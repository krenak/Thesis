\documentclass[a4paper, 12pt]{article}
\usepackage[brazil]{babel}
\usepackage[utf8]{inputenc} % indicação de caracteres especiais - contidos no pt
\usepackage[a4paper, inner=3cm, outer=2cm, top=3cm, bottom=2cm]{geometry}
\usepackage{charter} % fonte utilizada no documento
                    %\usepackage[nf]{coelacanth}% fontes alternativas
                    %\usepackage[T1]{fontenc} % fontes alternativas
\usepackage{microtype} %server para melhorar a justificação
\usepackage{graphicx} % pacote de formatação e inserção de figuras
\usepackage{wrapfig} % pacote de formatação e inserção de figuras
%\usepackage{enumitem} % pacote de formatação e inserção listas enumeradas
\usepackage{fancyhdr} % pacote de formatação e inserção de cabeçalho e rodapé
\usepackage{amsmath} % pacote de formatação e inserção de equações e símb. matemáticos
\usepackage{float} % pacote de formatação de tabelas e figuras
\usepackage{subcaption} % pacote de formatação de tabelas e figuras
\usepackage{multirow} % pacote de formatação de linhas de tabelas
\usepackage{fdsymbol} % adiciona símbolos matemáticos de grandeza
%\usepackage{index} % pacote de formatação de sumário
\usepackage{booktabs} % pacote de formatação e inserção de tabelas
\usepackage{hyperref} % pacote para uso de hyperlinks e referências cruzadas
\usepackage[style=abnt, sorting=ynt]{biblatex} % formatação e configuração das referências
\addbibresource{ref.bib}
\usepackage{csquotes}
\usepackage{caption} % ajuda a alterar o tamanho da fonte das captions nas tabelas e figuras
\usepackage{parskip} % resolve o problema do hbox badness 10000
\usepackage{makecell} % ajuda a ajustar o conteúdo das células das tabelas
\usepackage{pdfpages} % pacote para incluir/anexar páginas em pdf
\usepackage[at]{easylist} % pacote mais fácil pra editar listas e enumerações
\usepackage[singlespacing]{setspace} % pacote de formatação de espaçamento entre linhas
%\usepackage{blindtext} % pacote de inserção de texto randomizado - lorem ipsum
\newenvironment{graph}{\centering}{\par} % novo ambiente para inserção de gráficos
\captionsetup[table]{font=small,labelfont=small} % altera o tamanho da fonte nas tabelas para small (~10pt)
\captionsetup[figure]{font=small,labelfont=small} % altera o tamanho da fonte nas tabelas para small (~10pt)

%-----------cabeçalho e rodapé--------------%
\pagestyle{fancy} % estilo do cabeçalho e rodapé
\fancyhf{}
\rhead{FRAGA, Anderson A.}
\lhead{\leftmark}
\rfoot{Página \thepage}
\renewcommand{\headrulewidth}{2pt}
\renewcommand{\footrulewidth}{1pt}
\setlength{\headheight}{15.7pt} % formatando o headheight, eliminamos o conflito com o pacote fancyhdr
%\setlenght{\parindent}{1.3cm}
%\setlenght{\parskip}{0.2cm}
\DeclareUnicodeCharacter{001A}{~}
\DeclareUnicodeCharacter{03B1}{~} % caracteres que aparecem pra dar erro no doc...

%---------início da dissertação-------------%
\begin{document} %capa e contracapa
\begin{titlepage}
    \begin{center}
        \begin{figure}
            \centering
            \includegraphics[scale=1.15]{F:/Root Files/Academic/masters/_desenvolvimento/_regular/_dissertacao/fase final/_estrutura/_texto/tex/figures/brasao-rep-br.png}
        \end{figure}
        \vspace*{0.1cm}
        \textbf{\large Universidade Federal do Espírito Santo}\\
        \large Centro de Artes\\
        \large Programa de Pós-Graduação em Arquitetura e Urbanismo\\
        \vspace*{3cm}
        \textbf{\large Anderson Azevedo Fraga}\\
        \vspace*{4cm}
        \textbf{Potencial de Adoção do Conceito Zero Energy para Edifícios Comerciais em Vitória-ES}\\
        \vfill % preenche em branco o espaço até o final da página
        Vitória\\
        2020\pagebreak

        \begin{figure}
            \thispagestyle{empty} % limpa cabeçalho, rodapé e número de página
            \centering
            \includegraphics[scale=1.15]{F:/Root Files/Academic/masters/_desenvolvimento/_regular/_dissertacao/fase final/_estrutura/_texto/tex/figures/brasao-rep-br.png}
        \end{figure}
        \vspace*{0.1cm}
        \textbf{\large Universidade Federal do Espírito Santo}\\
        \large Centro de Artes\\
        \large Programa de Pós-Graduação em Arquitetura e Urbanismo\\
        \vspace*{3cm}
        \textbf{\large Anderson Azevedo Fraga}\\
        \vspace*{4cm}
        \textbf{Potencial de Adoção do Conceito Zero Energy para Edifícios Comerciais em Vitória-ES}\\
        \vfill % preenche em branco o espaço até o final da página
        Vitória\\
        2020\pagebreak

    \end{center}
\end{titlepage}\pagebreak
\section*{Agradecimentos}
\vspace*{1.5cm}
\thispagestyle{empty}
\begin{onehalfspace}
À minha mãe, Marilene, e meu pai, Nilson, por todo o suporte até aqui, principalmente emocional. Dedico essa conquista a vocês especialmente, que não tiveram a mesma oportunidade que tive e, mesmo assim, fizeram o possível e o impossível para garantir minha educação. Amo vocês.\\
À Úrsula, por me aguentar durante toda esta difícil caminhada e suportar todas as encrencas e momentos em que o mestrado me fez perder a cabeça. Sempre esteve comigo e eu a agradeço de todo o meu coração. Te amo.\\
À Cris, por me proporcionar a oportunidade de evoluir intelectualmente e como ser humano. Termino esta etapa e devo a ela o voto de confiança depositado em mim. Serei eternamente grato a você, Cris, e espero não a ter decepcionado. Conte comigo sempre.\\
À Jéssica, Lohane, Bruna, Nayara, por todo o suporte acadêmico e pela amizade ímpar que desenvolvemos. Pelos cafezinhos e risadas gostosas durante os momentos mais angustiantes da dissertação. Um grande beijo e estarei aqui sempre que precisarem.\\
Ao Filipe e Lucas, grandes amigos que levarei pra sempre comigo. Sempre tínhamos alguma piada para fazer nas horas mais oportunas, e sempre estávamos dispostos a ajudar uns aos outros.\\
À Júlia e a Lara por me auxiliarem no desenvolvimento das partes braçais. Emprestaram-me a paciência, suas mãos e suas amizades em um momento bastante importante para mim. Grandes amigas, agradeço imensamente a ajuda que tive de vocês neste trabalho. Muito obrigado!\\
À Edna, por todas as orientações, ensinamentos e por todas as contribuições valorosas; à Luciana pela disponibilidade em participar da avaliação do meu trabalho. Espero que tenham gostado.\\
A todos os meus amigos do LPP e professores, vocês moram no meu coração e se precisarem de alguma coisa, é só chamar. A todos os amigos que não foram citados aqui, mas que de alguma forma foram igualmente importantes nesta trajetória, me ajudaram a construir este projeto e hoje sou grato a todos vocês.\\
À CAPES (Coordenação de Aperfeiçoamento de Pessoal de Nível Superior) agradeço a bolsa concedida, o que viabilizou a realização desta pesquisa.\\
Muito obrigado a todos.\\
\end{onehalfspace}
\pagebreak
\input{sections/resumo.tex} % preâmbulo e listas
\begin{abstract}
    \noindent O consumo de energia no uso de edificações vem crescendo gradativamente ao longo
    das últimas décadas, fruto do desenvolvimento industrial e da revolução tecnológica
    que vem acompanhando este movimento. A emissão de gases poluentes e a modificação 
    do clima são consequências desse cenário de desenvolvimento e consumo. Aliado a 
    esses fatores, as edificações contribuem para o agravamento  desse  cenário,  uma  
    vez  que  o  uso  destas  acarreta  em  impactos  negativos significativos ao meio 
    ambiente. Em contraponto, edificações energeticamente eficientes vêm se tornando 
    pré-requisito  para  o  planejamento  de  novos  ambientes  construídos,  
    modificando  a forma  como  a  comunidade  percebe  a  relação entre a edificação  
    e  o  consumo de  energia.  Este trabalho  tem  como  objetivo  estudar  o  
    potencial  de  aplicação  do  conceito Zero Energy  para edificações comerciais, 
    com o intuito de verificar a validade do método para o cenário construtivo 
    brasileiro adotando como estudo de caso  uma edificação  em Vitória (ES). 
    Metodologicamente, este  estudo  foi  desenvolvido  com  base  em  três  grandes  
    etapas,  onde  a  primeira  consistiu  em realizar  o  levantamento  das  
    edificações  dentro  de  um  recorte  territorial  pré-estabelecido, selecionar 
    as características construtivas e arquitetônicas mais frequentes entre elas e 
    construir modelos  representativos  do  cenário  observado;  a  segunda  consistiu  
    em  submeter  os  modelos representativos à simulações computacionais para avaliar 
    o desempenho energético, as possíveis formas  de  eficientização  e  de  produção  
    de  energia;  e  por  fim,  a  terceira  etapa,  na  qual  foi realizada  avaliação  
    dos  resultados  e  da  viabilidade  econômica  de  implantação  do  sistema  
    de produção de energia. Os resultados mostraram que as estratégias de implementação 
    de sistemas de condicionamento de ar, de equipamentos e iluminação mais eficientes 
    são muito importantes para  a  economia  de  energia.  É  perceptível  que  a  
    proposição  de  soluções  construtivas  e arquitetônicas  mais  eficientes  em  
    relação  ao  desempenho  energético  associado  a  técnicas  de obtenção  de  
    energia  podem  resultar  em  uma  edificação  com  o  balanço  energético  nulo  
    ou próximo ao nulo. Esses resultados indicam que a adoção desse conceito para novas 
    edificações é factível e cada vez mais acessível à comunidade.
    \paragraph{Palavras-chave:} zero energy buildings; balanço energético nulo; edifício de escritório 12 
\end{abstract}\pagebreak
\section*{Lista de Abreviaturas}
\vspace*{1.5cm} % verificar se é permitido manter este espaçamento
\thispagestyle{empty}
    \begin{onehalfspace}
        ABIVIDRO – Associação Técnica Brasileira das Indústrias Automáticas de Vidro\\
        ABNT – Associação Brasileira de Normas Técnicas\\
        ANAEEL – Agencia Nacional de Energia Elétrica\\
        ARSP – Agência de Regulação de Serviços Públicos do Espírito Santo\\
        ASHRAE – American Society of Heating, Refrigerating and Air-conditioning Engineers\\
        ASPE – Agência de Serviços Públicos de Energia do Espírito Santo\\
        CB3E – Centro Brasileiro de Eficiência Energética em Edificações\\
        CBCS – Conselho Brasileiro de Construção Sustentável\\
        CNI – Confederação Nacional da Industria\\
        DOE – Department of Energy of United States of America\\
        EPE – Empresa de Pesquisa Energética\\
        IBGE – Instituto Brasileiro de Geografia e Estatística\\
        IEA – International Energy Agency\\
        INCAPER – Instituto Capixaba de Pesquisa, Assistência Técnica e Extensão Rural\\
        INMET – Instituto Nacional de Meteorologia\\
        INMETRO – Instituto Nacional de Metrologia, Qualidade e Tecnologia\\
        ONS – Operador Nacional do Sistema Elétrico\\
        PMV – Prefeitura Municipal de Vitória\\
        SIN – Sistema Interligado Nacional\\
        SINDUSCON – Sindicato da Indústria da Construção Civil do Espírito Santo\\
        UNDP – United Nations Development Programme\pagebreak
    \end{onehalfspace}
\listoffigures \thispagestyle{empty} \pagebreak
\section*{Lista de Gráficos}
\vspace*{1.5cm} % verificar se é permitido manter este espaçamento
\thispagestyle{empty}
    \begin{onehalfspace}
        Gráfico 1 - Relação entre demanda energética e créditos para edificação NZEB.\par
        Gráfico 2 - Oferta interna de energia elétrica no Brasil (a) e a 
        participação setorial de consumo de eletricidade (b).\par
        Gráfico 3 - Evolução da geração de energia elétrica por fonte renovável e não-renovável no Espírito Santo.\par
        Gráfico 4 - Consumo de energia elétrica no Espírito Santo por classe.\par
        Gráfico 5\par
        Gráfico 6\par
        Gráfico 7\par
        Gráfico 8\par
        Gráfico 9\par
        Gráfico 10\par
    \end{onehalfspace}\pagebreak
\listoftables \thispagestyle{empty} \pagebreak
\tableofcontents \thispagestyle{empty} \pagebreak

\section{Introdução}
\linespread{2.5}
     A energia elétrica é um recurso essencial para o desenvolvimento econômico de um país, 
     para a qualidade  de  vida  da  população  e para  a  manutenção  do meio  ambiente
     por  meio  de  seu  uso eficiente \cite{Fonseca2016}. A importância do uso racional
     e eficiente deste recurso torna imprescindível a conservação e redução do seu 
     desperdício para a sustentabilidade do ambiente em que se vive.\\ Desde  a  crise  do  
     petróleo,  ocorrida nos  anos  de 1970,  a eficiência  energética tem  a  função  de 
     proporcionar  condições  para  suprir  à  demanda  futura  de  energia.  Esta  gestão  
     eficiente  do consumo  de  energia  é  essencial  para  reduzir  o  impacto  energético  
     de  setores  como  o  de edificações, o qual consome de 36 a 40\% da energia total final
     global. A necessidade de expansão dos   setores   econômicos   provoca   demanda   por
     energia   elétrica.   Esta   busca   resulta   em desperdícios oriundos da falta de 
     políticas públicas efetivas, de investimento em tecnologia e de fiscalização   sobre 
     o   consumo   deste   insumo \cite{InternationalEnergyAgency-IEA2019,InternationalEnergyAgency-IEA2019a,UnitedNationsEnvironmentProgramme-UNEP2019,UnitedNations2017}.\\
     Em contraponto à demanda e ineficiência energética, as edificações comerciais, em particular 
     as de  escritório,  podem  desempenhar  funções  estratégicas  como  minimizar  o  uso  
     energético  e produzir eletricidade, aproximando ou equalizando a zero a razão entre a 
     produção e o consumo de energia. Estas edificações são denominadas edificações com balanço 
     energético nulo, ou Zero Energy Buildings  –  ZEB \cite{Crawley2009,Torcellini2006,Kurnitski2011,Kurnitski2015,Torcellini2015}.\\
     Calcula-se  que  a  tendência  de  adoção  desta  forma  de  projetar edificações crescerá 
     até 2050, haja vista que a publicação de normas e regulamentações acerca do tema vêm 
     crescendo ao redor do mundo \cite{UnitedNationsEnvironmentProgramme-UNEP2019}. Com  a  introdução  de  uma  ZEB,  a  
     exploração  de  recursos  renováveis  complementares  como  a energia solar, e a utilização 
     de tecnologia solar fotovoltaica, surgem como opção para minimizar as    consequências    
     negativas    causadas    por    condições    climáticas,    de    infraestrutura    
     e socioeconômicas adversas \cite{Pikas2014,Pikas2017}.\\
     A quantidade de radiação solar recebida no Brasil, por exemplo, alcança a ordem de 1.013 MWh, 
     nível acima de países com grande capacidade de geração de energia solar. Este fato 
     torna viável a  adoção  deste  recurso  como  forma  de  reduzir  o  uso  de  fontes  
     de  energia  fósseis  e  como economia  no  consumo  de  água.  A  disponibilidade  de  energia
     solar  no  Brasil  alcança  cerca  de 6,5 kWh/m²  ao  ano  e,  no  Espírito  Santo,  
     entre  4,8  a  5,2 kWh/m²  ao  ano  (AGÊNCIA...,  2019; DIDONÉ, 2014; INTERNATIONAL..., 2018).\\ 
     A relação entre fontes da matriz energética brasileira é composta por 45\% de fontes renováveis
     e 55\% de fontes não-renováveis de energia. Há, ainda, a previsão de que a parcela de geração
     de eletricidade  por  meio  de  fontes  renováveis,  que  em  2018  era  de  83,3\%,  atinja  
     87\%  até  2040 (EMPRESA DE PESQUISA ENERGÉTICA, 2017b). No entanto, há controvérsias na 
     classificação da fonte hidrelétrica como renovável, considerando a dependência da água, 
     dos ciclos de chuva e dos impactos gerados na construção das usinas (LEME, 2012).\\
     Dentro  do  contexto  de  segurança  energética,  a  crise  brasileira,  ocorrida  em  2001,  
     provocou mudanças no planejamento do fornecimento de energia elétrica, com o posterior 
     surgimento de medidas  atenuantes  às dificuldades  de  cunho  ambiental  e  de  infraestrutura  
     da  época.  Em  seu ápice, no ano de 1999, o país passou pelo período popularmente denominado “apagão”, 
     o qual representou  a  falta  de  fornecimento  em  70\%  do  território  nacional.  
     O  consumo  de  energia elétrica, entre os anos de 1990 e 2000, sofreu aumento de 49\%, 
     enquanto a capacidade instalada foi expandida em 35\%, ocasionando o descompasso entre 
     consumo e fornecimento nesta época (CONEJERO; CALIA; SAUAIA, 2016; TOLMASQUIM, 2000).\\ 
     Verifica-se  também  que  a  centralização  de  geração  de  energia  representa  fragilidade  
     para  o modelo  de  comercialização  utilizado  no  Brasil  (PINTO;  MARTINS;  PEREIRA,  2017). 
     Logo,  as mudanças observadas sobre a incorporação e aumento da participação de fontes 
     renováveis de energia  no mix  energético  brasileiro,  além  da  inserção  de  edificações  
     com  alto  desempenho energético como as ZEB’s, pode servir como uma das respostas necessárias 
     visando a segurança energética  e  elevando  a  confiabilidade  do  SIN  -  Sistema  Interligado
     Nacional  (EMPRESA  DE PESQUISA ENERGÉTICA, 2017a).\\ No âmbito estadual, o Espírito Santo 
     vem apresentando redução na produção de energia limpa nos últimos 8 anos, quando comparado 
     proporcionalmente ao consumo de fontes tradicionais. Existe ainda a parcela de geração de 
     energia elétrica oriunda de fontes não-renováveis de energia, como usinas termelétricas, 
     correspondendo a 65\% de toda a capacidade instalada em operação do  Espírito  Santo,  
     restando  35\%  de  fontes  renováveis,  composta  por  usinas  hidrelétricas,  com 
     participação  de  34\%,  e  geradores  de  energia  solar  fotovoltaica,  com  1\%  
     (AGÊNCIA...,  2019; ENERGIAS DE PORTUGAL, 2017).\\ Sabe-se  que  as  edificações  comerciais  
     no  Brasil  utilizam  majoritariamente  a  eletricidade,  em especial  as  edificações  de  
     escritório,  com  aproximadamente  92\%  do  consumo  total,  enquanto edificações  de  
     uso  não-comercial  utilizam  fontes  de  energia  diversificadas.  Assim,  a  redução 
     superficial de consumo de energia destas edificações nos últimos 5 anos, quando comparado 
     com os  outros  setores  econômicos,  é  da  ordem  de  2,74\%,  o  que  reforça  a  
     importância  em proporcionar  o  aumento  da  eficiência  energética  para  o  segmento  
     de  edificações  comerciais (AGÊNCIA..., 2018, 2019; EMPRESA..., 2018).\\ À vista  destes  
     dados,  a presente pesquisa objetivou avaliar  o  potencial de adoção  do  conceito Zero  
     Energy   enquanto   uma  das  possíveis   estratégias  visando   a   redução   dos  
     problemas energéticos e ambientais relacionados às edificações de escritório, como forma 
     de contribuir aos novos mecanismos para planejar e projetar o ambiente construído. Da mesma 
     forma, busca-se evidenciar  medidas que propiciem  a  redução  do impacto  do  consumo  
     energético  vinculado  ao uso destes edifícios.  % corpo da dissertação
\subsection{Questionamentos}
\begin{onehalfspace}
        Considerando que:
    \begin{itemize}
        \item Existe uma parcela de energia elétrica proveniente de fontes fósseis no 
        Estado e que este quadro pode se agravar ao longo do tempo, visto a falta de 
        representatividade das fontes alternativas de geração de energia na matriz 
        energética do Espírito Santo;
        \item A demanda energética das edificações comerciais poderia  ser reduzida,  se 
        desde a fase projetual fosse considerada as potencialidades e restrições 
        ambientais do entorno;
        \item A micro e mini geração de energia elétrica é uma possibilidade que deve ser 
        incrementada no Brasil, principalmente considerando o potencial de queda de custos 
        na implementação de fontes de geração de energia elétrica descentralizada;
        \item Os   componentes   da   edificação,   como   envoltória   e   os   sistemas   
        de   conforto termoenergético, são subutilizados ou mal dimensionados no âmbito do 
        recorte territorial considerado, acarretando a baixa eficiência energética do 
        edifício.
    \end{itemize}
    A pergunta foi estabelecida a partir do seguinte questionamento: 
    considerando as características do  ambiente  construído  no  âmbito  da  Região  
    Metropolitana  da  Grande  Vitória,  é  possível desenvolver edificações cujos valores 
    de demanda e produção de energia elétrica resultem em nulo ou quase nulo?
\end{onehalfspace}
\subsection{Objetivos}
\begin{onehalfspace}
Diante do exposto, o objetivo principal desta pesquisa foi avaliar 
a aplicabilidade do conceito Zero Energy  em  edificações  comerciais,  
especificamente  de  escritório,  com  estudo  de  caso  para  o 
município  de  Vitória  (ES).  Este  setor  foi  selecionado  por  
apresentar  padrões  amplamente difundidos  de  uso  e  ocupação,  
de  equipamentos  e  da  conformação  do  espaço,  que  não  
só viabilizam  a  adoção  de  tecnologias  de  produção  de  energia  
elétrica,  como  também  facilita  a análise de desempenho termoenergético, 
quando comparado às edificações do setor industrial e do setor residencial.\newline
Visando alcançar os resultados esperados, foram definidos os seguintes 
objetivos específicos:
\begin{itemize}
    \item Identificar os parâmetros aplicáveis às edificações de escritório 
    inerentes ao conceito Zero Energy e Near Zero Energy, assim como sua 
    viabilidade econômica;
    \item Mapear e diagnosticar as edificações comerciais concluídas a partir 
    de 2003, em Vitória – ES,  como  recorte  da  pesquisa,  com  a  
    caracterização  da  envoltória  e  dos  sistemas  de iluminação e 
    condicionamento de ar;
    \item Identificar métodos para a geração energética e formas de 
    racionalização do consumo de energia, estabelecendo diretrizes para 
    situações semelhantes.
\end{itemize}\pagebreak
\end{onehalfspace}
\section{Referencial Teórico}
\begin{onehalfspace}
    Este capítulo trata das referências utilizadas para o desenvolvimento da 
    pesquisa. O referencial teórico foi organizado, em grande parte, com base 
    nos estudos de \textcite{Didone2014}, \textcite{Didone2014a}, 
    \textcite{Kurnitski2011a}  e \textcite{Torcellini2006}.  Estas  referências  tratam  
    das definições sobre \textit{Zero Energy Buildings}, sobre conforto ambiental por 
    meio de estratégias passivas e ativas, a eficiência energética voltada a 
    edificações e a produção de energia elétrica por meio de tecnologias 
    fotovoltaicas. Estes autores foram escolhidos por serem referências 
    recorrentes em pesquisas posteriores as publicações citadas e apresentarem 
    metodologias e embasamentos teóricos importantes para o desenvolvimento de 
    pesquisas sobre o tema \textit{Zero Energy}. Da mesma forma, foram utilizados 
    conceitos abordados pela Instrução Normativa Inmetro para Classe de 
    Eficiência Energética de Edificações Comerciais, de Serviços e Públicas 
    – INI-C (2018) e pelas  normas  NBR  15.220  (2019),  e \textit{The American 
    Society of Heating, Refrigerating and Air-Conditioning 
    Engineers} – ASHRAE \textit{Standard} 55 (2017), 140 (2017) e 90.1 (2010). 
    O contexto socioeconômico e climático de Vitória, assim como a caracterização 
    da tipologia de referência  para  suporte  metodológico  das  simulações  
    e  constituição  dos  modelos  genéricos foram abordados neste capítulo.

\subsection{Introdução ao conceito \textit{Zero Energy}}
Define-se que um edifício \textit{Zero Energy} – ZEB, ou em português, balanço 
energético nulo, é uma edificação  energeticamente  eficiente  onde,  
considerada  a  fonte  energética,  a  energia  elétrica fornecida pela 
concessionária é anualmente menor ou igual à quantidade de energia renovável 
exportada pela edificação para a rede 
\cite{Torcellini2006,U.S.DepartmentofEnergy-USDOE2012,U.S.DepartmentofEnergy-USDOE2015}.\vspace{0.5cm} \newline
\textcite{Domingos2014} define  que  o  balanço  energético  nulo  pressupõe  
uma  arquitetura adequada ao uso de elementos construtivos e equipamentos 
de alta eficiência energética, aliado ao desempenho da fonte geradora de 
energia elétrica a partir de fontes renováveis.\newline A redução do consumo de 
energia em novas edificações ou em processo de melhoria pode ser alcançada 
por meio de projetos integrados à tecnologias de produção de energia, com 
adoção de soluções   energeticamente   eficientes,   e   por   programas   
de   economia   de   energia \cite{U.S.DepartmentofEnergy-USDOE2015}.\vspace{0.5cm} \newline
\textcite{Torcellini2006} estabelecem quatro definições acerca das formas de 
se atingir o ZEB em edificações  de  baixo  consumo  de  energia,  ou  
comumente  denominadas \textit{Low-Energy Buildings}. Dentre as formas estudadas estão: 
\begin{itemize}
    \item \textit{Zero Site Energy}, ou energia local zero ou ainda energia da 
    edificação \cite{U.S.DepartmentofEnergy-USDOE2015}, onde é avaliada a 
    potencialidade de produção de energia elétrica para a  edificação  
    utilizando  os  recursos  presentes  no  local,  ou on-site,  onde  o  
    edifício  está implantado. É minimamente avaliado o consumo dos 
    sistemas de condicionamento de ar, de aquecimento quando existente, 
    ventilação, cargas de equipamentos e de sistema de iluminação;
    \item \textit{Zero Source Energy},  ou  fonte  de  energia  zero,  trata-se  do  
    conceito  onde  é  levado  em consideração toda a cadeia de produção 
    total anual de energia utilizada pela edificação e de consumo de 
    energia primária do edifício. Esta avaliação leva em conta a 
    eletricidade, combustíveis utilizados em processamento e transporte de 
    materiais e componentes para o local da edificação, entre outros aspectos;
    \item \textit{Zero Energy Cost}, ou custo de energia zero, é avaliada a razão, 
    no mínimo igual, entre a quantidade total de dinheiro que é arrecadado 
    com a venda de energia produzida on-siteà  concessionaria,  e  a  
    quantidade  total  paga  pela  utilização  de  serviços  e  energia 
    consumida ao longo do ano; e 
    \item \textit{Zero Energy Emissions},  ou  emissão  zero,  onde  a  edificação  
    produz  uma  quantidade  de energia  renovável  livre  de  emissão  
    de  GEE  ao  menos  igual  a  quantidade  de  energia consumida 
    proveniente de fontes de energia emissoras de GEE. 
\end{itemize}
As definições descritas para o ZEB, que avaliam fontes de energia variadas, 
os custos atrelados à implantação da edificação, a avaliação do ciclo de vida 
dos materiais e componentes, assim como emissões de GEE, demandam pesquisas e 
desenvolvimento de metodologia particular para cada meio de avaliação. 
Portanto, para este trabalho foi estabelecido o recorte de avaliação para 
as edificações utilizando recursos on-site, classificado como \textit{Zero Site Energy}. 
Tal recorte justifica-se em função da exclusividade dada pela definição 
sobre a utilização de fontes de energia renováveis locais, 
disponíveis no sítio onde a edificação foi implantada, utilizada para pesquisa 
em detrimento das outras definições que tratam de formas de avaliação não 
abordadas nesta pesquisa.\vspace{0.5cm} \newline
\textcite{Didone2014} e \textcite{Athienitis2015} definem que o planejamento de 
uma edificação \textit{Zero Energy} considera a integração de estratégias energéticas 
e soluções passivas; de otimização da edificação em seu projeto, execução e 
operação; e a utilização de tecnologia de produção de energia solar 
fotovoltaica, entre outras formas de produção de energia, considerando a 
forma da edificação e soluções que aproveitem a disponibilidade de energia 
elétrica solar para o meio urbano, aquecimento solar e luz natural.\vspace{0.5cm} \newline
A \textcite{InternationalEnergyAgency-IEA2014} publicou um estudo detalhado de 
30 edificações \textit{Zero Energy} em vários países e em diversas situações climáticas. 
O \textit{“Towards Net Zero Energy Solar Buildings: A review of 30 Net ZEBs case studies”} 
foi parâmetro para o desenvolvimento de referências para o planejamento de edificações 
energeticamente nulas. Este estudo concluiu que é possível atingir o balanço energético 
nulo para diversos usos residenciais e comerciais da realidade construtiva americana.\vspace{0.5cm} \newline
A respeito da nomenclatura utilizada para descrever uma edificação que atinge 
o equilíbrio entre consumo e produção de energia, na série de Guias Avançados de Projeto 
Energético – AEDG \cite{AmericanSocietyofHeatingRefrigeratingandAir-ConditioningEngineers-ASHRAE2019} 
há a utilização do termo \textit{Zero Energy} em oposição aos termos \textit{Net Zero Energy} 
e \textit{Zero Net Energy} em consonância com os termos utilizados pelo Departamento 
de Energia dos Estados Unidos, da mesma forma para as políticas federais e municipais 
de desempenho energético em vigor. O Departamento de Energia dos Estados Unidos também 
utiliza o termo \textit{Zero Energy} justificando que a inserção de “\textit{Net}” não 
adiciona significado substancial a expressão \cite{U.S.DepartmentofEnergy-USDOE2015a}. 
Este trabalho utilizou a nomenclatura em consonância com a AEDG da norma americana ASHRAE 
e demais artigos científicos que adotam \textit{Zero Energy} como balanço energético nulo.

\subsubsection{\textit{Near Zero Energy}}
 Baseado na norma europeia EN 15603:2008, a definição sobre o conceito proposta por
 \textcite{Kurnitski2011a} para edificações \textit{Near Net Zero Energy}, nZEB, e 
 em português, próximo ao balanço energético nulo, se apoia na premissa do aproveitamento 
 máximo de recursos para produção de energia, implementando mecanismos à edificação 
 de forma que este aproveitamento aconteça, e a utilização à nível ótimo da energia 
 primária, para um consumo maior que 0 kWh/m² ao ano. Segundo os autores, são pontos 
 importantes para atender a definição do conceito:
\begin{itemize}
    \item O custo otimizado e considerável aproveitamento técnico do uso da energia primária; e
    \item A porcentagem de energia primária coberta pela geração de energia proveniente de 
    fontes renováveis.
\end{itemize}
A nomenclatura adotada, tal qual para \textit{Zero Energy}, foi \textit{Near Zero Energy}, 
em concordância com a nomenclatura encontrada nos estudos consultados e com o termo 
\textit{Zero Energy} adotado anteriormente \cite{AmericanSocietyofHeatingRefrigeratingandAir-ConditioningEngineers-ASHRAE2019}.
Esta variação do conceito de balanço energético nulo é adotada como parâmetro de avaliação 
para a presente pesquisa, uma vez que se adequa ao cenário energético e tecnológico do 
recorte territorial estabelecido, ou seja, a Região Metropolitana da Grande Vitória, no 
Espírito Santo.

\subsubsection{Métrica para balanço energético nulo}
Um edifício energeticamente balanceado produz, consome e, eventualmente, exporta energia 
para a concessionária quando as condições climáticas e energéticas são favoráveis para 
este cenário. A avaliação depende de definição do espaço utilizado para geração de energia, 
\textit{on-site} ou \textit{off-site}, e do tempo avaliado deste processo, horário, 
mensal ou, mais comumente, anual \cite{AmericanSocietyofHeatingRefrigeratingandAir-ConditioningEngineers-ASHRAE2019}.\vspace{0.5cm} \newline
Métodos de avaliação foram desenvolvidas para contemplar as variações de se atingir o 
balanço energético nulo como demanda/geração de energia e exportação/importação de energia. 
O primeiro método é direcionado ao \textit{Zero Site Energy}, o que permite avaliar 
anualmente as opções de produção de energia elétrica \textit{on-site} e a demanda de 
energia calculada. O segundo método é normalmente aplicado ao \textit{Zero Source Energy}, 
onde são balanceadas as fontes de energia, carga de energia e interação com a rede da 
concessionária \cite{Didone2014a}.\vspace{0.5cm} \newline
Para tal, é determinada a interação energética entre a concessionária e a edificação, 
como observado na Figura \ref{Figura 1}.\pagebreak

\begin{figure}[h]
   \centering
   \caption{\small Diagrama de interação entre a edificação e as variáveis externas.}
   \includegraphics[width=0.85\textwidth]{figures/esquema_kurnitski _2011.png}
   \par \small Fonte: adaptado de Kurnitski et al. (2011, tradução nossa)
   \label{Figura 1}
\end{figure}

\noindent Foram desconsideradas as medidas que envolviam sistema de aquecimento, por não condizer 
com a realidade observada no recorte territorial e por ser uma medida de inclusão facultativa 
em regulamentações nacionais \cite{InstitutoNacionaldeMetrologiaNormalizacaoeQualidadeIndustrial-INMETRO2018,InstitutoNacionaldeMetrologiaNormalizacaoeQualidadeIndustrial-INMETRO2018a}.
De forma a estabelecer a métrica necessária para a avaliação do balanço energético da edificação, 
devem ser conhecidos os componentes de uso final de energia, a energia primária total utilizada 
pela edificação, os custos com o consumo de energia elétrica, e a quantidade de energia 
exportada à concessionária. Da mesma forma, devem ser definidos o período de balanço a ser 
analisado e eventuais créditos provenientes de fornecimento de energia à concessionária. 
Esta relação é ilustrada na Figura \ref{Figura 2}.\vspace*{-0.4cm}

    %\begin{figure}
    %    \centering
    %    \caption{\small Gráfico de relação entre demanda energética e alimentação de créditos para edificação NZEB.}
    %    \includegraphics[width=0.3\textwidth]{figures/esquema_iea_2014-2.png}
        %\small Fonte: adaptado de Kurnitski et al. (2011, tradução nossa)
    %    \label{Grafico 6}
    %\end{figure}

    \begin{figure}[ht]
        \centering
        \caption{\small Gráfico de relação entre demanda energética e créditos para edificação NZEB.}
        \includegraphics[width=0.3\textwidth]{figures/esquema_iea_2014-2.png}
        \par \small Fonte: adaptado de Kurnitski et al. (2011, tradução nossa)
        \label{Figura 2}
    \end{figure}
\subsection{Cenários energéticos e a matriz elétrica brasileira}
\begin{onehalfspace}
    O cenário energético mundial vem apresentando progressos quanto ao desenvolvimento da 
    eficiência energética e da busca de fontes limpas e renováveis de energia. A criação de 
    políticas de redução de consumo energético, assim como a promoção de congressos, eventos 
    e demais incentivos à pesquisa e desenvolvimento acadêmico apontam melhorias neste âmbito 
    da energia \cite{InternationalEnergyAgency-IEA2014}.\vspace{0.3cm} \newline
    Estudo desenvolvido pela \textcite{UnitedNations2017} aponta que 103 países definiram 
    a eficiência energética e uso de energias renováveis como parte importante do seu 
    planejamento estratégico, e destes, 79 são países emergentes e em desenvolvimento. 
    Constata-se, ainda, que o consumo de energia poderia ter sido 12\% maior em 2017 caso as 
    políticas públicas mencionadas anteriormente não tivessem sido implementadas desde o ano 
    2000 \cite{InternationalEnergyAgency-IEA2019b}.\vspace{0.3cm} \newline
    Entre os países emergentes e em desenvolvimento, nota-se que há um esforço para redução 
    de consumo de energia, o que reflete em fatores como o aumento da segurança energética, 
    aumento na competitividade industrial, redução de emissão de poluentes e da degradação 
    ambiental, expansão ao acesso de energia, além da indução ao crescimento econômico 
    \cite{BancoMundial2018}. Entretanto, de acordo com \textcite{Abramovay2010,Abramovay2014}, 
    a matriz energética mundial ainda será predominantemente composta por fontes fósseis de 
    energia até meados do século XXI.\vspace{0.3cm} \newline
    No Brasil, a taxa de consumo energético, assim como em outros países, é definida pelo 
    aquecimento econômico e cenários estabelecidos para o desenvolvimento esperado para o país. 
    Nesse sentido, espera-se que o Brasil, até 2026, apresente crescimento econômico e, 
    concomitantemente, consuma energia de forma modesta. Projeta-se que este crescimento seja 
    da ordem de 1,9\% ao ano até a metade da década analisada, com variações que definem o 
    crescimento do consumo em 2,3\% anuais, indicando otimismo para o setor de energia 
    brasileiro \cite{EmpresadePesquisaEnergetica-EPE2017,EmpresadePesquisaEnergetica-EPE2017a}.\vspace{-0.6cm}\newline \enlargethispage{0.25\baselineskip}
    
        \begin{graph}
            \par \small Gráfico 2 - Oferta interna de energia elétrica no Brasil (a) e a participação setorial de consumo de eletricidade (b).
            \begin{minipage}[ht]{1\textwidth}\centering
                \includegraphics[width=0.7\textwidth]{graphs/graph1.png}            
            \end{minipage}
            \begin{flushleft}
                \par \small Fonte: adaptado de EPE (2019).            
            \end{flushleft}
        \end{graph}\pagebreak
    
    \noindent Em 2018, a geração hídrica respondeu por 66,6\% da oferta interna entre as fontes de
    produção de energia elétrica no Brasil, seguido do gás natural, com 8,6\%, da biomassa, 
    com 8,5\% e de outras fontes, 16,3\%, como mostrado no Gráfico 1. Deste modo, 
    as centrais hidráulicas de serviço público e de autoprodução contribuíram para expansão 
    da capacidade total instalada de geração de energia elétrica, com acréscimo de 3.864 MW 
    dos 5.728 MW, ou 67,5\% do total adicionado \cite{EmpresadePesquisaEnergetica-EPE2019}.\vspace*{0.3cm} \newline
    \noindent Esta contribuição para a expansão energética, além da representação na oferta interna de 
    energia elétrica, demonstra a importância da fonte para o país. Entretanto, fontes renováveis 
    de energia elétrica, como a solar e a eólica, representam uma alternativa a momentos de 
    condição desfavorável para oferta hídrica, como registrado em 2017, quando foi verificada 
    queda de 3,4\% da energia hidráulica disponibilizada em relação ao ano anterior \cite{EmpresadePesquisaEnergetica-EPE2018}.\vspace*{0.3cm} \newline
    A representatividade nacional da fonte solar na geração de energia elétrica aumentou em 316,1\% 
    entre os anos de 2017 e 2018, crescendo de 832 GWh para 3.431 GWh. A potência instalada solar 
    fotovoltaica atingiu 1.798 MW em 2018, 47,99\% potência a mais disponível em relação ao ano 
    anterior, com 935MW \cite{EmpresadePesquisaEnergetica-EPE2019,EmpresadePesquisaEnergetica-EPE2019a}.\vspace{0.3cm} \newline
    Vale ressaltar o avanço na oferta de energia elétrica proveniente de micro e mini geração 
    distribuída, saltando de 359 GW, em 2017, para 828 GW, em 2018, resultando em um aumento de 131\%. 
    A contribuição para este crescimento se deu majoritariamente pela energia solar, com 63,5\%, 
    enquanto as fontes hídrica, gás natural, eólica e outras fontes renováveis contribuíram, respectivamente, 
    em 19,1\%, 1,8\%, 1,7\% e 13,9\% \cite{EmpresadePesquisaEnergetica-EPE2019}.\vspace{0.3cm} \newline
    O entendimento sobre a matriz elétrica nacional e a disponibilidade de fontes de geração de 
    energia elétrica serve como importante base para a definição de medidas de produção de energia, 
    assim como para o balanço energético das edificações. Este mapeamento de fontes energéticas 
    indica a potencialidade de geração de energia descentralizada, reforçado pelo crescimento 
    no número de geradoras de energia solar fotovoltaica no país e pelo avanço da oferta de energia 
    elétrica oriunda de micro e mini geração distribuída \cite{EmpresadePesquisaEnergetica-EPE2019,EmpresadePesquisaEnergetica-EPE2019a,Pereira2017}.\vspace{0.3cm} \newline
    Assim, a utilização de fontes renováveis de energia representa um aspecto importante para as 
    edificações Zero Energy aplicadas ao cenário brasileiro. Visto que é significativo o potencial 
    de uso da energia solar como fonte de produção de energia renovável, este recurso pode resultar 
    em reduções importantes no consumo de energia para uma parcela de consumidores dos setores 
    industrial, residencial e comercial, com quase 79,8\% de participação no consumo de energia 
    elétrica, como apresentado no Gráfico 2b \cite{Cronemberger2012,EmpresadePesquisaEnergetica-EPE2019,Sorgato2018,Sudhakar2019}.\vspace{0.3cm} \newline
    A composição da matriz elétrica do Espirito Santo mostra que a predominância da geração de 
    energia elétrica por fonte é termelétrica, ou térmica de gases de processo, em 2018, com 35,10\%, 
    enquanto a parcela de participação da fonte hidrelétrica é de 24,72\% \cite{AgenciadeRegulacaodeServicosPublicosdoEspiritoSanto-ARSP2018}.\vspace{0.3cm} \newline
    A evolução na geração de energia elétrica aponta que as fontes renováveis de energia estão 
    regredindo em participação, como mostrado no Gráfico 3 que entre os anos de 
    2009 a 2012, compunham mais da metade da geração de energia, e atualmente estão em um 
    patamar de 35\%.%\vspace*{-0.4cm}

        \begin{graph}
            \par \small Gráfico 3 - Evolução da geração de energia elétrica por fonte renovável e não-renovável no Espírito Santo.
            \begin{minipage}[ht]{1\textwidth}\centering
                \includegraphics[width=0.8\textwidth]{graphs/graph2_geracao_de_energia_eletrica_no_es-arsp_2018.png}
            \end{minipage}
            \begin{flushleft}
                \par \small Fonte: adaptado de ARSP (2018).
            \end{flushleft}
        \end{graph}

    \noindent O Espírito Santo conta ainda com a uma parcela geradora solar fotovoltaica, inaugurada em 2016 
    com capacidade de 1GW, de 2.820 usinas fotovoltaicas, configurando 28,8 MW de potência instalada 
    e geração de 8 GW \cite{AgenciadeRegulacaodeServicosPublicosdoEspiritoSanto-ARSP2018}.\vspace{0.3cm} \newline
    Observa-se que houve uma sensível variação no consumo de energia elétrica no Espírito Santo 
    entre os anos de 2016 e 2018. Houve o aumento de consumo das classes residencial e industrial, 
    compensado pela redução das classes comercial e rural, como observado no Gráfico 4. O consumo de 
    energia elétrica no Estado, em 2018, foi predominantemente da classe industrial, perfazendo 
    40,2\%, seguido pela classe residencial, com 24,1\%, comercial, consumindo 17,3\%, rural, 
    com 9,3\% e outros consumidores, com 9,1\% \cite{AgenciadeRegulacaodeServicosPublicosdoEspiritoSanto-ARSP2019}.
    
        \begin{graph}
            \par \small Gráfico 4 - Consumo de energia elétrica no Espírito Santo por classe.
            \begin{minipage}[ht]{1\textwidth}\centering
                \includegraphics[width=0.7\textwidth]{graphs/graph3_consumo_de_energia_eletrica_no_es_por_classe-arsp_2019.png}
            \end{minipage}
            \begin{flushleft}
                \par \small Fonte: adaptado de ARSP (2019).
            \end{flushleft}
        \end{graph}

    \subsubsection{Potencial de geração de energia solar no Brasil e no Espírito Santo}
    A avaliação do potencial de geração de energia estritamente solar se justifica, principalmente, 
    pela natural abundância do recurso disponível em território nacional \cite{Pereira2017}. Além 
    da disponibilidade de energia solar, a versatilidade da tecnologia fotovoltaica para adaptar-se 
    ao meio urbano e a redução de custo de instalação e manutenção são fatores importantes que 
    tornam a tecnologia acessível em detrimento de outras formas de geração de energia provenientes 
    de fontes renováveis, tais como a eólica, a geotérmica, a maremotriz, a biomassa, entre outras 
    \cite{AgenciadeRegulacaodeServicosPublicosdoEspiritoSanto-ARSP2019,Didone2014,Didone2014a,InternationalEnergyAgency-IEA2019b,UnitedNationsEnvironmentProgramme-UNEP2019}.\vspace{0.3cm} \\
    A oferta de energia solar no país varia de acordo com a região analisada, dada as dimensões 
    continentais do país. Segundo \textcite{Pereira2017}, a região Nordeste apresentou a menor 
    variabilidade interanual de energia solar, indicando maior estabilidade na produção de energia 
    solar, com valores entre 5,39 e 5,59 kWh/m². Já a região Sudeste apresentou a maior variabilidade 
    interanual, com médias entre 4,97 e 5,11 kWh/m² entre os anos de 2005 e 2015. A abrangência deste 
    rendimento energético anual pode ser observada na Figura \ref{Figura 2}.\pagebreak
    
        \begin{figure}[ht]
            \centering
            \caption{\small Rendimento solar anual brasileiro.}
            \includegraphics[width=0.9\textwidth]{figures/fig3-niveis_de_irradiacao_por_regiao_brasileira_pereira_et_al_2017.png}
                \begin{flushleft}
                    \par \small Fonte: adaptado de Pereira et al. (2017).
                \end{flushleft}
            \label{Figura 2}
        \end{figure}\vspace*{-0.4cm}
    \noindent Os resultados de \textcite{Cronemberger2012}, após a conclusão do estudo em 78 cidade brasileiras, 
    apontaram que o Brasil é caracterizado como um país em baixa latitude e com alta disponibilidade 
    e uniformidade de radiação solar. Esta conclusão apontou a potencialidade das cidades brasileiras 
    para a geração de energia solar, tanto em superfícies planas como nas coberturas quanto em superfícies 
    verticais como as fachadas.\vspace{0.3cm} \newline
    \textcite{Pereira2017} complementam acerca do potencial brasileiro em gerar energia solar. Os autores 
    mencionam que mesmo nos locais menos ensolarados do Brasil, como as regiões Sul e Norte, apresentadas 
    na Figura \ref{Figura 3}, é possível gerar mais eletricidade solar do que no local mais ensolarado da Alemanha, 
    país com maior parque solar do mundo. As regiões Sul e Norte brasileiras recebem menos irradiação 
    solar por apresentarem as latitudes mais altas e, assim, com maiores diferenças entre a duração do dia; 
    e nebulosidade frequente, reduzindo a irradiância solar na superfície receptora. Isto indica a 
    característica do país para a produção de energia solar.%\vspace{0.3cm}
        \begin{figure}[ht]
            \centering
            \caption{\small Mapa do total anual de irradiação solar direta normal.}
            \includegraphics[width=0.85\textwidth]{figures/fig4-mapa2.png}
            \begin{flushleft}
                \par \small Fonte: adaptado de Pereira et al. (2017).            
            \end{flushleft}
            \label{Figura 3}
        \end{figure}\vspace*{-0.4cm}

    \noindent Outro ponto importante no estudo de \textcite{Pereira2017} foi a constatação de que a geração máxima 
    nos estados da região Sudeste, nos meses de verão, coincide com os máximos de demanda registrados 
    pelo Operador Nacional do Sistema – ONS, para a mesma região. Esta coincidência entre gerações 
    máximas de energia elétrica pode aliviar os períodos de pico de demanda de energia elétrica no país.\vspace{0.3cm} \newline
    \noindent O Espírito Santo gerou cerca de 8 GW de energia elétrica proveniente de energia solar em 2018. 
    Esta geração foi realizada por empreendimentos particulares, visto que o estado não abriga usinas 
    solares cadastradas junto ao SIN \cite{EmpresadePesquisaEnergetica-EPE2019a}. O estado também 
    apresenta variação no nível de radiação menor do que em estados com maior produção de energia solar 
    como a Bahia, com variação de 6,5 kWh/m²/dia \cite{AgenciadeServicosPublicosdeEnergiadoEstadodoEspiritoSanto-ASPE2013}.\vspace{0.3cm} \newline
    A Região Metropolitana da Grande Vitória – RGMV, em particular Vitória, apresenta variação baixa, 
    entre 5,39 a 5,48 kWh/m²/dia, como exposto na Figura \ref{Figura 4}. Esta variação indica que a geração de 
    energia elétrica pode ser melhor aproveitada em comparação a outros estados com maior potência 
    instalada \cite{AgenciadeServicosPublicosdeEnergiadoEstadodoEspiritoSanto-ASPE2013}.\vspace*{-0.3cm}
        \begin{figure}[ht]
            \centering
            \caption{\small Radiação solar no plano inclinado do ES.}
            \includegraphics[width=0.55\textwidth]{figures/fig5-mapa3.jpg}
            \begin{flushleft}
                \par \small Fonte: adaptado de ARSP (2013).
            \end{flushleft}
            \label{Figura 4}
        \end{figure}\vspace*{-0.63cm}
    \subsubsection{Energia solar fotovoltaica}
    As tecnologias de geração de energia fotovoltaica vêm evoluindo ao longo dos últimos anos. 
    Este segmento está em pleno crescimento quando observado o acesso técnico e econômico ao sistema de 
    geração de energia à população \cite{Pereira2017}. Equipamentos fotovoltaicos são fonte promissora 
    de diversidade em produção de energia, dada a versatilidade de aplicação e integração entre 
    sistemas para a envoltória da edificação \cite{Sorgato2018}.\vspace{0.3cm} \newline
    A geração de energia elétrica por meio de células fotovoltaicas ocorre pela conversão de radiação 
    incidente sobre a área da célula em uma diferença de potencial entre suas extremidades. Esta célula 
    é constituída por duas camadas de elementos semicondutores dopados positiva e negativamente, 
    normalmente silício, o que propicia o ordenamento da corrente de elétrons, como exemplificado na 
    Figura \ref{Figura 5}.

        \begin{figure}[ht]
            \centering
            \caption{\small Esquema de geração de energia por célula fotovoltaica.}
            \includegraphics[width=1\textwidth]{figures/fig6_geracao_de_conrrente_continua_em_celulcas_fotovoltaicas_ARSP_2018.png}
            \begin{flushleft}
                \par \small Fonte: adaptado de ARSP (2013).
            \end{flushleft}
            \label{Figura 5}
        \end{figure}\vspace*{-0.4cm}
    \noindent Desde 1883, quando a primeira célula fotovoltaica foi constituída, a eficiência de conversão das 
    células avançou, partindo de 1\% a uma taxa de 19\% para os módulos comercializados atualmente. 
    Além da eficiência, o custo desta tecnologia foi barateado, aumentando o potencial de acesso pela população. 
    Os custos para a implementação da tecnologia saíram de US\$ 100/Wp, no início da década de 1970, 
    até US\$ 0,39/Wp, em 2016 \cite{AgenciadeServicosPublicosdeEnergiadoEstadodoEspiritoSanto-ASPE2013,Pereira2017}.\vspace{0.3cm} \newline
    Uma vantagem atribuída à aplicação de tecnologias fotovoltaicas na envoltória é a relação direta 
    entre quantidade de área de fachada exposta à radiação solar e a quantidade em potencial de produção 
    de energia solar fotovoltaica \cite{Veloso2017}. Outro fato importante é a verificação da viabilidade 
    econômica em climas tropicais como as cidades brasileiras apresentam \cite{Didone2014,Sorgato2018}.\vspace{0.3cm} \newline
    A versatilidade alcançada pela tecnologia é verificada pela facilidade de adaptação das células 
    fotovoltaicas para envoltória das edificações. Os componentes fotovoltaicos integrados ao edifício, 
    ou \textit{Building Integrated Photovoltaic} – BIPV, e os componentes fotovoltaicos adicionados/anexados a 
    edificação, ou \textit{Building Added/Attached Photovoltaic} – BAPV, são formas de introdução das 
    células para aproveitar a área disponível de fachada para produção de energia elétrica 
    \cite{AmericanSocietyofHeatingRefrigeratingandAir-ConditioningEngineers-ASHRAE2019}, como apresentado 
    pela Figura \ref{Figura 6}.\vspace*{-0.4cm}
        \begin{figure}[ht]
            \centering
            \caption{\small Componentes fotovoltaicos integrados a fachada do edifício Boulder Commons, localizado em Boulder, Colorado (EUA).}
            \includegraphics[width=0.75\textwidth]{figures/fig7_BIPV_ASHRAE_ZEB_2019.png}
            \begin{flushleft}
                \par \small Fonte: adaptado de ASHRAE (2019).
            \end{flushleft}
            \label{Figura 6}
        \end{figure}\vspace*{-0.4cm}

    \noindent As edificações recebem a radiação solar de acordo com a latitude e a orientação solar onde estão situadas. 
    Estas características influenciam diretamente na produção de energia fotovoltaica de um painel, onde a 
    inserção dos painéis depende da orientação das superfícies das células fotovoltaicas voltadas para a posição 
    perpendicular em relação a latitude do local. Este posicionamento tende a maximizar a incidência de radiação 
    sobre as células e, assim, a produção de energia. Regiões com diferentes latitudes tendem a apresentar 
    variabilidades de fotoperíodo, onde as baixas latitudes geram menos energia com os painéis posicionados 
    verticalmente e mais energia para os posicionados horizontalmente, sendo que o inverso aplica-se às regiões 
    com altas latitudes \cite{Pereira2017}.\vspace{0.3cm} \newline
    No exemplo da Figura 7, o edifício comercial \textit{Boulder Commons}, localizado em Boulder, Colorado, 
    exemplifica a utilização de um sistema BIPV para geração de energia solar fotovoltaica em uma região de 
    alta latitude, com a utilização de painéis posicionados horizontal e verticalmente \cite{AmericanSocietyofHeatingRefrigeratingandAir-ConditioningEngineers-ASHRAE2019,Pereira2017}.
    Dentre as tecnologias de células fotovoltaicas disponíveis no mercado, destacam-se as tecnologias de 
    silício cristalino e filmes finos. As caraterísticas gerais desses componentes são apresentadas na Tabela \ref{tab:tabela1}.
        \begin{table}[ht]\centering
            \caption{\small Tecnologias de células fotovoltaicas.}
            \vspace*{0.2cm}
            \label{tab:tabela1}
            \begin{tabular}{lcc}
            \hline
            \textbf{Tecnologia}                     & \textbf{Média anual total}    & \textbf{Área/kWp} \\ \hline
            \multicolumn{3}{l}{Silício cristalino}                                                    \\ \hline
            Monocristalino                          & 20,67                         & ~8m²              \\ \hline
            Policristalino                          & 29,00                         & ~7m²              \\ \hline
            \multicolumn{3}{l}{Filmes finos}                                                          \\ \hline
            Silício amorfo (a-SI)                   & 76,92\%                       & ~15m²             \\ \hline
            Telureto de cadmio (Cd-Te)              & 181,04                        & ~10m²             \\ \hline
            Disseleno de cobre-índio-gálio (CIGS)   & 2,05                          & ~10m²             \\ \hline
            \end{tabular}
            \begin{flushleft}
                \par \small Fonte: ARSP (2013).
            \end{flushleft}
        \end{table}\vspace*{-0.6cm}

\subsubsection{Legislação para a eficiência energética}
As políticas adotadas internacionalmente demonstram a urgência na busca de soluções relacionadas 
à mitigação do consumo de energia e redução de emissão de GEE, além de buscar, também, a redução 
dos impactos ambientais indiretamente relacionados ao ambiente construído \cite{InternationalMonetaryFund-IMF2018,InternationalEnergyAgency-IEA2018a}.\vspace{0.3cm} \newline
A Diretiva Europeia EPDB/31 (2010) estabeleceu como meta o balanço energético próximo a zero para 
novas edificações até dezembro de 2020. Nos Estados Unidos, o \textit{\textcite{U.S.DepartmentofEnergy-USDOE2015}}
apresentou medidas e programas para balanço energético nulo para as edificações comerciais e residenciais. 
O Departamento definiu como meta alcançar o balanço energético nulo em edificações residenciais 
até 2020 e edificações comerciais até 2025. Novas edificações comerciais, a partir de 2015, deveriam 
apresentar um plano de redução de energia desde sua concepção.\vspace{0.3cm} \newline
No Brasil, desde outubro de 2003 o Programa Nacional de Eficiência Energética em Edificações – 
PROCEL EDIFICA \cite{Brasil2001,Brasil2001a} atua de forma conjunta com o poder público, privado e a comunidade 
acadêmica para promover o uso racional da energia elétrica e recursos naturais em edificações. 
Nesta pesquisa, este marco legal foi utilizado como critério temporal de seleção para as edificações 
em Vitória concluídas após o início do programa, sendo então estabelecido o recorte temporal de 
estudos para edificações construídas após 2003.\vspace{0.3cm} \newline
Outro marco legal importante para o contexto desta pesquisa foi a Resolução Normativa nº 687 de 2015, 
substituindo a resolução precedente nº 482 de 2012, que trata sobre micro e mini geração de energia 
elétrica. Este conceito consiste em uma central geradora de energia elétrica que utilize fontes 
renováveis, com potência instalada superior a 75 kW e/ou capacidade instalada menor que 3 MW, 
classificado como mini geração; e potência instalada menor ou igual a 75 kW, classificado como 
micro geração \cite{AgenciaNacionaldeEnergiaEletricaANEEL2015}.\vspace{0.3cm} \newline
Com a regulamentação de micro e mini geração em território nacional, foi viabilizada a geração de 
energia descentralizada, característica importante para a coprodução de energia para uma edificação 
com balanço energético nulo.\vspace{0.3cm} \newline
Posteriormente, a Instrução Normativa nº 02 de 2014 institui a obrigatoriedade de submeter as edificações 
públicas ao Programa Brasileiro de Etiquetagem – PBE EDIFICA. Da mesma forma, foram criadas regulamentações 
relacionadas à eficiência energética e à certificações de edifícios de uso comercial e residencial, 
denominados Instrução Normativa Inmetro para a Classe de Eficiência Energética de Edificações Comerciais, 
de Serviços e Públicas, INI-C, e Instrução Normativa Inmetro para a Classe de Eficiência Energética 
de Edificações Residenciais, INI-R \cite{Dalbem2017,InstitutoNacionaldeMetrologiaNormalizacaoeQualidadeIndustrial-INMETRO2018}.\vspace{0.3cm} \newline
Os níveis de eficiência estabelecidos pelo regulamento partem do mais eficiente, classificado como 
“A”, para o menos eficiente, classificado como nível “E”. O regulamento classifica por meio de uma 
etiqueta indicativa do nível de eficiência energética da envoltória da edificação, dos sistemas de 
iluminação artificial e de condicionamento de ar \cite{InstitutoNacionaldeMetrologiaNormalizacaoeQualidadeIndustrial-INMETRO2018a}.\vspace{0.3cm} \newline
Dada a abrangência de cenários do regulamento e a utilização de referências como normas nacionais 
e internacionais para o desenvolvimento dos parâmetros de avaliação, o INI-C foi adotado como 
ferramenta de qualificação da eficiência energética e otimização dos modelos genéricos apresentados 
na etapa de metodologia. Esta escolha foi baseada na proximidade das características construtivas 
e de materiais sugeridas pela Instrução Normativa à realidade brasileira, sendo este fator pouco 
presente ou inexistente nas normas e regulamentos avaliadas.
\end{onehalfspace}

\end{onehalfspace}
\section{Metodologia}
\begin{onehalfspace}
    Assim, neste capítulo são apresentadas as três principais etapas utilizadas na 
metodologia para esta pesquisa. Estas etapas podem ser descritas como:
\begin{easylist}
    \ListProperties(Numbers1=r,Numbers2=l,Hide2=1,Hang=true,Margin1=3ex,Margin2=6ex,FinalSpace=1em)
    @ Definição dos modelos genéricos. A etapa de definição dos modelos foi elaborada em 3 partes, 
    dentre as quais:
        @@ Coleta    de    dados    sobre    as    características    das    edificações    comerciais, 
        especificamente de escritório, em Vitória (ES);
        @@ Levantamento e definição das variáveis sobre os padrões de uso e ocupação das 
        salas  de  escritório,  assim  como  padrões  de  conforto  e  níveis  de  
        eficiência energética dos equipamentos de condicionamento de ar e iluminação;
        @@ Estabelecimento  dos  modelos  genéricos com  o  intuito de  evidenciar o  
        consumo total  final  de  energia  elétrica  por  meio  da  determinação  da  classe  
        de  eficiência energética  da  edificação,  proposta  pela  INI-C,  e  o  
        potencial  de  otimização  e produção de energia elétrica a partir de fontes renováveis.
    @ Simulações.  Nesta  etapa  são  avaliadas  as  características  mais  influentes  
    no  consumo energético da edificação de referência e o potencial de geração de energia 
    solar. Ambas as  avaliações  serão  feitas  por  meio  de  simulação  computacional.  
    As  simulações  foram fracionadas em 3 partes, dentre as quais:
        @@ Simulação dos modelos real e de referência, onde é feita a determinação da classe 
        de desempenho energético das edificações observadas em campo; 
        @@ Otimização    dos    modelos    genéricos,    representando    a    etapa    onde    
        são implementadas estratégias passivas e ativas visando a eficientização da edificação;
\end{easylist}
\vspace*{0.3cm}
\noindent Com a aplicação da metodologia, simplificada no Fluxograma da Figura \ref{fig:figura7}, busca-se identificar se há 
condições para o balanço energético nulo total ou parcial dos modelos analisados.
\begin{figure}[ht]
    \centering
    \footnotesize
    \caption{\small Esquema simplificado da metodologia criada.}
    \includegraphics[width=1\textwidth]{figures/fig8_Fluxogramas_1.jpg}
    \begin{flushleft}
        \par \small Fonte: autor (2019).
    \end{flushleft}
    \label{fig:figura7}
\end{figure}\vspace*{-0.3cm}
\subsection{Levantamento das características dos edifícios de escritório de Vitória}
As edificações de escritório de Vitória selecionadas após a definição do recorte territorial, apresentam características que foram complementadas aos atributos observados em edificações comerciais brasileiras, como suporte as informações não encontradas in loco. As características com maior frequência de ocorrência no levantamento realizado são apresentadas na Tabela \ref{tab:tabela4}. Todavia, a amostra coletada abrange edifícios iniciados em 2003 e concluídos até o fim do primeiro trimestre de 2018, data do início do levantamento. Este fato inviabiliza aplicar a última revisão do Plano à amostra.\newline \vspace*{0.3cm}
Foram considerados para o levantamento atributos como gabarito, número de pavimentos-tipo, número de salas por pavimento-tipo, dimensão e forma, altura dos pavimentos-tipo. Não foram consideradas as dimensões dos lotes onde as edificações da amostra estacam implantadas, já que este atributo não foi pertinente ao objetivo do trabalho. Estes parâmetros foram reunidos em consulta ao material técnico disponibilizado pelas construtoras, visitas a campo e complementação de dados utilizando a ferramenta computacional \textit{Google Street View}.\newline \vspace*{0.3cm}
Os trabalhos de \textcite{Lamberts2006,AmericanSocietyofHeatingRefrigeratingandAir-ConditioningEngineers-ASHRAE2010,Bernabe2012,Ramos2013,Didone2014,Didone2014a,ConselhoBrasileirodeConstrucaoSustentavel-CBCS2015,Fonseca2016,Werneck2017,InstitutoNacionaldeMetrologiaNormalizacaoeQualidadeIndustrial-INMETRO2018}, foram utilizados como principais fontes de informação para a análise de envoltória e sistema de iluminação e condicionamento de ar.%\vspace{-0.25cm}
    \begin{table}[H]
        \centering
        \small
        \caption{Características observadas em campo e em pesquisas anteriores.}
        \begin{tabular*}{\columnwidth}{@{\extracolsep{\fill}}lll}
        \hline
        \textbf{Parâmetro}                                             & \textbf{Descrição}                                                                    & \textbf{Referências} \\ \hline
        Gabarito                                                       & 24 a 60 m (8 a 19 pav.)                                                               & \makecell[l]{Levantamento \textit{in loco} e referências\\ (BERNABÉ, 2012; CBCS, 2015; \\FONSECA et al., 2016; LAMBERTS; \\GHISI; RAMOS, 2006; RAMOS \\et al., 2013).} \\ \hline
        Altura do pavimento                                            & 3 m                                                                                   & Levantamento \textit{in loco}.                                                                                                                                         \\ \hline
        Planta-baixa (forma)                                           & Retangular                                                                            & \makecell[l]{Levantamento \textit{in loco} e referências\\ (FONSECA et al., 2016; INMETRO,\\ 2018).}                                                                   \\ \hline
        \makecell[l]{Dimensão das salas\\ por pav.-tipo}               & 40 m²                                                                                 & \makecell[l]{Foi fixado a área das salas \\(zonas térmicas) de acordo com a \\média de ofertas de salas observadas\\ em levantamento \textit{in loco}.}                  \\ \hline
        \multicolumn{3}{c}{Continua}\\\hline
    \end{tabular*}
    \label{tab:tabela4}
    \end{table}\pagebreak
    \begin{table}[H]
        \centering
        \small
        \begin{tabular*}{\columnwidth}{@{\extracolsep{\fill}}lll}
        \hline
        \multicolumn{3}{c}{Conclusão}\\\hline
        \makecell[l]{Componentes da\\ parede}                          & \makecell[l]{Bloco cerâmico,\\ 8 furos; \\14x19x29 cm; \\argamassa de\\ assentamento} & Levantamento \textit{in loco}.                                                                                                                                         \\ \hline
        Proteção solar                                                 & Sem proteção                                                                          & \makecell[l]{Levantamento \textit{in loco} e referências\\ (FONSECA et al., 2016; WERNECK \\et al., 2017).}                                                            \\ \hline
        Cobertura                                                      & \makecell[l]{Laje impermeabilizada\\ com 20 cm de\\ espessura}                        & \makecell[l]{Levantamento \textit{in loco} e referências \\(CB3E; ABIVIDRO, 2015).}                                                                                    \\ \hline
        Vidros                                                         & \makecell[l]{Laminado; Reflexivo;\\ 8 mm; Verde}                                      & \makecell[l]{(FONSECA et al., 2016; \\INMETRO, 2018).}                                                                                                                   \\ \hline
        PAF\textsubscript{T}                                           & 30\%; 50\%; 80\%                                                                      & \makecell[l]{Levantamento \textit{in loco}\\ e referências.}                                                                                                             \\ \hline
        \makecell[l]{Orientação solar da \\fachada principal}          & Sul                                                                                   & \makecell[l]{Levantamento \textit{in loco}\\ e referências.}                                                                                                             \\ \hline
        \makecell[l]{Densidade de Carga de\\ Iluminação Limite – DCIL} & 14,1 W/m²                                                                             & \makecell[l]{Consulta pública do RTQ-C \\(INMETRO, 2018).}                                                                                                               \\ \hline
        \makecell[l]{Densidade de Carga de\\ Equipamentos – DCE}       & 9,7 W/m²                                                                              & \makecell[l]{Consulta pública do RTQ-C \\(INMETRO, 2018).}                                                                                                               \\ \hline
        \makecell[l]{Absortância/transmitância \\das paredes}          & \makecell[l]{0,59 (cor \\camurça)/3,75}                                               & \makecell[l]{Valores consultados na \\NBR 15220-2 e referências \\(ABNT, 2003; FONSECA et \\al., 2016; INMETRO, 2018).}                                                                     \\ \hline
        \makecell[l]{Absortância/transmitância \\das coberturas}       & \makecell[l]{0,65 (concreto \\aparente)/2,06}                                         & \makecell[l]{Valores consultados na \\NBR 15220-2 e referências \\(ABNT, 2003; FONSECA et \\al., 2016; INMETRO, 2018).}                                                 \\ \hline
        \end{tabular*}
        \begin{flushleft}
            \par \small Fonte: autor (2019).
        \end{flushleft}
    \end{table}\vspace*{-0.3cm}

\noindent As informações coletadas nos estudos e em levantamento formam a base conceitual para determinar os aspectos arquitetônicos relevantes para compor os modelos genéricos e, posteriormente, determinar os parâmetros de otimização e do consumo energético padrão aproximado para um edifício de escritório. Além disso, a quantidade de simulações necessárias para determinar o consumo de energia dos modelos genéricos é identificada por meio da organização dos dados coletados em campo e, assim, a resultante do número de variáveis.

\subsubsection{Consumo de energia elétrica das edificações}
\begin{minipage}[H]{1\textwidth}
    O consumo de energia elétrica em edificações de escritório no Brasil é determinado principalmente por sua tipologia. As configurações predominantes no Brasil compreendem, em sua maioria, pequenas edificações, abaixo de 8 pavimentos, a edifícios grandes, acima de 15 pavimentos \cite{Carlo2008,Ramos2013,ConselhoBrasileirodeConstrucaoSustentavel-CBCS2015,Fonseca2016}.\vspace*{0.3cm} \newline
    Correlacionado às tipologias arquitetônicas, os dados de consumo de energia, expressos em Intensidade de Uso de Energia – IUE, kWh/m²/ano, foram necessários para validação das simulações iniciais quanto ao consumo de energia esperado dos modelos. Visto que a calibração dos modelos genéricos não foi possível, pois eles não dispunham de memorial de massa como ferramenta de comparação ao consumo simulado computacionalmente, foram adotados como método de comparação os valores médios registrados no Relatório Final do CBCS (2015), como mostra o Histograma no Gráfico 5. Foi relacionado o consumo de energia das edificações levantadas à frequência de ocorrência da quantidade de pavimentos de edificações comerciais brasileiras e suas respectivas áreas comuns.
\end{minipage}
    \begin{graph}
        \par \small Gráfico 5 - Consumo energético em edificações de escritório brasileiras.
        \begin{minipage}[ht]{1\textwidth}\centering
            \includegraphics[width=1.0\textwidth]{figures/fig9_consumo-total-das-edificacoes-levantadas_cbcs_2015.png}            
        \end{minipage}
        \begin{flushleft}
            \par \small Fonte: adaptado de CBCS (2015).
        \end{flushleft}
    \end{graph}

\noindent Logo, os dados de IUE adotados para a comparação e avaliação inicial do consumo energético dos modelos foram baseados nos valores máximo e mínimo estabelecidos pelo CBCS (2015). A média entre o consumo em áreas comuns e total relacionado à frequência de quantidade de pavimentos define a quantidade total de consumo de energia para cada tipologia.\vspace*{0.3cm} \newline
Os autores do Relatório atribuem 133 kWh/m²/ano aos edifícios de pequeno porte, e 268 kWh/m²/ano às edificações de grande porte. Vale ressaltar que os dados de consumo energético, assim como a média apontada no Relatório, de 191 kWh/m²/ano, desprezam as distorções causadas por edificações com particularidades de consumo como datacenters ou erros de cálculo de área útil.\vspace*{0.3cm} \newline
Ao estabelecer as Intensidades de Uso de Energia equivalentes a cada modelo, assim como os padrões de uso e ocupação, pode-se estimar o consumo de energia durante determinado período.

\subsubsection{Padrões de uso e ocupação em edifícios de escritório}
Os padrões de uso e ocupação da edificação foram baseados em normas, regulamentos, relatórios técnicos e referências acadêmicas que consideraram os níveis de atividades desenvolvidas nos ambientes, tratadas neste trabalho como zonas térmicas, como apresentado na Tabela \ref{tab:tabela5}.\vspace*{0.3cm} \newline
Como o intuito do trabalho foi criar modelos genéricos que representassem minimamente o cenário encontrado na cidade de Vitória, as características de uso e ocupação escolhidas foram integradas como forma de aproximar as tipologias ao cenário observado. Dentre elas, pode-se citar:
\begin{itemize}
    \item O nível metabólico apresentado em atividades de escritório;
    \item O horário de funcionamento dos escritórios, determinando os intervalos de tempo de ocupação total e parcial, onde, respectivamente, a capacidade máxima e parcial de ocupação das zonas térmicas é atingida;
    \item A densidade de pessoas por metro quadrado para cada zona térmica;
    \item A temperatura de conforto em cada zona térmica, de acordo com as normas de conforto térmico consultadas; e
    \item A umidade relativa do ar nos ambientes de escritório.
\end{itemize}

\noindent A obtenção dos dados acerca da atividade desempenhada nas edificações de escritório, além do horário de funcionamento e densidade de ocupação serão utilizados para estimar o consumo de energia elétrica do espaço utilizado por meio de simulação computacional.
\begin{table}[H]
    \centering
    \small
    \caption{Padrões de uso e ocupação}
    \begin{tabular*}{\columnwidth}{@{\extracolsep{\fill}}llll}
        \hline
        \textbf{Parâmetro}                                & \multicolumn{2}{c}{\textbf{Descrição}}                                                                                                                          & \textbf{Referências} \\ \hline
        \multirow{2}{*}{Atividades}                       & \makecell[l]{Escritório:\vspace*{0,2cm}\\Fator metabólico:\vspace*{0,2cm}} & \makecell[l]{Leve\vspace*{0,2cm}\\0,9 met}                                         & \makecell[{{p{5cm}}}]{As salas de escritório da cidade são utilizadas, em sua maioria, para atividades especializadas de âmbito jurídico, relacionadas à construção civil, a saúde e atividades financeiras.} \\ \hline
        \multirow{2}{*}{Horário de funcionamento}         & \makecell[l]{Ocupação\\ total:\vspace*{0,2cm}\\Ocupação \\parcial (50\%):\vspace*{0,6cm}}  & \makecell[l]{8h às 12h; \\13h às 18h \vspace*{0,6cm}\\12h às 13h}  & \makecell[{{p{5cm}}}]{Segundo normas e pesquisas sobre o horário de funcionamento de escritórios, o início da ocupação se dá às 6h e pode se estender até às 24h. Entretanto, visando a aproximação às condições praticadas no mercado brasileiro, adota-se a redução de ocupação durante o horário de almoço, denominada ocupação parcial.}  \\ \hline
        Densidade de ocupação                             & \multicolumn{2}{c}{0,14 pessoas/m²}                                                                                                                             & \makecell[{{p{5cm}}}]{(CONSELHO BRASILEIRO DE CONSTRUÇÃO SUSTENTÁVEL - CBCS, 2015; LAMBERTS; GHISI; RAMOS, 2006; MORAES; PEREIRA, 2014).} \\ \hline
        Temperatura de controle                           & \multicolumn{2}{c}{24°C}                                                                                                                                        & \makecell[{{p{5cm}}}]{Temperatura limite de acionamento do sistema de condicionamento de ar (ASHRAE, 2010, 2017a; INMETRO, 2010a).} \\ \hline
        \makecell[l]{Nível de iluminância\\ de referência}& \multicolumn{2}{c}{500 lux}                                                                                                                                     & \makecell[{{p{5cm}}}]{Iluminância mínima (entorno de trabalho) para atividades visuais (ASHRAE, 2010; ABNT, 2013).} \\ \hline
        \makecell[l]{Umidade Relativa\\ Interna}          & \multicolumn{2}{c}{40\%-60\%}                                                                                                                                   & \makecell[{{p{5cm}}}]{Faixa recomentada pela ASHRAE 55 (2017a).}\\ \hline
    \end{tabular*}
    \begin{flushleft}
        \par \small Fonte: autor (2019).
    \end{flushleft}
\end{table}
\vspace*{-0.5cm}
\subsection{Padrões de conforto}
O conforto térmico e de iluminação natural são parâmetros essenciais para a avaliação 
da qualidade do ambiente em que se vive. Não obstante, é necessário estabelecer padrões 
de conforto para a simulação de desempenho termoenergético do modelo genérico. Os 
próximos subitens tratam dos parâmetros adotados para a posterior inserção de dados 
e informações nos processos de simulação.
\subsubsection{Conforto térmico}
Utilizando os conceitos de conforto adaptativo, foi calculada a temperatura de conforto (t\textsubscript{c}) 
descrita pela \textcite{AmericanSocietyofHeatingRefrigeratingandAir-ConditioningEngineers-ASHRAE2017}, e assim estipulada a faixa de temperaturas de conforto e 
set point para controle de temperatura dos ambientes dos modelos genéricos. Este modelo 
avalia a adaptação do usuário em diferentes campos como o fisiológico, o comportamental 
e o psicológico \cite{AmericanSocietyofHeatingRefrigeratingandAir-ConditioningEngineers-ASHRAE2017a} e pode ser expresso pela Equação \ref{eq:eq1} tratada no Referencial Teórico.\vspace*{0.3cm} \newline
Foram utilizadas as temperaturas de bulbo seco máximas e mínimas como valores representantes 
da temperatura externa da edificação \cite{InstitutoNacionaldeMetereologia-INMET2018}. Estes dados de temperatura serão 
fundamentais para a caracterização do meio em que o modelo genérico será simulado, 
influenciando no desempenho simulado da envoltória e dos sistemas de condicionamento de ar 
e iluminação.
\subsubsection{Conforto lumínico}
A norma NBR/ISO CIE 8995-1 \cite{AssociacaoBrasileiradeNormasTecnicas-ABNT2013} determina que a condição de conforto visual 
em ambientes de escritório requer valor igual ou maior que 500 lux de iluminância de entorno 
imediato da tarefa a ser desempenhada \cite{AssociacaoBrasileiradeNormasTecnicas-ABNT2013,Ramos2013}. Assegurar a 
iluminância é uma condição importante para garantir o conforto, desempenho e segurança visual 
aos usuários, seja por fonte luminosa artificial ou natural.\vspace*{0.3cm} \newline
Para que a iluminação interna se mantenha dentro do padrão estabelecido por norma, foi 
necessário calcular o Fator de Luz Diurna – FDL, como forma de verificar se a adoção 
iluminância proposta pela norma seria adequada as condições empregadas aos modelos de 
referência. Este fator, uma vez incorporado nas rotinas de simulação computacional, 
influencia diretamente nas condições de iluminação no ambiente, indicando se estão adequadas 
para a realização de tarefas que exigem um determinado volume de iluminação.\vspace*{0.3cm} \newline
Considerando a ocupação anual parcial dos ambientes, foi possível estabelecer a relação 
entre a iluminância interna desejada, 500 lux, e a mediana da iluminância difusa horizontal 
externa, dado retirado do arquivo climático de Vitória \cite{InstitutoNacionaldeMetrologiaNormalizacaoeQualidadeIndustrial-INMETRO2018}, de 50019,16 lux. 
Assim, o FLD necessário para manter a iluminância interna das salas de escritório durante o 
período de ocupação levantado pode ser expresso pela Equação \ref{eq:eq2}.
\begin{equation}\label{eq:eq2}
    FLD=\frac{E_{interna}}{E_{externa}}*100
\end{equation}

Onde:\par
\setlength\parindent{1.5cm} \textit{FLD} é o Fluxo de Luz Diurna, em porcentagem;\par
\setlength\parindent{1.5cm} E\textsubscript{interna} é a iluminância interior em um ponto de um plano, em lux; e\par
\setlength\parindent{1.5cm} E\textsubscript{externa} é a iluminância externa simultânea em um plano horizontal, em lux.\vspace*{0.3cm} \newline
\noindent Concluída a caracterização dos padrões de conforto a serem avaliados, são definidos os 
parâmetros para os níveis de eficiência energética esperados dos equipamentos de 
condicionamento de ar e iluminação artificial presentes no modelo genérico. A partir desta 
definição embasada em regulamentos como o INI-C, é possível estimar a otimização de consumo 
energético da edificação proposta.
\subsection{Níveis de eficiência energética}
Os níveis de eficiência dos aparelhos utilizados na manutenção do conforto da edificação exercem importante papel quando há a necessidade de racionamento do consumo energético. Para tal, estes aparelhos são avaliados segundo critérios de desempenho energético e, ao final dessa avaliação, é atribuído um índice representativo da eficiência energética alcançada. Os índices de desempenho apresentados neste trabalho foram baseados na Instrução Normativa Inmetro para Classe de Eficiência Energética de Edificações Comerciais, de Serviços e Públicas – INI-C.

\subsubsection{Determinação da classe de eficiência energética dos modelos genéricos}
As primeiras simulações energéticas servem como meio de ajuste e verificação de erros entre os dados de saída de consumo energético e intensidade de uso de energia. Os resultados das simulações iniciais foram comparados às análises feitas de Determinação de Classe de Desempenho Energético, baseado na metodologia do INI-C, e nos resultados apresentados pelo Relatório Final de Desempenho Energético do CBCS. A partir da discrepância dos valores obtidos entre as simulações iniciais e as referências selecionadas, foram realizados ajustes a fim de adequar o modelo computacional à tolerância estabelecida.\vspace*{0.3cm} \newline
Os modelos genéricos foram avaliados segundo a definição da Instrução Normativa, onde é definido que o Modelo Real representa a edificação a ser classificada, enquanto o Modelo de Referencia representa a edificação com baixo desempenho energético. Desta forma, os edifícios propostos são comparados com um modelo de baixa performance, evidenciando, assim, seu desempenho energético. As variáveis analisadas para a comparação com o Modelo de Referência foram o consumo de energia térmica, elétrica e primaria dos modelos.\vspace*{0.3cm} \newline
A partir do diagnóstico de desempenho energética dos modelos genéricos, foram implementadas as otimizações e medidas de produção de energia em contraponto ao consumo energético constatado em simulação.

\subsubsection{Eficiência energética do sistema de condicionamento de ar}
Os sistemas de condicionamento de ar presentes em edifícios de escritório são compostos basicamente por dois tipos de equipamentos: ar-condicionado Split e o Sistema Central de Água Gelada – CAG \cite{ConselhoBrasileirodeConstrucaoSustentavel-CBCS2015}.\vspace*{0.3cm} \newline
Segundo o \textcite{InstitutoNacionaldeMetrologiaNormalizacaoeQualidadeIndustrial-INMETRO2018}, esses equipamentos são classificados segundo a capacidade de resfriamento e a potência absorvida pelos motores em pleno funcionamento. Esta relação é representada pelo Coeficiente de Performance – COP, expresso em W/W. O COP para equipamentos de ar condicionado é categorizado segundo uma Classe de Eficiência Energética – CEE, como apresentado na Tabela \ref{tab:tabela6}.

\begin{table}[ht]\centering
    \caption{\small  Relação entre o COP e as Classes de Eficiência Energética de Condicionadores de ar.}
    \vspace*{0.3cm}
    \label{tab:tabela6}
    \begin{tabular}{cccc}
        \hline
        \textbf{Classes}     & \multicolumn{3}{c}{\textbf{Densidade de Potência de Iluminação (DPI – W/m²)}}        \\ \hline
        A                    & \makecell[c]{3,23}       & \makecell[c]{\textless CEE}                  &            \\ \hline
        B                    & \makecell[c]{3,02}       & \makecell[c]{\textless CEE =\textless} & 3,23       \\ \hline
        C                    & \makecell[c]{2,81}       & \makecell[c]{\textless CEE =\textless} & 3,02       \\ \hline
        D                    & \makecell[c]{2,60}       & \makecell[c]{\textless CEE =\textless} & 2,81       \\ \hline
    \end{tabular}
    \begin{flushleft}
        \par \small Fonte: adaptado de INMETRO (2018).
    \end{flushleft}
\end{table}\vspace*{-0.5cm}

\noindent Esta classificação atribuída ao sistema de ar condicionado dos modelos genéricos, juntamente ao nível de eficiência energética do sistema de iluminação, é essencial para a etapa de simulação e identificação do consumo final de energia dos modelos genéricos, representantes do cenário observado dentro do recorte territorial.\vspace*{0.3cm} \newline
Como o sistema mais frequente observado nas edificações levantadas foi o Sistema Central de Água Gelada – CAG, com volume de ar constante, e este, tomando como base os modelos mais populares do mercado brasileiro, possui COP base próximo a classe “D” por demandar muita energia à sua operação, foi utilizado o valor de COP indicado à classe “D” pela tabela da INI-C. Posterior a etapa de determinação de classe, foi feita a substituição dos aparelhos pertencentes à classe “D” – 2,60, para a classe “A” – 3,23 como forma de otimização utilizando estratégia ativa de redução de consumo de energia.

\subsubsection{Eficiência energética do sistema iluminação artificial e equipamentos}
A eficiência do sistema de iluminação e equipamentos de um edifício de escritório, tal qual o sistema de condicionamento de ar, representa uma parcela importante no consumo final de energia elétrica da edificação \cite{AmericanSocietyofHeatingRefrigeratingandAir-ConditioningEngineers-ASHRAE2019,ConselhoBrasileirodeConstrucaoSustentavel-CBCS2015}.\vspace*{0.3cm} \newline
Dessa forma, para certificar que o sistema de iluminação artificial seja energeticamente eficiente e reduza o impacto desse sistema no consumo de energia elétrica, é avaliada a razão entre o somatório das potências das lâmpadas e reatores instalados e a área de um ambiente ou zona térmica, razão denominada como Densidade de Potência de Iluminação – DPI, expressa em W/m². O mesmo procedimento é aplicado aos equipamentos, definidos pela Densidade de Potência de Equipamentos – DPE, expressa em W/m². A união das duas densidades é definida pela Densidade de Carga Interna – DCI.\vspace*{0.3cm} \newline
Após a avaliação do DPI, os equipamentos de iluminação artificial são classificados segundo a classe de eficiência energética estabelecida pelo PBE/Inmetro, conforme a Tabela \ref{tab:tabela7}.

\begin{table}[ht]\centering
    \caption{\small Densidades de Potência de Iluminação definidas pelo INI-C e as Classes de Eficiência Energética.}
    \vspace*{0.3cm}
    \label{tab:tabela7}
    \begin{tabular}{cc}
        \hline
        \textbf{Classes}                & \textbf{Densidade de Potência de Iluminação (DPI – W/m²)}\\ \hline
        A                               & 8,50                                                     \\ \hline
        B                               & 10,40                                                    \\ \hline
        C                               & 12,20                                                    \\ \hline
        D                               & 14,10                                                    \\ \hline
    \end{tabular}
    \begin{flushleft}
        \par \small Fonte: adaptado de INMETRO (2018).
    \end{flushleft}
\end{table}\vspace*{-0.5cm}
\noindent Assim como o sistema de condicionamento de ar, a definição da DPI possibilita classificar o sistema de iluminação artificial do modelo genérico segundo sua eficiência energética. Para a determinação da classe de eficiência energética, o procedimento é o mesmo adotado para definição do COP para a etapa de simulação, e neste caso, como os equipamentos elétricos e de iluminação não foram levantados nominalmente, partiu-se da situação requerida e indicada pelo INI-C, com DPI classe “D”, de 14,10 W/m², e equipamentos com DPE de 9,7 W/m². Desta forma, espera-se que seja evidenciada a influência do sistema de iluminação no contexto geral de otimização da edificação.

\subsection{Definição dos modelos genéricos}
Com base no INI-C (2018) e no levantamento das edificações de escritório de Vitória, foram 
propostos dois tipos de modelos genéricos como base para o estudo das modificações de 
otimização e de produção de energia. Estes modelos representam os dois cenários de ambiente 
construído mais observados na cidade de Vitória. Estes cenários são formados por edificações 
mais baixas, com 8 pavimentos, e as mais altas, com 19 pavimentos. As dimensões utilizadas 
como referência para a construção dos modelos genéricos foram resultado dos valores médios 
observados nas edificações que compõe o levantamento.\vspace*{0.3cm} \newline
As características predominantes aplicadas aos modelos genéricos foram:
    \begin{itemize}
        \item Número de pavimentos;
        \item Forma - retangular;
        \item Altura - gabarito e dimensões das fachadas;
        \item Layout interno dos pavimento-tipo;
        \item Ausência de proteção solar;
        \item Percentual Total de Área de Abertura da Fachada.
    \end{itemize}
A composição dos modelos é baseada nas características predominantes e nos dados coletados 
\textit{in site}.
\subsubsection{Composição dos modelos genéricos}
A composição construtiva atribuída aos modelos utilizados neste trabalho mostra fundamentalmente 
os parâmetros necessários para a avaliação do desempenho energético segundo o INI-C. Os 
atributos utilizados serviram como ponto de partida para as análises subsequentes sugeridas nas 
etapas metodológicas e estão dispostos no Fluxograma da Figura \ref{fig:figura10}.
    \begin{figure}[ht]
        \centering
        \caption{\small Fatores utilizados como parâmetros de configuração volumétrica dos modelos genéricos.}
        \includegraphics[width=0.9\textwidth]{figures/fig10_Fluxogramas-2.jpg}
        \begin{flushleft}
            \par \small Fonte: autor (2019).
        \end{flushleft}
        \label{fig:figura10}
    \end{figure}
    
Apresentados na Tabela 7 e exemplificado na Figura \ref{fig:figura11}, os atributos estudados foram Fator de 
Forma, FF, Fator Altura, FA, Percentual de Área de Abertura da Fachada Total, PAFT, Ângulo 
Vertical de Sombreamento, AVS, e Ângulo Horizontal de Sombreamento, AHS.
    \begin{figure}[ht]
        \centering
        \caption{\small Estrutura arquitetônica dos modelos genéricos.}
        \includegraphics[width=0.9\textwidth]{figures/fig11_8-19-2pav.png}
        \begin{flushleft}
            \par \small Fonte: autor (2019).
        \end{flushleft}
        \label{fig:figura11}
    \end{figure}


\end{onehalfspace}
\section{Resultados e discussão}
\noindent Neste capítulo são apresentados os resultados obtidos sobre o potencial de balanço energético nulo e quase nulo para as edificações propostas. No primeiro momento, os modelos foram classificados segundo a classe de eficiência energética disposta pelo INI-C, seguidos das otimizações que resultaram na análise de impacto das medidas implementadas. Após as otimizações foram avaliados os sistemas de produção de energia solar fotovoltaica propostos, e a capacidade de geração de energia anual.\vspace*{0.3cm} \newline
\noindent Verificou-se que a otimização e as estratégias ativas e passivas proporcionaram as condições necessárias para o balanço energético dos modelos. Ao final das simulações de consumo e produção de energia, foi analisada a viabilidade econômica das soluções propostas. Esta análise mostrou condições econômicas favoráveis de implantação do sistema de produção de energia proposto aos modelos.  Estes dados foram expressos em gráficos e tabelas a fim de facilitar a compreensão dos resultados alcançados.
\subsection{Classificação de desempenho energético dos modelos genéricos}
\noindent Para a determinação da classe de eficiência energética dos modelos genéricos, segundo a metodologia descrita pelo INI-C, é exigida a comparação entre dois cenários definidos como modelo real e o modelo de referência. Desta forma, foram configurados os cenários para a identificação do consumo total de energia térmica, elétrica e de energia primária, como descrito no capítulo 3. Como resultado, foi constatada a baixa eficiência energética dos modelos, como exposto na Tabela 15.

\subsection{Impacto das variáveis sobre o consumo anual de energia elétrica}
\noindent A etapa de otimização possibilitou visualizar a influência das variáveis sobre o consumo anual final das edificações para cada cenário definido. Os resultados da implementação das medidas ativas de redução de consumo de energia, disponíveis nos Apêndices deste trabalho, demonstraram maior peso entre as variáveis de estratégia ativa, pertencentes aos blocos de simulação 2, 3 e 9, e passivas, formadas pelos blocos de simulação 1, e de 4 a 10, como exposto nos Gráficos 4 e 5.\vspace*{0.3cm} \newline
\noindent Nota-se que as variáveis de estratégias ativas reduziram significativamente o consumo anual, com curvas de queda expressivas, direcionando a média de consumo para baixo, posicionando alguns cenários abaixo da média linear, como observado entre os blocos de simulação 1 ao 7 do Gráfico 4. Este comportamento se estende até o bloco de simulação 7 no modelo de 19 pavimentos, desempenho demonstrado no Gráfico 5. Os blocos de simulação compostos por variáveis passivas apresentaram constância de forma e estreitamento ao longo do avanço das implementações das medidas. O estreitamento gradativo da amplitude da curva de consumo final anual em ambos os resultados sugere que o avanço cumulativo das medidas de mitigação de consumo é importante para o sucesso do processo de adequação da edificação ao conceito \textit{Zero Energy}.\vspace*{0.3cm} \newline
\noindent A tendência ascendente das curvas de consumo, a partir do bloco de simulação 4, é causada pela alteração do PAF\textsubscript{T}, partindo da menor razão, de 30\%, até alcançar a maior razão entre área de fachada e área de aberturas envidraçadas, com 80\%. Esta medida aponta a importância da adoção de aberturas menores, em torno de 30\%, para a melhor relação entre o PAF\textsubscript{T} e o consumo energético da edificação.\vspace*{0.3cm} \newline
\noindent Os gráficos serviram como instrumento de tomada de decisão acerca da seleção dos melhores cenários para a etapa de simulação de geração de energia solar fotovoltaica. Além disso, os gráficos facilitaram a visualização da importância de cada estratégia adotada para o balanço energético nulo das edificações avaliadas. A leitura dos gráficos pode ser feita da seguinte forma:
\begin{itemize}
    \item Os gráficos consistem em evidenciar o impacto das medidas adotadas no consumo da edificação. Para tal, foi definida a comparação entre o desempenho dos modelos por meio do percentual de redução de consumo de energia entre as edificações. Esta comparação desconsidera a escala dos gráficos, já que ambos correspondem a edificações de portes diferentes, e considera o comportamento das medidas aplicadas por meio de análise das curvas de consumo;
    \item Os pontos na cor laranja representam cada cenário montado, os quais variam entre 4 conjuntos de implementação sequencial de medidas de redução de energia. Cada conjunto de cenários pertencentes aos 10 blocos de simulação estão definidos e representados nos Gráficos 4 e 5;
    \item As curvas em verde representam a tendência de aumento ou redução de consumo de 
    energia entre a simulação dos cenários;
    \item Como a mudança entre os tipos de vidro avaliados, foi necessário realizar novas simulações de todos os blocos anteriores a esta medida, sendo essa etapa representada pelas curvas acentuadas entre conjuntos de cenários, como, por exemplo, observado na curva entre os cenários 12 e 13 do bloco de simulação 10, na página 93.            
\end{itemize}
\noindent A redução de consumo provocada pela implementação de equipamentos e iluminação mais eficientes atinge o patamar de 22,48\% de redução global, assim como o sistema de arrefecimento, com 34,78\% para o modelo de 19 pavimentos. As reduções no modelo de 8 pavimentos foram tênues, da ordem de 16,09 e 16,66\%, respectivamente, para os sistemas de arrefecimento, e equipamentos e iluminação, como observado na Figura 21.\vspace*{0.3cm} \newline
\noindent Entretanto, cabe mencionar que o desempenho de controle térmico do sistema Split para o modelo de 19 pavimentos, nos cenários iniciais, provocou maior consumo de energia em relação aos dois outros sistemas de arrefecimento avaliados, CAG e VRF, em 22,73\%. Após a implementação do vidro mais eficiente, houve redução do consumo em cerca de 24,91\%, superando ligeiramente em performance o sistema CAG. Em contraste a esta observação do Split para o modelo de 19 pavimentos, este sistema obteve comportamento semelhante ao sistema VRF para a edificação de 8 pavimentos, com considerável performance e redução de consumo de energia. O \textit{Split} apresentou controle satisfatório das médias de temperatura de conforto das zonas térmicas. A diferença de desempenho dos sistemas de condicionamento de ar entre os dois modelos pode ser atribuída a diferença de carga térmica entre as edificações, onde o modelo de 8 pavimentos apresenta menor carga térmica e menor exposição à radiação solar em comparação ao modelo de 19 pavimentos.\vspace*{0.3cm} \newline
\begin{figure}[H]
    \centering
    \rotatebox{90}{
        \begin{minipage}{\textheight}
            \caption{Consumo \textit{versus} medidas do modelo genérico de 8 pavimentos.}
            \includegraphics[width=1.0\textwidth]{figures/result/fig23-grafico4.png}
            \begin{flushleft}
                \par \small Fonte: autor, (2020). Legenda: Bloco de simulações 1 a 3 – Blocos com implementação do vidro, orientações solares e sistemas de condicionamento de ar; Bloco de simulações 4 –Bloco com implementação das variações de PAF\textsubscript{T}; Bloco de simulações 5 e 6 – Blocos com implementação de paredes e coberturas; Bloco de simulações 7 e 8 e 9 – Blocos com implementação das proteções solares; Bloco de simulações 10 – Blocos com implementação das medidas de redução de carga.
            \end{flushleft}
            \label{fig:figure23}
        \end{minipage}
    }
\end{figure}
\pagebreak

\begin{figure}[H]
    \centering
    \rotatebox{90}{
        \begin{minipage}{\textheight}
            \caption{Consumo \textit{versus} medidas do modelo genérico de 19 pavimentos.}
            \includegraphics[width=1.0\textwidth]{figures/result/fig24-grafico5.png}
            \begin{flushleft}
                \par \small Fonte: autor, (2020). Legenda: Bloco de simulações 1 a 3 – Blocos com implementação do vidro, orientações solares e sistemas de condicionamento de ar; Bloco de simulações 4 –Bloco com implementação das variações de PAF\textsubscript{T}; Bloco de simulações 5 e 6 – Blocos com implementação de paredes e coberturas; Bloco de simulações 7 e 8 e 9 – Blocos com implementação das proteções solares; Bloco de simulações 10 – Blocos com implementação das medidas de redução de carga.
            \end{flushleft}
            \label{fig:figure24}
        \end{minipage}
    }
\end{figure}
\pagebreak
\noindent Como os vidros foram utilizados em todos os cenários, a importância da variável se torna mais visível quando comparado com todas as variáveis, assim como apresentado na Figura \ref{fig:figure25}. Verifica-se que, ao longo das implementações de medidas passivas, a influência é reduzida de 14,40\% para 4,00\%, conforme observado anteriormente nos blocos de simulação 3 e 4.\vspace*{0.3cm} \newline
\noindent Em ambos os resultados foi notado que a implementação de componentes construtivos com menor transmitância térmica acentua a diferença os cenários entre os modelos otimizados e os não-otimizados. Estes resultados retratam a importância das estratégias ativas para a redução do consumo total anual.
\begin{figure}[H]
    \centering
    \caption{Gráfico dos blocos de simulação das variáveis de vidro, orientação solar e do sistemas de condicionamento de ar e medidas de redução de carga dos modelos genérico de 8 (esq.) e 19 pavimentos (dir.)}
    \begin{subfigure}[b]{0.49\textwidth}
        \includegraphics[width=\textwidth]{figures/result/fig25-bloco1-3.png}
    \end{subfigure}
    \begin{subfigure}[b]{0.49\textwidth}
        \includegraphics[width=\textwidth]{figures/result/fig26-bloco1-3.png}
    \end{subfigure}
    \begin{subfigure}[b]{0.49\textwidth}
        \includegraphics[width=\textwidth]{figures/result/fig27-bloco10.png}
    \end{subfigure}
    \begin{subfigure}[b]{0.49\textwidth}
        \includegraphics[width=\textwidth]{figures/result/fig28-bloco10.png}
    \end{subfigure}
    \begin{flushleft}
        \par \small Fonte: autor, (2020).
    \end{flushleft}
    \label{fig:figure25}
\end{figure}
\noindent As características das paredes e cobertura, blocos de simulação 5 e 6 respectivamente, na Figura 22, quando observadas isoladamente, são medidas importantes na redução de consumo, que neste caso representam 20\%, entre os componentes construtivos observados in loco e os componentes mais eficientes propostos para comparação. É verificada a suavização da curva de consumo entre o segundo conjunto de simulações, retratado pelos cenários 13 a 24, em relação ao primeiro conjunto, de 1 a 12, o que sinaliza a importância da envoltória e da redução da ação da radiação solar sobre os ambientes internos da edificação. A escala de consumo entre as edificações é acentuada nestes cenários, dada a semelhança do comportamento das curvas, porém, com amplitudes de 50 MW para o modelo de 8 pavimentos, e quase 200 MW para o modelo de 19 pavimentos.\vspace*{0.3cm} \newline
\noindent A influência destes componentes sobre o consumo final, junto ao vidro com baixa emissividade, atinge  o  patamar  de  26,72\%  para  a  análise  do  resultado  isolado,  como  exposto  na  Figura  \ref{fig:figure26}. Verifica-se, também, que as mudanças provocadas pela implementação das estratégias passivas, ao  longo  do  processo  de  otimização,  reduzem  a  relevância  do  vidro  com  menor  transmitância térmica para a composição construtiva e performance energética dos modelos.
\begin{figure}[H]
    \centering
    \caption{Gráficos dos blocos de simulação de paredes, 3, e coberturas, 4, dos modelos genéricos de 8 (esq.) e 19 pavimentos (dir.).}
    \begin{subfigure}[b]{0.49\textwidth}
        \includegraphics[width=\textwidth]{figures/result/fig29-bloco5.png}
    \end{subfigure}
    \begin{subfigure}[b]{0.49\textwidth}
        \includegraphics[width=\textwidth]{figures/result/fig30-bloco5.png}
    \end{subfigure}
    \begin{subfigure}[b]{0.49\textwidth}
        \includegraphics[width=\textwidth]{figures/result/fig31-bloco6.png}
    \end{subfigure}
    \begin{subfigure}[b]{0.49\textwidth}
        \includegraphics[width=\textwidth]{figures/result/fig32-bloco6.png}
    \end{subfigure}
    \begin{flushleft}
        \par \small Fonte: autor, (2020).
    \end{flushleft}
    \label{fig:figure26}
\end{figure}
\noindent As alterações sobre as variáveis do Percentual de Área de Abertura da Fachada Total, presentes na Figura \ref{fig:figure27}, representaram pouco impacto sobre o consumo final total quando comparado com as outras medidas implementadas. Entretanto, cabe citar que esta variável, quando observada isoladamente, corresponde a mesma grandeza de redução de consumo que as variáveis de componentes construtivos, reduzindo o consumo nos cenários simulados em cerca 18,95\% e 26,71\% para as edificações de 8 e 19 pavimentos, respectivamente. 
\begin{figure}[H]
    \centering
    \caption{Gráficos dos blocos de simulação de PAF\textsubscript{T} dos modelos genérico de 8 (esq.) e 19 pavimentos (dir.).}
    \begin{subfigure}[b]{0.49\textwidth}
        \includegraphics[width=\textwidth]{figures/result/fig33-bloco4.png}
    \end{subfigure}
    \begin{subfigure}[b]{0.49\textwidth}
        \includegraphics[width=\textwidth]{figures/result/fig34-bloco4.png}
    \end{subfigure}
    \begin{flushleft}
        \par \small Fonte: autor, (2020).
    \end{flushleft}
    \label{fig:figure27}
\end{figure}
A implementação das proteções solares, Figura \ref{fig:figure28}, apresentou redução de 6,96\% sobre o consumo total entre os blocos de simulação 5 e 6. Além desta redução, nota-se o estreitamento entre os cenários 12 e 13 neste bloco, em 23,52\%, reduzindo em 3,19\% a em relação aos blocos anteriores. Nota-se também a suavização da curva nos cenários 13 a 24 em ambas as simulações do modelo de 8 pavimentos, o que sugere a influência sobre o consumo ao controlar a quantidade de luz e de radiação no ambiente, por meio da implantação de proteções solares e de vidros mais eficientes.
\begin{figure}[H]
    \centering
    \caption{Bloco de simulações de protetores solares dos modelos genérico de 8 (esq.) e 19 pavimentos (dir.).}
    \begin{subfigure}[b]{0.49\textwidth}
        \includegraphics[width=\textwidth]{figures/result/fig35-bloco7.png}
    \end{subfigure}
    \begin{subfigure}[b]{0.49\textwidth}
        \includegraphics[width=\textwidth]{figures/result/fig36-bloco7.png}
    \end{subfigure}
    \begin{subfigure}[b]{0.49\textwidth}
        \includegraphics[width=\textwidth]{figures/result/fig37-bloco8.png}
    \end{subfigure}
    \begin{subfigure}[b]{0.49\textwidth}
        \includegraphics[width=\textwidth]{figures/result/fig38-bloco8.png}
    \end{subfigure}
    \begin{subfigure}[b]{0.49\textwidth}
        \includegraphics[width=\textwidth]{figures/result/fig39-bloco9.png}
    \end{subfigure}
    \begin{subfigure}[b]{0.49\textwidth}
        \includegraphics[width=\textwidth]{figures/result/fig40-bloco9.png}
    \end{subfigure}
    \begin{flushleft}
        \par \small Fonte: autor, (2020).
    \end{flushleft}
    \label{fig:figure28}
\end{figure}
\subsection{Geração de energia solar fotovoltaica}
\noindent Os modelos genéricos mais eficientes reduziram globalmente o consumo de energia elétrica, quando comparados com os modelos sem otimização, em cerca de 39,76\% para o modelo genérico otimizado mais eficiente, de 8 pavimentos, e 57,07\% para o modelo genérico otimizado mais eficiente, de 19 pavimentos. A partir destes resultados, foram selecionados os modelos mais eficientes entre os cenários simulados para cada tipologia e foram simuladas a implementação dos sistemas de produção de energia solar fotovoltaica.\vspace*{0.3cm} \newline
\noindent Observa-se que a energia elétrica gerada ao longo do ano, supriu a demanda de energia da edificação de 8 pavimentos para metade do ano e ligeiramente igualado em dois meses, abril e junho, como demonstrado no Gráfico 7. Anualmente, esta diferença é reduzida, produzindo maior energia do que a demanda anual, atingindo o balanço energético nulo, como apresentado no Gráfico 9. Nota-se, também, a proporção dos sistemas de condicionamento de ar – HVAC, iluminação e equipamentos em relação à produção de energia, correspondendo entre 30 a 40\% da composição do consumo para cada sistema avaliado. Os resultados com as configurações detalhadas dos equipamentos especificados estão descritos no Anexo I e II deste trabalho.
\begin{figure}[H]
    \centering
    \caption{Sistemas, consumo e produção de energia elétrica do modelo genérico de 8 pavimentos.}
    \includegraphics[width=1.0\textwidth]{figures/result/fig41-consumomod.png}
    \begin{flushleft}
        \par \small Fonte: autor, (2020).
    \end{flushleft}
    \label{fig:figure29}
\end{figure}
\noindent O desempenho do sistema de produção de energia fotovoltaica do modelo de 19 pavimentos atendeu à demanda anual. Nos meses de outubro e janeiro a demanda não foi correspondida, sendo quase nula a relação entre demanda e produção em dezembro, como observado no Gráfico 8. Observa-se que a proporção dos sistemas percebida no modelo de 8 pavimentos, em relação à produção de energia, é similarmente constatada neste modelo de 19 pavimentos.\vspace*{0.3cm} \newline
\noindent Os resultados de produção de energia para este modelo evidenciaram as características mais importantes para proporcionar maior geração de energia, como a área de fachada exposta a radiação solar e a relação entre a quantidade de pavimentos e a área de piso da edificação. Estes resultados de produção de energia em relação ao consumo foram importantes para indicar a potencialidade das edificações com dimensões como o modelo genérico proposto.
\begin{figure}[H]
    \centering
    \caption{Sistemas, consumo e produção de energia elétrica do modelo genérico de 19 pavimentos.}
    \includegraphics[width=1.0\textwidth]{figures/result/fig42-consumomod.png}
    \begin{flushleft}
        \par \small Fonte: autor, (2020).
    \end{flushleft}
    \label{fig:figure30}
\end{figure}
\vspace{-0.5cm} \noindent No Gráfico 9 e 14 são dispostas as médias mensais de consumo e produção de energia dos modelos genéricos otimizados de 8 e 19 pavimentos, respectivamente, assim como as médias anuais, as curvas tracejadas de média de produção de energia solar fotovoltaica e seus desvios padrões máximos e mínimos. Verificam-se os limites do sistema fotovoltaico para o modelo de 8 pavimentos, onde a relação quase nula entre geração e consumo de energia mensal é notada ao longo do ano simulado. Esta relação representa uma média anual de produção de energia superior ao consumo em 1,02\%, o que atesta o estado \textit{Zero Energy} para o modelo avaliado. Entretanto, a curva de desvio padrão mínimo mostra que em um cenário de pouca produção, o modelo seria dependente da energia fornecida pela concessionaria para atender as demandas mensais.\vspace{-0.3cm} 
\begin{figure}[H]
    \centering
    \caption{Consumo final mensal, Consumo vs. Produção de energia e Desvio Padrão das médias de produção de energia do modelo genérico otimizado de 8 pavimentos.}
    \includegraphics[width=1.0\textwidth]{figures/result/fig43-consumofinal.png}
    \begin{flushleft}
        \par \small Fonte: autor, (2020).
    \end{flushleft}
    \label{fig:figure31}
\end{figure}
\noindent Entretanto, para o cenário ótimo simulado para o modelo genérico otimizado de 19 pavimentos, percebe-se que não há a mesma dependência da Rede Pública para o fornecimento de energia. Este fato é constatado em um cenário onde a curva de desvio padrão mínimo seja a média de produção de energia do sistema sugerido, apresentando resultados superiores ao consumo na maior parte do ano.
\begin{figure}[H]
    \centering
    \caption{Consumo final mensal, Consumo vs. Produção de energia e Desvio Padrão das médias de produção de energia do modelo genérico otimizado de 19 pavimentos.}
    \includegraphics[width=1.0\textwidth]{figures/result/fig44-consumofinal.png}
    \begin{flushleft}
        \par \small Fonte: autor, (2020).
    \end{flushleft}
    \label{fig:figure32}
\end{figure}
\subsection{Análise de viabilidade econômica}
\noindent Os dados tarifários necessários para o cálculo de Custo Anual de Energia, CAE, foram elaborados conforme a atualização da concessionaria para o mês de janeiro de 2020. A edificação proposta é classificada como comercial e está alocada no subgrupo B3, modalidade tarifária convencional. Para o mês estudado, foi atribuído à Tarifa de Energia, TE, um acréscimo proveniente da bandeira amarela vigente. As Tarifas de Uso de Sistema de Distribuição, TUSD, e os impostos fiscais ICMS, PIS e COFINS para o período estão demonstrados na Tabela \ref{a}.
\begin{table}[H]
    \centering
    \small
    \caption{Tarifas e impostos para a modalidade tarifária convencional.}
    \begin{tabular}{ll}
    \hline
        \textbf{Tarifas e impostos}     &   \textbf{Valor}  \\\hline
        TE + Bandeira amarela           &   0,26484         \\
        TUSD                            &   0,27440         \\
        ICMS                            &   25\%            \\
        PIS+CONFINS                     &   1,57\%          \\\hline
    \end{tabular}
    \begin{flushleft}
        \par \small Fonte: autor, (2020).
    \end{flushleft}
    \label{a}
\end{table}
\noindent Para o cálculo de custo anual de energia, levou-se em consideração o consumo anual dos modelos otimizados de 8 e 19 pavimentos mais eficientes energeticamente, sendo 605.853 kWh/ano e 965.186 kWh/ano, seus respectivos consumos anuais de energia. Na Equação \ref{eq:eq11} é estabelecido o custo anual de energia sem impostos, considerando TE e TUSD.
\begin{align}\label{eq:eq11}
    &Custo_{8pav}=E_{consumida} \times (TE+TSUD)\nonumber \\
    &Custo_{8pav}=605.853 \times (0,26484+0,27440)\nonumber \\
    &Custo_{8pav}=326.700,17 \text{ reais}\nonumber \\
    &Custo_{19pav}=E_{consumida} \times (TE+TSUD)\nonumber \\
    &Custo_{19pav}=965.186 \times (0,26484+0,27440)\nonumber \\
    &Custo_{19pav}=520.166,89 \text{ reais}
\end{align}
\noindent Entretanto, os custos sofrem tributação de impostos pela distribuidora de energia, o que foi considerado para a composição real do custo anual de energia, como definido pela EDP e apresentado na Equação \ref{eq:eq12}.
\begin{align}\label{eq:eq12}
    &Custo_{8pav}=\frac{Custo_{s/imposto}}{1-(PIS+COFINS+ICMS)} \nonumber \\
    &Custo_{8pav}=\frac{326.700,17}{1-0,2675}\nonumber \\
    &Custo_{8pav}=444.913,75 \text{ reais}\nonumber \\
    &Custo_{19pav}=\frac{Custo_{s/imposto}}{1-(PIS+COFINS+ICMS)} \nonumber \\
    &Custo_{19pav}=\frac{520.166,89}{1-0,2675}\nonumber \\ 
    &Custo_{19pav}=708.384,70 \text{ reais}
\end{align}
\noindent Na Tabela \ref{tab:17} são apresentadas a quantidade de energia gerada no ano, assim como os custos dos sistemas de geração de energia solar fotovoltaica indicados, sendo que o custo de instalação por kWp dos sistemas foi orçado, em janeiro de 2020, em 784 reais para o filme fino Cd-Te \cite{Sorgato2018} e 3.682 reais para o sistema mono-Si sugerido, segundo a média de preço do mercado local.\newline
\begin{table}[H]
    \centering
    \caption{Produção de energia e custo de implantação dos sistemas fotovoltaicos.}
    \begin{tabular}{llll}
        \hline
        \multicolumn{4}{c}{\textbf{Modelo genérico de 8 pavimentos}}                                                                                                                                                                                                                \\ 
        \hline
        \textbf{Características}                                                        & \makecell[l]{\textbf{Cobertura e}\\ \textbf{estacionamento}}      & \makecell[l]{\textbf{Proteção}\\ \textbf{Solar}}          & \textbf{Fachada}                                          \\ 
        \hline
        Módulo                                                                          & \makecell[l]{SunPower \\SPR-E20-\\435-COM}                        & \makecell[l]{SunPower \\SPR-E20-\\435-COM}                & \makecell[l]{First Solar \\FS-4122-2}                     \\
        Inversor                                                                        & \makecell[l]{Fronius \\International \\IG Plus 150 V-3}           & \makecell[l]{Fronius \\International \\ECO 25.0-3-S}      & \makecell[l]{Fronius \\International \\IG Plus 150 V-3}   \\
        \makecell[l]{Energia gerada\\ (MWh/ano)}                                        & 236,40                                                            & 154,42                                                    & 215,65                                                    \\ 
        \hline
        \makecell[l]{Custo total por \\potência instalada (R\$)}                        & 552.300,00                                                        & 371.587,44                                                & 214.988,48                                                \\ 
        \hline
        \multicolumn{3}{l}{\makecell[l]{Energia total \\gerada (MWh/ano)}}                                                                                  & 606,47                                                                                                                \\
        \multicolumn{3}{l}{\makecell[l]{Custo total dos \\sistemas instalados (R\$)}}                                                                       & 1.138.875,92                                                                                                          \\ 
        \hline
        \multicolumn{4}{c}{\textbf{Modelo genérico de 19 pavimentos}}                                                                                                        \\
        \hline
        \textbf{Características}                                                        & \makecell[l]{\textbf{Cobertura e}\\ \textbf{estacionamento}}      & \textbf{Proteção Solar}                                   & \textbf{Fachada}                                          \\ 
        \hline
        Módulo                                                                          & \makecell[l]{SunPower \\SPR-E20-\\435-COM}                        & \makecell[l]{SunPower \\SPR-E20-\\435-COM}                & \makecell[l]{First Solar \\FS-4122-2}                     \\
        Inversor                                                                        & \makecell[l]{Fronius \\International \\ECO 27.0-3-S}              & \makecell[l]{Fronius \\International \\IG Plus 120 V-3}   & \makecell[l]{Fronius \\International \\IG Plus 150 V-3}   \\
        \makecell[l]{Energia gerada\\ (MWh/ano)}                                        & 236,40                                                            & 370,02                                                    & 397,12                                                    \\ 
        \hline
        \makecell[l]{Custo total por \\potência instalada (R\$)}                        & 552.300,00                                                        & 896.198,80                                                & 376.084,80                                                \\ 
        \hline
        \multicolumn{3}{l}{\makecell[l]{Energia total \\gerada (MWh/ano)}}                                                                                  & 1003,54                                                                                                               \\
        \multicolumn{3}{l}{\makecell[l]{Custo total dos \\sistemas instalados (R\$)}}                                                                       & 1.824.583,60                                                                                                          \\
        \hline
    \end{tabular}
    \begin{flushleft}
        \par \small Fonte: autor, (2020).
    \end{flushleft}
\end{table}
\noindent Os valores de retorno, V\textsubscript{r}, dos custos de implementação dos sistemas foram calculados com base na taxa de atratividade Selic de 4,5\% para o período de janeiro de 2020, representada pela incógnita i. Além disso, foram utilizados o custo anual de energia, A, e a vida útil especificada pela fabricante dos módulos utilizados, de 25 anos, representada pela incógnita \textit{n}.\newline
\begin{align}
    &V_{ret8pav}=444.913,75 \times \left\{\frac{1-\left[\frac{1}{(1+0,045)^25}\right]}{0,045}\right\} \nonumber \\
    &V_{ret8pav}=6.597.274,05 \text{ reais} \nonumber \\
    &V_{ret19pav}=708.384,70 \times \left\{\frac{1-\left[\frac{1}{(1+0,045)^25}\right]}{0,045}\right\} \nonumber \\
    &V_{ret19pav}=10.504.070,01 \text{ reais} \nonumber
\end{align}
\noindent O valor de retorno calculado dos investimentos é utilizado para estipular o valor presente, VP, de cada um dos sistemas propostos.
\begin{align}
    &V_{8pav}=6.597.274,05 - 1.138.875,92 = 5.458.398,13 \nonumber \\
    &V_{19pav}=10.504.070,01 - 1.824.583,60 = 8.679.486,41 \nonumber
\end{align}
\noindent Assim, constata-se que os sistemas de produção de energia propostos aos modelos genéricos são economicamente viáveis, visto que os VP’s são positivos. Da mesma forma, o \textit{payback} da implantação dos sistemas estudados é de 2,55 anos para o modelo de 8 pavimentos e 2,57 anos para o modelo de 19 pavimentos. Vale mencionar que as variações das taxas de energia e a perda de eficiência do sistema ao longo da vida útil não foram consideradas na avaliação de viabilidade econômica \cite{Sorgato2018}.\vspace*{0.3cm} \newline
\noindent Os resultados indicam, para o modelo de edificação adotado, a teórica viabilidade de utilização do conceito \textit{Zero Energy} para as edificações comerciais de escritório de pequeno, médio e grande porte para o recorte territorial utilizado. As condições climáticas e urbanas favoráveis, juntamente a adoção de materiais e componentes adequados ao cenário brasileiro, priorizando soluções que proporcionaram a máxima extração dos recursos energéticos on site foram essenciais para alcançar o balanço energético nulo.
\section{Considerações Finais}
\noindent O conceito \textit{Zero Energy} vem ganhando repercussão no cenário brasileiro nos últimos anos, visto que este conceito vai ao encontro à frequente discussão sobre a necessidade de diversificação da matriz elétrica e do significativo avanço na geração de energia solar fotovoltaica no Brasil. Estas são realidades cada vez mais próximas do consumidor comum. Simbolizando a importância do tema acerca da economia de energia elétrica nacional, o Ministério de Minas e Energia, por meio de portaria publicada em dezembro de 2018, lança o Plano Anual de Aplicação de Recursos (PAR-2018) do PROCEL. Em dezembro de 2019 foi publicada a chamada publica, fruto do PAR-2018. Esta chamada pública descreve o projeto “Concurso NZEB – Edificações \textit{Near Zero Energy Building}”, onde é proposta a construção de quatro edificações \textit{Near Zero Energy} em território nacional, a fim de monitorar e avaliar a aplicabilidade do conceito.\vspace*{0.3cm} \newline
\noindent O barateamento progressivo do custo de instalação e manutenção dos sistemas de produção de energia solar e de outras fontes menos populares, porém direcionadas a escala da edificação, são pontos positivos e um atrativo para o engajamento a respeito do balanço energético nulo. Entretanto, contrariamente a esta evolução de cenário acerca da energia solar, políticas públicas buscam penalizar a progressão deste setor com tributações sobre a energia gerada e sobre a comercialização de painéis em território nacional \cite{Warth2019a,Warth2019}. A mini e micro geração também demonstram potencial de desenvolvimento no Brasil, e em especial, no Espirito Santo, contando com usinas de geração solar de pequeno porte no interior do estado, além de regulamentações e normas que favorecem à implantação e expansão deste segmento no país.\vspace*{0.3cm} \newline
\noindent O caminho para a popularização dos métodos de redução de consumo de energia e da preocupação com a sustentabilidade no ambiente construído passa pela disseminação de conhecimento e desenvolvimento de conceitos levados a público.\vspace*{0.3cm} \newline
\noindent Este trabalho aponta para uma alternativa ao modo construtivo convencional, indicando formas de tornar a edificação eficiente e produtiva, do ponto de vista energético, utilizando estratégias que resultem em um ambiente construído mais eficaz. Para tal, buscou-se construir um cenário que viabilizasse a observação de evidências do balanço energético nulo de um edifício.\vspace*{0.3cm} \newline
\noindent Assim, foram verificadas dificuldades em levantar as características das edificações para a construção da tipologia genérica, uma vez que as fontes de informação sobre consumo de energia ou dados técnicos sobre o processo construtivo das edificações estudadas era de difícil acesso.\vspace*{0.3cm} \newline
\noindent A escolha da cidade de Vitoria para o desenvolvimento da pesquisa teve como objetivo facilitar a obtenção de dados, o que não se confirmou na maioria dos casos do levantamento em campo. As edificações de escritório por si só apresentam dificuldades em obtenção de informações, uma vez que possuem muitos proprietários e abriga uma complexidade maior em termos de número de usuários e horários de ocupação. Contudo, a quantidade de estudos em relação ao tema, tanto nacional quanto internacionalmente, facilitou o preenchimento de lacunas de informação para a modelagem da tipologia genérica proposta.\vspace*{0.3cm} \newline
\noindent Em seguida, a proposição de modelos computacionais que reunissem os atributos levantados em campo resultou em testes de ferramentas para a avaliação de confiabilidade e complexidade. O \textit{EnergyPlus}, juntamente ao \textit{plug-in OpenStudio}, se destacaram por serem ferramentas gratuitas, \textit{open-source}, e de ampla utilização no meio acadêmico. Entretanto, esta escolha se mostrou limitada em aspectos que exigiram análises mais específicas de variáveis, ou de simulações que requeriam maior poder de processamento por apresentar uma quantidade de dados acima da capacidade de processamento da ferramenta de simulação, limitando o alcance dos resultados pretendidos para a pesquisa.\vspace*{0.3cm} \newline
\noindent Inicialmente, a função destes modelos foi atestar a baixa eficiência energética das edificações observadas Vitória, servindo como objeto para melhorias dos aspectos construtivos e energéticos. No entanto, as restrições de equipamento e de tempo para simulação foram cruciais para a determinação de resultados mais abrangentes, exigindo adaptações e simplificações de processo de simulação e análise de dados para a validação do método proposto. A simulação paramétrica necessária para a análise aleatória dos cenários, de forma plena, foi impossibilitada, em parte, devido aos custos financeiros exigidos pela ferramenta. Os recursos computacionais disponíveis seriam incapazes de processar a grande quantidade de variáveis simultaneamente, o que resulta normalmente na utilização do processamento de dados via nuvem. Este recurso é oferecido pela ferramenta, porém, por meio de um servidor pago estrangeiro, o que tornou financeiramente inviável a adoção deste tipo de simulação para a metodologia.\vspace*{0.3cm} \newline
\noindent Contudo, os resultados obtidos se mostraram satisfatórios para constatar a potencialidade de adoção do conceito \textit{Zero Energy} para as edificações propostas, assim como para os futuros edifícios do recorte territorial utilizado. O estudo econômico sobre a implantação dos sistemas propostos de produção de energia solar fotovoltaica e o resultado positivo reforça o alcance e a viabilidade de implementação da tecnologia.\vspace*{0.3cm} \newline
\noindent Porém, foi observado que edificações de menor porte, como o modelo de 8 pavimentos utilizado nesta pesquisa, podem enfrentar dificuldades em atingir o estado \textit{Zero Energy}, principalmente se estas construções estiverem em meio a outras com altura maior ou mais próximas, dificultando a produção de energia solar. Observou-se que a área de fachada e de piso são características importantes na produção de energia das tipologias estudadas, contrariando o senso comum para uma edificação com 19 pavimentos e toda a demanda energética que esta apresenta, o que impediria seu balanço energético nulo.\vspace*{0.3cm} \newline
\noindent Os parâmetros de construção dos modelos computacionais foram definidos de forma a tornar suas atribuições genéricas, o que resultou na potencialidade de autonomia energética dos edifícios de escritório de Vitória. Além das atribuições genéricas, os resultados alcançados nesta pesquisa representam o objetivo final de proporcionar ferramentas e dados para incentivar o estudo do tema e dar suporte a futuros trabalhos.\vspace*{0.3cm} \newline
\noindent O processo de elaboração desta pesquisa demonstra que a forma como o consumo de energia é abordada hoje é passível de reavaliação. Esta reavaliação é importante principalmente em aspectos práticos, como na utilização de equipamentos energeticamente eficientes, de investimento em tecnologias de maior eficiência energética aplicadas ao ambiente construído e a reeducação dos usuários destes espaços. Estas medidas são parte integrante de um processo maior de configuração e de adaptação do ambiente edificado à conscientização do uso racional de energia.\vspace*{0.3cm} \newline
\noindent Por fim, a continuidade deste trabalho é fundamental para refinar os resultados alcançados, estendendo a metodologia criada a outras tipologias como a residencial, às edificações públicas, ou relacionar o balanço energético nulo a um conjunto de edificações, estudando diferentes escalas de consumo e produção de energia, por exemplo. Estudos de implementação do conceito \textit{Zero Energy} direcionados a comunidade em situação de vulnerabilidade socioeconômica, aplicado a habitações de interesse social, também representam tópicos relevantes a um estudo aprofundado. A integração de novas ferramentas ao ato de projetar, principalmente as que consideram o planejamento do uso e geração de energia do edifício, é significativo como complemento ao processo construtivo praticado no Brasil.\pagebreak

%\bibliography{ref.bib}
\pagebreak
\printbibliography

\section{Apêndices}
\subsection{Apêndice A}
\begin{table}[H]
    \centering
    \caption{Levantamento das edificações de escritório de Vitória.}
    \begin{tabular}{l}
        \includegraphics[width=0.9\textwidth]{figures/appendices/edificio01.png}
    \end{tabular}
    \label{tab:18}
\end{table}
\pagebreak
\begin{table}[H]
    \centering
    \begin{tabular}{l}
        \includegraphics[width=\textwidth]{figures/appendices/edificio02.jpg}
    \end{tabular}
\end{table}
\pagebreak
\begin{table}[H]
    \centering
    \begin{tabular}{l}
        \includegraphics[width=\textwidth]{figures/appendices/edificio03.jpg}
    \end{tabular}
\end{table}
\pagebreak
\begin{table}[H]
    \centering
    \begin{tabular}{l}
        \includegraphics[width=\textwidth]{figures/appendices/edificio04.jpg}
    \end{tabular}
\end{table}
\pagebreak
\begin{table}[H]
    \centering
    \begin{tabular}{l}
        \includegraphics[width=\textwidth]{figures/appendices/edificio05.jpg}
    \end{tabular}
\end{table}
\pagebreak
\begin{table}[H]
    \centering
    \begin{tabular}{l}
        \includegraphics[width=\textwidth]{figures/appendices/edificio06.jpg}
    \end{tabular}
\end{table}
\pagebreak
\begin{table}[H]
    \centering
    \begin{tabular}{l}
        \includegraphics[width=\textwidth]{figures/appendices/edificio07.jpg}
    \end{tabular}
\end{table}
\pagebreak
\begin{table}[H]
    \centering
    \begin{tabular}{l}
        \includegraphics[width=\textwidth]{figures/appendices/edificio08.jpg}
    \end{tabular}
\end{table}
\pagebreak
\begin{table}[H]
    \centering
    \begin{tabular}{l}
        \includegraphics[width=\textwidth]{figures/appendices/edificio09.jpg}
    \end{tabular}
\end{table}
\pagebreak
\begin{table}[H]
    \centering
    \begin{tabular}{l}
        \includegraphics[width=\textwidth]{figures/appendices/edificio10.jpg}
    \end{tabular}
\end{table}
\subsection{Apêndice B}
\begin{table}[H]
    \centering
    \caption{Resultados de otimização para o modelo genérico de 8 pavimentos.}
    \begin{tabular}{l}
        \includegraphics[width=0.85\textwidth]{figures/appendices/tabela01.png}
    \end{tabular}
    \label{tab:19}
\end{table}
\pagebreak
\begin{table}[H]
    \centering
    \begin{tabular}{l}
        \includegraphics[width=0.9\textwidth]{figures/appendices/tabela02.png}
    \end{tabular}
\end{table}
\pagebreak
\begin{table}[H]
    \centering
    \begin{tabular}{l}
        \includegraphics[width=0.9\textwidth]{figures/appendices/tabela03.png}
    \end{tabular}
\end{table}
\pagebreak
\begin{table}[H]
    \centering
    \begin{tabular}{l}
        \includegraphics[width=\textwidth]{figures/appendices/tabela04.png}
    \end{tabular}
\end{table}
\subsection{Apêndice C}
\begin{table}[H]
    \centering
    \caption{Resultados de otimização para o modelo genérico de 19 pavimentos.}
    \begin{tabular}{l}
        \includegraphics[width=0.9\textwidth]{figures/appendices/tabela05.png}
    \end{tabular}
    \label{tab:20}
\end{table}
\pagebreak
\begin{table}[H]
    \centering
    \begin{tabular}{l}
        \includegraphics[width=0.9\textwidth]{figures/appendices/tabela06.png}
    \end{tabular}
\end{table}
\pagebreak
\begin{table}[H]
    \centering
    \begin{tabular}{l}
        \includegraphics[width=0.9\textwidth]{figures/appendices/tabela07.png}
    \end{tabular}
\end{table}
\pagebreak
\begin{table}[H]
    \centering
    \begin{tabular}{l}
        \includegraphics[width=0.9\textwidth]{figures/appendices/tabela08.png}
    \end{tabular}
\end{table}
\section{Anexos}
\subsection{Anexo I}
\begin{table}[H]
    \centering
    \caption{Resultados das simulações de produção de energia no \textit{PVSyst} para o modelo genérico de 8 pavimentos.}
    \begin{tabular}{l}
        \includegraphics[width=0.9\textwidth]{figures/attachments/resultpv1.jpg}
    \end{tabular}
    \label{tab:21}
\end{table}
\pagebreak
\begin{table}[H]
    \centering
    \begin{tabular}{l}
        \includegraphics[width=\textwidth]{figures/attachments/resultpv2.jpg}
    \end{tabular}
\end{table}
\pagebreak
\begin{table}[H]
    \centering
    \begin{tabular}{l}
        \includegraphics[width=\textwidth]{figures/attachments/resultpv3.jpg}
    \end{tabular}
\end{table}
\pagebreak
\begin{table}[H]
    \centering
    \begin{tabular}{l}
        \includegraphics[width=\textwidth]{figures/attachments/resultpv4.jpg}
    \end{tabular}
\end{table}
\pagebreak
\begin{table}[H]
    \centering
    \begin{tabular}{l}
        \includegraphics[width=\textwidth]{figures/attachments/resultpv5.jpg}
    \end{tabular}
\end{table}
\pagebreak
\begin{table}[H]
    \centering
    \begin{tabular}{l}
        \includegraphics[width=\textwidth]{figures/attachments/resultpv6.jpg}
    \end{tabular}
\end{table}
\pagebreak
\begin{table}[H]
    \centering
    \begin{tabular}{l}
        \includegraphics[width=\textwidth]{figures/attachments/resultpv7.jpg}
    \end{tabular}
\end{table}
\pagebreak
\begin{table}[H]
    \centering
    \begin{tabular}{l}
        \includegraphics[width=\textwidth]{figures/attachments/resultpv8.jpg}
    \end{tabular}
\end{table}
\pagebreak
\begin{table}[H]
    \centering
    \begin{tabular}{l}
        \includegraphics[width=\textwidth]{figures/attachments/resultpv9.jpg}
    \end{tabular}
\end{table}
\pagebreak
\begin{table}[H]
    \centering
    \begin{tabular}{l}
        \includegraphics[width=\textwidth]{figures/attachments/resultpv10.jpg}
    \end{tabular}
\end{table}
\pagebreak
\begin{table}[H]
    \centering
    \begin{tabular}{l}
        \includegraphics[width=\textwidth]{figures/attachments/resultpv11.jpg}
    \end{tabular}
\end{table}
\pagebreak
\begin{table}[H]
    \centering
    \begin{tabular}{l}
        \includegraphics[width=\textwidth]{figures/attachments/resultpv12.jpg}
    \end{tabular}
\end{table}
\pagebreak
\begin{table}[H]
    \centering
    \begin{tabular}{l}
        \includegraphics[width=\textwidth]{figures/attachments/resultpv13.jpg}
    \end{tabular}
\end{table}
\pagebreak
\begin{table}[H]
    \centering
    \begin{tabular}{l}
        \includegraphics[width=\textwidth]{figures/attachments/resultpv14.jpg}
    \end{tabular}
\end{table}
\pagebreak
\begin{table}[H]
    \centering
    \begin{tabular}{l}
        \includegraphics[width=\textwidth]{figures/attachments/resultpv15.jpg}
    \end{tabular}
\end{table}
\pagebreak
\begin{table}[H]
    \centering
    \begin{tabular}{l}
        \includegraphics[width=\textwidth]{figures/attachments/resultpv16.jpg}
    \end{tabular}
\end{table}
\pagebreak
\begin{table}[H]
    \centering
    \begin{tabular}{l}
        \includegraphics[width=\textwidth]{figures/attachments/resultpv17.jpg}
    \end{tabular}
\end{table}
\pagebreak
\begin{table}[H]
    \centering
    \begin{tabular}{l}
        \includegraphics[width=\textwidth]{figures/attachments/resultpv18.jpg}
    \end{tabular}
\end{table}
\pagebreak
\begin{table}[H]
    \centering
    \begin{tabular}{l}
        \includegraphics[width=\textwidth]{figures/attachments/resultpv19.jpg}
    \end{tabular}
\end{table}
\pagebreak
\begin{table}[H]
    \centering
    \begin{tabular}{l}
        \includegraphics[width=\textwidth]{figures/attachments/resultpv20.jpg}
    \end{tabular}
\end{table}
\vspace*{-0.3cm}
\subsection{Anexo II}
\begin{table}[H]
    \centering
    \caption{Resultados das simulações de produção de energia no \textit{PVSyst} para o modelo genérico de 19 pavimentos.}
    \begin{tabular}{l}
        \includegraphics[width=\textwidth]{figures/attachments/resultpv21.jpg}
    \end{tabular}
    \label{tab:22}
\end{table}
\pagebreak
\begin{table}[H]
    \centering
    \begin{tabular}{l}
        \includegraphics[width=\textwidth]{figures/attachments/resultpv22.jpg}
    \end{tabular}
\end{table}
\pagebreak
\begin{table}[H]
    \centering
    \begin{tabular}{l}
        \includegraphics[width=\textwidth]{figures/attachments/resultpv23.jpg}
    \end{tabular}
\end{table}
\pagebreak
\begin{table}[H]
    \centering
    \begin{tabular}{l}
        \includegraphics[width=\textwidth]{figures/attachments/resultpv24.jpg}
    \end{tabular}
\end{table}
\pagebreak
\begin{table}[H]
    \centering
    \begin{tabular}{l}
        \includegraphics[width=\textwidth]{figures/attachments/resultpv25.jpg}
    \end{tabular}
\end{table}
\pagebreak
\begin{table}[H]
    \centering
    \begin{tabular}{l}
        \includegraphics[width=\textwidth]{figures/attachments/resultpv26.jpg}
    \end{tabular}
\end{table}
\pagebreak
\begin{table}[H]
    \centering
    \begin{tabular}{l}
        \includegraphics[width=\textwidth]{figures/attachments/resultpv27.jpg}
    \end{tabular}
\end{table}
\pagebreak
\begin{table}[H]
    \centering
    \begin{tabular}{l}
        \includegraphics[width=\textwidth]{figures/attachments/resultpv28.jpg}
    \end{tabular}
\end{table}
\pagebreak
\begin{table}[H]
    \centering
    \begin{tabular}{l}
        \includegraphics[width=\textwidth]{figures/attachments/resultpv29.jpg}
    \end{tabular}
\end{table}
\pagebreak
\begin{table}[H]
    \centering
    \begin{tabular}{l}
        \includegraphics[width=\textwidth]{figures/attachments/resultpv30.jpg}
    \end{tabular}
\end{table}
\pagebreak
\begin{table}[H]
    \centering
    \begin{tabular}{l}
        \includegraphics[width=\textwidth]{figures/attachments/resultpv31.jpg}
    \end{tabular}
\end{table}
\pagebreak
\begin{table}[H]
    \centering
    \begin{tabular}{l}
        \includegraphics[width=\textwidth]{figures/attachments/resultpv32.jpg}
    \end{tabular}
\end{table}
\pagebreak
\begin{table}[H]
    \centering
    \begin{tabular}{l}
        \includegraphics[width=\textwidth]{figures/attachments/resultpv33.jpg}
    \end{tabular}
\end{table}
\pagebreak
\begin{table}[H]
    \centering
    \begin{tabular}{l}
        \includegraphics[width=\textwidth]{figures/attachments/resultpv34.jpg}
    \end{tabular}
\end{table}

\end{document}