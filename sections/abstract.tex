\section*{Abstract}
    \thispagestyle{empty}
    The  energy consumption in using a building is growing in constant
    pace  in  the  last  decades, coming from the industrial development and 
    technological revolution which comes together with this movement. 
    The emission of Greenhouse Gas and climate change are consequences of these 
    scenario  of  development  and  consumption.  Allied  to  these  factors,  
    buildings  contribute  to  the worsening of this scenario, since the use of 
    these  constructions mean negative  impacts on  the environment.  In  contrast,  
    energy  efficient  buildings  have  become  a  must-do  for  planning  
    new built  environments,  changing  the  way  the  community  perceives  
    the  relationship  between  the building and energy consumption. This work 
    aims to study the potential application of the Zero Energy concept for commercial 
    buildings in Vitória, in order to verify the validity of the method for the 
    Brazilian construction scenario and contribute to the dissemination of this 
    form of building planning. Therefore, this study was developed based on three 
    major stages, as the first consisted of surveying the buildings within a 
    pre-established territorial outline, selecting the most frequent construction  
    and  architectural  characteristics  among  them  and  build  representative  
    models  of the  observed  scenario;  the  second  stage  consisted  of  subjecting  
    the  representative  models  to computer  simulations  to  show  the  energy  
    performance  of  the  buildings  surveyed,  ways  to optimize  and,  
    thus,  reduce  the  energy  consumption  of  the  models  and  forms  of  solar  
    energy production  applied  to  the  reference  buildings;  and  finally,  
    the  third  stage,  in  which  the  results were evaluated and the economic 
    feasibility of implementing the energy production system were done.  
    The  results  showed  that  the  strategies  for  implementing  more  efficient  
    air  conditioning systems,  equipment  and  lighting  are  very  important  
    for  energy  savings.  The  combination  of construction  and  architectural  
    modifications  for  materials  with  higher  energy  performance provide the 
    environment to achieve zero energy or near zero energy states. 
    These results indicate that the adoption of this way of thinking about building 
    is feasible and increasingly accessible to the community.\\

    \noindent Keywords: Zero Energy Buildings; near Zero Energy Buildings; office buildings.
    \pagebreak