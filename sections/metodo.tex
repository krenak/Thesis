\section{Metodologia}
\begin{onehalfspace}
    Assim, neste capítulo são apresentadas as três principais etapas utilizadas na 
metodologia para esta pesquisa. Estas etapas podem ser descritas como:
\begin{easylist}
    \ListProperties(Numbers1=r,Numbers2=l,Hide2=1,Hang=true,Margin1=3ex,Margin2=6ex,FinalSpace=1em)
    @ Definição dos modelos genéricos. A etapa de definição dos modelos foi elaborada em 3 partes, 
    dentre as quais:
        @@ Coleta    de    dados    sobre    as    características    das    edificações    comerciais, 
        especificamente de escritório, em Vitória (ES);
        @@ Levantamento e definição das variáveis sobre os padrões de uso e ocupação das 
        salas  de  escritório,  assim  como  padrões  de  conforto  e  níveis  de  
        eficiência energética dos equipamentos de condicionamento de ar e iluminação;
        @@ Estabelecimento  dos  modelos  genéricos com  o  intuito de  evidenciar o  
        consumo total  final  de  energia  elétrica  por  meio  da  determinação  da  classe  
        de  eficiência energética  da  edificação,  proposta  pela  INI-C,  e  o  
        potencial  de  otimização  e produção de energia elétrica a partir de fontes renováveis.
    @ Simulações.  Nesta  etapa  são  avaliadas  as  características  mais  influentes  
    no  consumo energético da edificação de referência e o potencial de geração de energia 
    solar. Ambas as  avaliações  serão  feitas  por  meio  de  simulação  computacional.  
    As  simulações  foram fracionadas em 3 partes, dentre as quais:
        @@ Simulação dos modelos real e de referência, onde é feita a determinação da classe 
        de desempenho energético das edificações observadas em campo; 
        @@ Otimização    dos    modelos    genéricos,    representando    a    etapa    onde    
        são implementadas estratégias passivas e ativas visando a eficientização da edificação;
\end{easylist}
\vspace*{0.3cm}
\noindent Com a aplicação da metodologia, simplificada no Fluxograma da Figura \ref{fig:figura7}, busca-se identificar se há 
condições para o balanço energético nulo total ou parcial dos modelos analisados.
\begin{figure}[ht]
    \centering
    \footnotesize
    \caption{\small Esquema simplificado da metodologia criada.}
    \includegraphics[width=1\textwidth]{figures/fig8_Fluxogramas_1.jpg}
    \begin{flushleft}
        \par \small Fonte: autor (2019).
    \end{flushleft}
    \label{fig:figura7}
\end{figure}\vspace*{-0.3cm}
\subsection{Levantamento das características dos edifícios de escritório de Vitória}
As edificações de escritório de Vitória selecionadas após a definição do recorte territorial, apresentam características que foram complementadas aos atributos observados em edificações comerciais brasileiras, como suporte as informações não encontradas in loco. As características com maior frequência de ocorrência no levantamento realizado são apresentadas na Tabela \ref{tab:tabela4}. Todavia, a amostra coletada abrange edifícios iniciados em 2003 e concluídos até o fim do primeiro trimestre de 2018, data do início do levantamento. Este fato inviabiliza aplicar a última revisão do Plano à amostra.\newline \vspace*{0.3cm}
Foram considerados para o levantamento atributos como gabarito, número de pavimentos-tipo, número de salas por pavimento-tipo, dimensão e forma, altura dos pavimentos-tipo. Não foram consideradas as dimensões dos lotes onde as edificações da amostra estacam implantadas, já que este atributo não foi pertinente ao objetivo do trabalho. Estes parâmetros foram reunidos em consulta ao material técnico disponibilizado pelas construtoras, visitas a campo e complementação de dados utilizando a ferramenta computacional \textit{Google Street View}.\newline \vspace*{0.3cm}
Os trabalhos de \textcite{Lamberts2006,AmericanSocietyofHeatingRefrigeratingandAir-ConditioningEngineers-ASHRAE2010,Bernabe2012,Ramos2013,Didone2014,Didone2014a,ConselhoBrasileirodeConstrucaoSustentavel-CBCS2015,Fonseca2016,Werneck2017,InstitutoNacionaldeMetrologiaNormalizacaoeQualidadeIndustrial-INMETRO2018}, foram utilizados como principais fontes de informação para a análise de envoltória e sistema de iluminação e condicionamento de ar.%\vspace{-0.25cm}
    \begin{table}[H]
        \centering
        \small
        \caption{Características observadas em campo e em pesquisas anteriores.}
        \begin{tabular*}{\columnwidth}{@{\extracolsep{\fill}}lll}
        \hline
        \textbf{Parâmetro}                                             & \textbf{Descrição}                                                                    & \textbf{Referências} \\ \hline
        Gabarito                                                       & 24 a 60 m (8 a 19 pav.)                                                               & \makecell[l]{Levantamento \textit{in loco} e referências\\ (BERNABÉ, 2012; CBCS, 2015; \\FONSECA et al., 2016; LAMBERTS; \\GHISI; RAMOS, 2006; RAMOS \\et al., 2013).} \\ \hline
        Altura do pavimento                                            & 3 m                                                                                   & Levantamento \textit{in loco}.                                                                                                                                         \\ \hline
        Planta-baixa (forma)                                           & Retangular                                                                            & \makecell[l]{Levantamento \textit{in loco} e referências\\ (FONSECA et al., 2016; INMETRO,\\ 2018).}                                                                   \\ \hline
        \makecell[l]{Dimensão das salas\\ por pav.-tipo}               & 40 m²                                                                                 & \makecell[l]{Foi fixado a área das salas \\(zonas térmicas) de acordo com a \\média de ofertas de salas observadas\\ em levantamento \textit{in loco}.}                  \\ \hline
        \multicolumn{3}{c}{Continua}\\\hline
    \end{tabular*}
    \label{tab:tabela4}
    \end{table}\pagebreak
    \begin{table}[H]
        \centering
        \small
        \begin{tabular*}{\columnwidth}{@{\extracolsep{\fill}}lll}
        \hline
        \multicolumn{3}{c}{Conclusão}\\\hline
        \makecell[l]{Componentes da\\ parede}                          & \makecell[l]{Bloco cerâmico,\\ 8 furos; \\14x19x29 cm; \\argamassa de\\ assentamento} & Levantamento \textit{in loco}.                                                                                                                                         \\ \hline
        Proteção solar                                                 & Sem proteção                                                                          & \makecell[l]{Levantamento \textit{in loco} e referências\\ (FONSECA et al., 2016; WERNECK \\et al., 2017).}                                                            \\ \hline
        Cobertura                                                      & \makecell[l]{Laje impermeabilizada\\ com 20 cm de\\ espessura}                        & \makecell[l]{Levantamento \textit{in loco} e referências \\(CB3E; ABIVIDRO, 2015).}                                                                                    \\ \hline
        Vidros                                                         & \makecell[l]{Laminado; Reflexivo;\\ 8 mm; Verde}                                      & \makecell[l]{(FONSECA et al., 2016; \\INMETRO, 2018).}                                                                                                                   \\ \hline
        PAF\textsubscript{T}                                           & 30\%; 50\%; 80\%                                                                      & \makecell[l]{Levantamento \textit{in loco}\\ e referências.}                                                                                                             \\ \hline
        \makecell[l]{Orientação solar da \\fachada principal}          & Sul                                                                                   & \makecell[l]{Levantamento \textit{in loco}\\ e referências.}                                                                                                             \\ \hline
        \makecell[l]{Densidade de Carga de\\ Iluminação Limite – DCIL} & 14,1 W/m²                                                                             & \makecell[l]{Consulta pública do RTQ-C \\(INMETRO, 2018).}                                                                                                               \\ \hline
        \makecell[l]{Densidade de Carga de\\ Equipamentos – DCE}       & 9,7 W/m²                                                                              & \makecell[l]{Consulta pública do RTQ-C \\(INMETRO, 2018).}                                                                                                               \\ \hline
        \makecell[l]{Absortância/transmitância \\das paredes}          & \makecell[l]{0,59 (cor \\camurça)/3,75}                                               & \makecell[l]{Valores consultados na \\NBR 15220-2 e referências \\(ABNT, 2003; FONSECA et \\al., 2016; INMETRO, 2018).}                                                                     \\ \hline
        \makecell[l]{Absortância/transmitância \\das coberturas}       & \makecell[l]{0,65 (concreto \\aparente)/2,06}                                         & \makecell[l]{Valores consultados na \\NBR 15220-2 e referências \\(ABNT, 2003; FONSECA et \\al., 2016; INMETRO, 2018).}                                                 \\ \hline
        \end{tabular*}
        \begin{flushleft}
            \par \small Fonte: autor (2019).
        \end{flushleft}
    \end{table}\vspace*{-0.3cm}

\noindent As informações coletadas nos estudos e em levantamento formam a base conceitual para determinar os aspectos arquitetônicos relevantes para compor os modelos genéricos e, posteriormente, determinar os parâmetros de otimização e do consumo energético padrão aproximado para um edifício de escritório. Além disso, a quantidade de simulações necessárias para determinar o consumo de energia dos modelos genéricos é identificada por meio da organização dos dados coletados em campo e, assim, a resultante do número de variáveis.

\subsubsection{Consumo de energia elétrica das edificações}
\begin{minipage}[H]{1\textwidth}
    O consumo de energia elétrica em edificações de escritório no Brasil é determinado principalmente por sua tipologia. As configurações predominantes no Brasil compreendem, em sua maioria, pequenas edificações, abaixo de 8 pavimentos, a edifícios grandes, acima de 15 pavimentos \cite{Carlo2008,Ramos2013,ConselhoBrasileirodeConstrucaoSustentavel-CBCS2015,Fonseca2016}.\vspace*{0.3cm} \newline
    Correlacionado às tipologias arquitetônicas, os dados de consumo de energia, expressos em Intensidade de Uso de Energia – IUE, kWh/m²/ano, foram necessários para validação das simulações iniciais quanto ao consumo de energia esperado dos modelos. Visto que a calibração dos modelos genéricos não foi possível, pois eles não dispunham de memorial de massa como ferramenta de comparação ao consumo simulado computacionalmente, foram adotados como método de comparação os valores médios registrados no Relatório Final do CBCS (2015), como mostra o Histograma no Gráfico 5. Foi relacionado o consumo de energia das edificações levantadas à frequência de ocorrência da quantidade de pavimentos de edificações comerciais brasileiras e suas respectivas áreas comuns.
\end{minipage}
    \begin{graph}
        \par \small Gráfico 5 - Consumo energético em edificações de escritório brasileiras.
        \begin{minipage}[ht]{1\textwidth}\centering
            \includegraphics[width=1.0\textwidth]{figures/fig9_consumo-total-das-edificacoes-levantadas_cbcs_2015.png}            
        \end{minipage}
        \begin{flushleft}
            \par \small Fonte: adaptado de CBCS (2015).
        \end{flushleft}
    \end{graph}

\noindent Logo, os dados de IUE adotados para a comparação e avaliação inicial do consumo energético dos modelos foram baseados nos valores máximo e mínimo estabelecidos pelo CBCS (2015). A média entre o consumo em áreas comuns e total relacionado à frequência de quantidade de pavimentos define a quantidade total de consumo de energia para cada tipologia.\vspace*{0.3cm} \newline
Os autores do Relatório atribuem 133 kWh/m²/ano aos edifícios de pequeno porte, e 268 kWh/m²/ano às edificações de grande porte. Vale ressaltar que os dados de consumo energético, assim como a média apontada no Relatório, de 191 kWh/m²/ano, desprezam as distorções causadas por edificações com particularidades de consumo como datacenters ou erros de cálculo de área útil.\vspace*{0.3cm} \newline
Ao estabelecer as Intensidades de Uso de Energia equivalentes a cada modelo, assim como os padrões de uso e ocupação, pode-se estimar o consumo de energia durante determinado período.

\subsubsection{Padrões de uso e ocupação em edifícios de escritório}
Os padrões de uso e ocupação da edificação foram baseados em normas, regulamentos, relatórios técnicos e referências acadêmicas que consideraram os níveis de atividades desenvolvidas nos ambientes, tratadas neste trabalho como zonas térmicas, como apresentado na Tabela \ref{tab:tabela5}.\vspace*{0.3cm} \newline
Como o intuito do trabalho foi criar modelos genéricos que representassem minimamente o cenário encontrado na cidade de Vitória, as características de uso e ocupação escolhidas foram integradas como forma de aproximar as tipologias ao cenário observado. Dentre elas, pode-se citar:
\begin{itemize}
    \item O nível metabólico apresentado em atividades de escritório;
    \item O horário de funcionamento dos escritórios, determinando os intervalos de tempo de ocupação total e parcial, onde, respectivamente, a capacidade máxima e parcial de ocupação das zonas térmicas é atingida;
    \item A densidade de pessoas por metro quadrado para cada zona térmica;
    \item A temperatura de conforto em cada zona térmica, de acordo com as normas de conforto térmico consultadas; e
    \item A umidade relativa do ar nos ambientes de escritório.
\end{itemize}

\noindent A obtenção dos dados acerca da atividade desempenhada nas edificações de escritório, além do horário de funcionamento e densidade de ocupação serão utilizados para estimar o consumo de energia elétrica do espaço utilizado por meio de simulação computacional.
\begin{table}[H]
    \centering
    \small
    \caption{Padrões de uso e ocupação}
    \begin{tabular*}{\columnwidth}{@{\extracolsep{\fill}}llll}
        \hline
        \textbf{Parâmetro}                                & \multicolumn{2}{c}{\textbf{Descrição}}                                                                                                                          & \textbf{Referências} \\ \hline
        \multirow{2}{*}{Atividades}                       & \makecell[l]{Escritório:\vspace*{0,2cm}\\Fator metabólico:\vspace*{0,2cm}} & \makecell[l]{Leve\vspace*{0,2cm}\\0,9 met}                                         & \makecell[{{p{5cm}}}]{As salas de escritório da cidade são utilizadas, em sua maioria, para atividades especializadas de âmbito jurídico, relacionadas à construção civil, a saúde e atividades financeiras.} \\ \hline
        \multirow{2}{*}{Horário de funcionamento}         & \makecell[l]{Ocupação\\ total:\vspace*{0,2cm}\\Ocupação \\parcial (50\%):\vspace*{0,6cm}}  & \makecell[l]{8h às 12h; \\13h às 18h \vspace*{0,6cm}\\12h às 13h}  & \makecell[{{p{5cm}}}]{Segundo normas e pesquisas sobre o horário de funcionamento de escritórios, o início da ocupação se dá às 6h e pode se estender até às 24h. Entretanto, visando a aproximação às condições praticadas no mercado brasileiro, adota-se a redução de ocupação durante o horário de almoço, denominada ocupação parcial.}  \\ \hline
        Densidade de ocupação                             & \multicolumn{2}{c}{0,14 pessoas/m²}                                                                                                                             & \makecell[{{p{5cm}}}]{(CONSELHO BRASILEIRO DE CONSTRUÇÃO SUSTENTÁVEL - CBCS, 2015; LAMBERTS; GHISI; RAMOS, 2006; MORAES; PEREIRA, 2014).} \\ \hline
        Temperatura de controle                           & \multicolumn{2}{c}{24°C}                                                                                                                                        & \makecell[{{p{5cm}}}]{Temperatura limite de acionamento do sistema de condicionamento de ar (ASHRAE, 2010, 2017a; INMETRO, 2010a).} \\ \hline
        \makecell[l]{Nível de iluminância\\ de referência}& \multicolumn{2}{c}{500 lux}                                                                                                                                     & \makecell[{{p{5cm}}}]{Iluminância mínima (entorno de trabalho) para atividades visuais (ASHRAE, 2010; ABNT, 2013).} \\ \hline
        \makecell[l]{Umidade Relativa\\ Interna}          & \multicolumn{2}{c}{40\%-60\%}                                                                                                                                   & \makecell[{{p{5cm}}}]{Faixa recomentada pela ASHRAE 55 (2017a).}\\ \hline
    \end{tabular*}
    \begin{flushleft}
        \par \small Fonte: autor (2019).
    \end{flushleft}
\end{table}
\vspace*{-0.5cm}
\subsection{Padrões de conforto}
O conforto térmico e de iluminação natural são parâmetros essenciais para a avaliação 
da qualidade do ambiente em que se vive. Não obstante, é necessário estabelecer padrões 
de conforto para a simulação de desempenho termoenergético do modelo genérico. Os 
próximos subitens tratam dos parâmetros adotados para a posterior inserção de dados 
e informações nos processos de simulação.
\subsubsection{Conforto térmico}
Utilizando os conceitos de conforto adaptativo, foi calculada a temperatura de conforto (t\textsubscript{c}) 
descrita pela \textcite{AmericanSocietyofHeatingRefrigeratingandAir-ConditioningEngineers-ASHRAE2017}, e assim estipulada a faixa de temperaturas de conforto e 
set point para controle de temperatura dos ambientes dos modelos genéricos. Este modelo 
avalia a adaptação do usuário em diferentes campos como o fisiológico, o comportamental 
e o psicológico \cite{AmericanSocietyofHeatingRefrigeratingandAir-ConditioningEngineers-ASHRAE2017a} e pode ser expresso pela Equação \ref{eq:eq1} tratada no Referencial Teórico.\vspace*{0.3cm} \newline
Foram utilizadas as temperaturas de bulbo seco máximas e mínimas como valores representantes 
da temperatura externa da edificação \cite{InstitutoNacionaldeMetereologia-INMET2018}. Estes dados de temperatura serão 
fundamentais para a caracterização do meio em que o modelo genérico será simulado, 
influenciando no desempenho simulado da envoltória e dos sistemas de condicionamento de ar 
e iluminação.
\subsubsection{Conforto lumínico}
A norma NBR/ISO CIE 8995-1 \cite{AssociacaoBrasileiradeNormasTecnicas-ABNT2013} determina que a condição de conforto visual 
em ambientes de escritório requer valor igual ou maior que 500 lux de iluminância de entorno 
imediato da tarefa a ser desempenhada \cite{AssociacaoBrasileiradeNormasTecnicas-ABNT2013,Ramos2013}. Assegurar a 
iluminância é uma condição importante para garantir o conforto, desempenho e segurança visual 
aos usuários, seja por fonte luminosa artificial ou natural.\vspace*{0.3cm} \newline
Para que a iluminação interna se mantenha dentro do padrão estabelecido por norma, foi 
necessário calcular o Fator de Luz Diurna – FDL, como forma de verificar se a adoção 
iluminância proposta pela norma seria adequada as condições empregadas aos modelos de 
referência. Este fator, uma vez incorporado nas rotinas de simulação computacional, 
influencia diretamente nas condições de iluminação no ambiente, indicando se estão adequadas 
para a realização de tarefas que exigem um determinado volume de iluminação.\vspace*{0.3cm} \newline
Considerando a ocupação anual parcial dos ambientes, foi possível estabelecer a relação 
entre a iluminância interna desejada, 500 lux, e a mediana da iluminância difusa horizontal 
externa, dado retirado do arquivo climático de Vitória \cite{InstitutoNacionaldeMetrologiaNormalizacaoeQualidadeIndustrial-INMETRO2018}, de 50019,16 lux. 
Assim, o FLD necessário para manter a iluminância interna das salas de escritório durante o 
período de ocupação levantado pode ser expresso pela Equação \ref{eq:eq2}.
\begin{equation}\label{eq:eq2}
    FLD=\frac{E_{interna}}{E_{externa}}*100
\end{equation}

Onde:\par
\setlength\parindent{1.5cm} \textit{FLD} é o Fluxo de Luz Diurna, em porcentagem;\par
\setlength\parindent{1.5cm} E\textsubscript{interna} é a iluminância interior em um ponto de um plano, em lux; e\par
\setlength\parindent{1.5cm} E\textsubscript{externa} é a iluminância externa simultânea em um plano horizontal, em lux.\vspace*{0.3cm} \newline
\noindent Concluída a caracterização dos padrões de conforto a serem avaliados, são definidos os 
parâmetros para os níveis de eficiência energética esperados dos equipamentos de 
condicionamento de ar e iluminação artificial presentes no modelo genérico. A partir desta 
definição embasada em regulamentos como o INI-C, é possível estimar a otimização de consumo 
energético da edificação proposta.
\subsection{Níveis de eficiência energética}
Os níveis de eficiência dos aparelhos utilizados na manutenção do conforto da edificação exercem importante papel quando há a necessidade de racionamento do consumo energético. Para tal, estes aparelhos são avaliados segundo critérios de desempenho energético e, ao final dessa avaliação, é atribuído um índice representativo da eficiência energética alcançada. Os índices de desempenho apresentados neste trabalho foram baseados na Instrução Normativa Inmetro para Classe de Eficiência Energética de Edificações Comerciais, de Serviços e Públicas – INI-C.

\subsubsection{Determinação da classe de eficiência energética dos modelos genéricos}
As primeiras simulações energéticas servem como meio de ajuste e verificação de erros entre os dados de saída de consumo energético e intensidade de uso de energia. Os resultados das simulações iniciais foram comparados às análises feitas de Determinação de Classe de Desempenho Energético, baseado na metodologia do INI-C, e nos resultados apresentados pelo Relatório Final de Desempenho Energético do CBCS. A partir da discrepância dos valores obtidos entre as simulações iniciais e as referências selecionadas, foram realizados ajustes a fim de adequar o modelo computacional à tolerância estabelecida.\vspace*{0.3cm} \newline
Os modelos genéricos foram avaliados segundo a definição da Instrução Normativa, onde é definido que o Modelo Real representa a edificação a ser classificada, enquanto o Modelo de Referencia representa a edificação com baixo desempenho energético. Desta forma, os edifícios propostos são comparados com um modelo de baixa performance, evidenciando, assim, seu desempenho energético. As variáveis analisadas para a comparação com o Modelo de Referência foram o consumo de energia térmica, elétrica e primaria dos modelos.\vspace*{0.3cm} \newline
A partir do diagnóstico de desempenho energética dos modelos genéricos, foram implementadas as otimizações e medidas de produção de energia em contraponto ao consumo energético constatado em simulação.

\subsubsection{Eficiência energética do sistema de condicionamento de ar}
Os sistemas de condicionamento de ar presentes em edifícios de escritório são compostos basicamente por dois tipos de equipamentos: ar-condicionado Split e o Sistema Central de Água Gelada – CAG \cite{ConselhoBrasileirodeConstrucaoSustentavel-CBCS2015}.\vspace*{0.3cm} \newline
Segundo o \textcite{InstitutoNacionaldeMetrologiaNormalizacaoeQualidadeIndustrial-INMETRO2018}, esses equipamentos são classificados segundo a capacidade de resfriamento e a potência absorvida pelos motores em pleno funcionamento. Esta relação é representada pelo Coeficiente de Performance – COP, expresso em W/W. O COP para equipamentos de ar condicionado é categorizado segundo uma Classe de Eficiência Energética – CEE, como apresentado na Tabela \ref{tab:tabela6}.

\begin{table}[ht]\centering
    \caption{\small  Relação entre o COP e as Classes de Eficiência Energética de Condicionadores de ar.}
    \vspace*{0.3cm}
    \label{tab:tabela6}
    \begin{tabular}{cccc}
        \hline
        \textbf{Classes}     & \multicolumn{3}{c}{\textbf{Densidade de Potência de Iluminação (DPI – W/m²)}}        \\ \hline
        A                    & \makecell[c]{3,23}       & \makecell[c]{\textless CEE}                  &            \\ \hline
        B                    & \makecell[c]{3,02}       & \makecell[c]{\textless CEE =\textless} & 3,23       \\ \hline
        C                    & \makecell[c]{2,81}       & \makecell[c]{\textless CEE =\textless} & 3,02       \\ \hline
        D                    & \makecell[c]{2,60}       & \makecell[c]{\textless CEE =\textless} & 2,81       \\ \hline
    \end{tabular}
    \begin{flushleft}
        \par \small Fonte: adaptado de INMETRO (2018).
    \end{flushleft}
\end{table}\vspace*{-0.5cm}

\noindent Esta classificação atribuída ao sistema de ar condicionado dos modelos genéricos, juntamente ao nível de eficiência energética do sistema de iluminação, é essencial para a etapa de simulação e identificação do consumo final de energia dos modelos genéricos, representantes do cenário observado dentro do recorte territorial.\vspace*{0.3cm} \newline
Como o sistema mais frequente observado nas edificações levantadas foi o Sistema Central de Água Gelada – CAG, com volume de ar constante, e este, tomando como base os modelos mais populares do mercado brasileiro, possui COP base próximo a classe “D” por demandar muita energia à sua operação, foi utilizado o valor de COP indicado à classe “D” pela tabela da INI-C. Posterior a etapa de determinação de classe, foi feita a substituição dos aparelhos pertencentes à classe “D” – 2,60, para a classe “A” – 3,23 como forma de otimização utilizando estratégia ativa de redução de consumo de energia.

\subsubsection{Eficiência energética do sistema iluminação artificial e equipamentos}
A eficiência do sistema de iluminação e equipamentos de um edifício de escritório, tal qual o sistema de condicionamento de ar, representa uma parcela importante no consumo final de energia elétrica da edificação \cite{AmericanSocietyofHeatingRefrigeratingandAir-ConditioningEngineers-ASHRAE2019,ConselhoBrasileirodeConstrucaoSustentavel-CBCS2015}.\vspace*{0.3cm} \newline
Dessa forma, para certificar que o sistema de iluminação artificial seja energeticamente eficiente e reduza o impacto desse sistema no consumo de energia elétrica, é avaliada a razão entre o somatório das potências das lâmpadas e reatores instalados e a área de um ambiente ou zona térmica, razão denominada como Densidade de Potência de Iluminação – DPI, expressa em W/m². O mesmo procedimento é aplicado aos equipamentos, definidos pela Densidade de Potência de Equipamentos – DPE, expressa em W/m². A união das duas densidades é definida pela Densidade de Carga Interna – DCI.\vspace*{0.3cm} \newline
Após a avaliação do DPI, os equipamentos de iluminação artificial são classificados segundo a classe de eficiência energética estabelecida pelo PBE/Inmetro, conforme a Tabela \ref{tab:tabela7}.

\begin{table}[ht]\centering
    \caption{\small Densidades de Potência de Iluminação definidas pelo INI-C e as Classes de Eficiência Energética.}
    \vspace*{0.3cm}
    \label{tab:tabela7}
    \begin{tabular}{cc}
        \hline
        \textbf{Classes}                & \textbf{Densidade de Potência de Iluminação (DPI – W/m²)}\\ \hline
        A                               & 8,50                                                     \\ \hline
        B                               & 10,40                                                    \\ \hline
        C                               & 12,20                                                    \\ \hline
        D                               & 14,10                                                    \\ \hline
    \end{tabular}
    \begin{flushleft}
        \par \small Fonte: adaptado de INMETRO (2018).
    \end{flushleft}
\end{table}\vspace*{-0.5cm}
\noindent Assim como o sistema de condicionamento de ar, a definição da DPI possibilita classificar o sistema de iluminação artificial do modelo genérico segundo sua eficiência energética. Para a determinação da classe de eficiência energética, o procedimento é o mesmo adotado para definição do COP para a etapa de simulação, e neste caso, como os equipamentos elétricos e de iluminação não foram levantados nominalmente, partiu-se da situação requerida e indicada pelo INI-C, com DPI classe “D”, de 14,10 W/m², e equipamentos com DPE de 9,7 W/m². Desta forma, espera-se que seja evidenciada a influência do sistema de iluminação no contexto geral de otimização da edificação.

\subsection{Definição dos modelos genéricos}
Com base no INI-C (2018) e no levantamento das edificações de escritório de Vitória, foram 
propostos dois tipos de modelos genéricos como base para o estudo das modificações de 
otimização e de produção de energia. Estes modelos representam os dois cenários de ambiente 
construído mais observados na cidade de Vitória. Estes cenários são formados por edificações 
mais baixas, com 8 pavimentos, e as mais altas, com 19 pavimentos. As dimensões utilizadas 
como referência para a construção dos modelos genéricos foram resultado dos valores médios 
observados nas edificações que compõe o levantamento.\vspace*{0.3cm} \newline
As características predominantes aplicadas aos modelos genéricos foram:
    \begin{itemize}
        \item Número de pavimentos;
        \item Forma - retangular;
        \item Altura - gabarito e dimensões das fachadas;
        \item Layout interno dos pavimento-tipo;
        \item Ausência de proteção solar;
        \item Percentual Total de Área de Abertura da Fachada.
    \end{itemize}
A composição dos modelos é baseada nas características predominantes e nos dados coletados 
\textit{in site}.
\subsubsection{Composição dos modelos genéricos}
A composição construtiva atribuída aos modelos utilizados neste trabalho mostra fundamentalmente 
os parâmetros necessários para a avaliação do desempenho energético segundo o INI-C. Os 
atributos utilizados serviram como ponto de partida para as análises subsequentes sugeridas nas 
etapas metodológicas e estão dispostos no Fluxograma da Figura \ref{fig:figura10}.
    \begin{figure}[ht]
        \centering
        \caption{\small Fatores utilizados como parâmetros de configuração volumétrica dos modelos genéricos.}
        \includegraphics[width=0.9\textwidth]{figures/fig10_Fluxogramas-2.jpg}
        \begin{flushleft}
            \par \small Fonte: autor (2019).
        \end{flushleft}
        \label{fig:figura10}
    \end{figure}
    
Apresentados na Tabela 7 e exemplificado na Figura \ref{fig:figura11}, os atributos estudados foram Fator de 
Forma, FF, Fator Altura, FA, Percentual de Área de Abertura da Fachada Total, PAFT, Ângulo 
Vertical de Sombreamento, AVS, e Ângulo Horizontal de Sombreamento, AHS.
    \begin{figure}[ht]
        \centering
        \caption{\small Estrutura arquitetônica dos modelos genéricos.}
        \includegraphics[width=0.9\textwidth]{figures/fig11_8-19-2pav.png}
        \begin{flushleft}
            \par \small Fonte: autor (2019).
        \end{flushleft}
        \label{fig:figura11}
    \end{figure}


\end{onehalfspace}