\section{Método}
Assim, neste capítulo são apresentadas as três principais etapas utilizadas na 
metodologia para esta pesquisa. Estas etapas podem ser descritas como:
\begin{easylist}
    \ListProperties(Numbers1=r,Numbers2=l,Hide2=1,Hang=true,Margin1=3ex,Margin2=6ex,FinalSpace=1em)
    @ Definição dos modelos genéricos. A etapa de definição dos modelos foi elaborada em 3 partes, 
    dentre as quais:
        @@ Coleta    de    dados    sobre    as    características    das    edificações    comerciais, 
        especificamente de escritório, em Vitória (ES);
        @@ Levantamento e definição das variáveis sobre os padrões de uso e ocupação das 
        salas  de  escritório,  assim  como  padrões  de  conforto  e  níveis  de  
        eficiência energética dos equipamentos de condicionamento de ar e iluminação;
        @@ Estabelecimento  dos  modelos  genéricos com  o  intuito de  evidenciar o  
        consumo total  final  de  energia  elétrica  por  meio  da  determinação  da  classe  
        de  eficiência energética  da  edificação,  proposta  pela  INI-C,  e  o  
        potencial  de  otimização  e produção de energia elétrica a partir de fontes renováveis.
    @ Simulações.  Nesta  etapa  são  avaliadas  as  características  mais  influentes  
    no  consumo energético da edificação de referência e o potencial de geração de energia 
    solar. Ambas as  avaliações  serão  feitas  por  meio  de  simulação  computacional.  
    As  simulações  foram fracionadas em 3 partes, dentre as quais:
        @@ Simulação dos modelos real e de referência, onde é feita a determinação da classe 
        de desempenho energético das edificações observadas em campo; 
        @@ Otimização    dos    modelos    genéricos,    representando    a    etapa    onde    
        são implementadas estratégias passivas e ativas visando a eficientização da edificação;

\end{easylist}