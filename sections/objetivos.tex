\subsection{Objetivos}
Diante do exposto, o objetivo principal desta pesquisa foi avaliar 
a aplicabilidade do conceito Zero Energy  em  edificações  comerciais,  
especificamente  de  escritório,  com  estudo  de  caso  para  o 
município  de  Vitória  (ES).  Este  setor  foi  selecionado  por  
apresentar  padrões  amplamente difundidos  de  uso  e  ocupação,  
de  equipamentos  e  da  conformação  do  espaço,  que  não  
só viabilizam  a  adoção  de  tecnologias  de  produção  de  energia  
elétrica,  como  também  facilita  a análise de desempenho termoenergético, 
quando comparado às edificações do setor industrial e do setor residencial.\\
Visando alcançar os resultados esperados, foram definidos os seguintes 
objetivos específicos:
\begin{itemize}
    \item Identificar os parâmetros aplicáveis às edificações de escritório 
    inerentes ao conceito Zero Energy e Near Zero Energy, assim como sua 
    viabilidade econômica;
    \item Mapear e diagnosticar as edificações comerciais concluídas a partir 
    de 2003, em Vitória – ES,  como  recorte  da  pesquisa,  com  a  
    caracterização  da  envoltória  e  dos  sistemas  de iluminação e 
    condicionamento de ar;
    \item Identificar métodos para a geração energética e formas de 
    racionalização do consumo de energia, estabelecendo diretrizes para 
    situações semelhantes.
\end{itemize}