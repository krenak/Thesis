\section{Referencial Teórico}
Este capítulo trata das referências utilizadas para o desenvolvimento da 
pesquisa. O referencial teórico foi organizado, em grande parte, com base 
nos estudos de Didoné (2014), Didoné, Wagner e  Pereira  (2014),  Kurnitski 
et al.  (2011)  e  Torcellini et al.  (2006).  Estas  referências  tratam  
das definições sobre Zero Energy Buildings, sobre conforto ambiental por 
meio de estratégias passivas e ativas, a eficiência energética voltada a 
edificações e a produção de energia elétrica por meio de tecnologias 
fotovoltaicas. Estes autores foram escolhidos por serem referências 
recorrentes em pesquisas posteriores as publicações citadas e apresentarem 
metodologias e embasamentos teóricos importantes para o desenvolvimento de 
pesquisas sobre o tema Zero Energy. Da mesma forma, foram utilizados 
conceitos abordados pela Instrução Normativa Inmetro para Classe de 
Eficiência Energética de Edificações Comerciais, de Serviços e Públicas 
– INI-C (2018) e pelas  normas  NBR  15.220  (2019),  e The American 
Society of Heating, Refrigerating and Air-Conditioning 
Engineers – ASHRAE Standard 55 (2017), 140 (2017) e 90.1 (2010). 
O contexto socioeconômico e climático de Vitória, assim como a caracterização 
da tipologia de referência  para  suporte  metodológico  das  simulações  
e  constituição  dos  modelos  genéricos foram abordados neste capítulo.

\subsection{Introdução ao conceito Zero Energy}
Define-se que um edifício Zero Energy – ZEB, ou em português, balanço 
energético nulo, é uma edificação  energeticamente  eficiente  onde,  
considerada  a  fonte  energética,  a  energia  elétrica fornecida pela 
concessionária é anualmente menor ou igual à quantidade de energia renovável 
exportada pela edificação para a rede 
\cite{Torcellini2006,U.S.DepartmentofEnergy-USDOE2012,U.S.DepartmentofEnergy-USDOE2015}. 
Domingos et al.  (2014)  define  que  o  balanço  energético  nulo  pressupõe  
uma  arquitetura adequada ao uso de elementos construtivos e equipamentos 
de alta eficiência energética, aliado ao desempenho da fonte geradora de 
energia elétrica a partir de fontes renováveis. A redução do consumo de 
energia em novas edificações ou em processo de melhoria pode ser alcançada 
por meio de projetos integrados à tecnologias de produção de energia, com 
adoção de soluções   energeticamente   eficientes,   e   por   programas   
de   economia   de   energia   (U.S. DEPARTMENT OF ENERGY, 2015). 
Torcellini et al. (2006) estabelecem quatro definições acerca das formas de 
se atingir o ZEB em edificações  de  baixo  consumo  de  energia,  ou  
comumente  denominadas Low-Energy Buildings. Dentre as formas estudadas estão: 
\begin{itemize}
    \item Zero Site Energy, ou energia local zero ou ainda energia da 
    edificação (U.S. DEPARTMENT OF ENERGY, 2015), onde é avaliada a 
    potencialidade de produção de energia elétrica para a  edificação  
    utilizando  os  recursos  presentes  no  local,  ou on-site,  onde  o  
    edifício  está implantado. É minimamente avaliado o consumo dos 
    sistemas de condicionamento de ar, de aquecimento quando existente, 
    ventilação, cargas de equipamentos e de sistema de iluminação;
    \item Zero Source Energy,  ou  fonte  de  energia  zero,  trata-se  do  
    conceito  onde  é  levado  em consideração toda a cadeia de produção 
    total anual de energia utilizada pela edificação e de consumo de 
    energia primária do edifício. Esta avaliação leva em conta a 
    eletricidade, combustíveis utilizados em processamento e transporte de 
    materiais e componentes para o local da edificação, entre outros aspectos;
    \item Zero Energy Cost, ou custo de energia zero, é avaliada a razão, 
    no mínimo igual, entre a quantidade total de dinheiro que é arrecadado 
    com a venda de energia produzida on-siteà  concessionaria,  e  a  
    quantidade  total  paga  pela  utilização  de  serviços  e  energia 
    consumida ao longo do ano; e 
    \item Zero Energy Emissions,  ou  emissão  zero,  onde  a  edificação  
    produz  uma  quantidade  de energia  renovável  livre  de  emissão  
    de  GEE  ao  menos  igual  a  quantidade  de  energia consumida 
    proveniente de fontes de energia emissoras de GEE.
\end{itemize}
