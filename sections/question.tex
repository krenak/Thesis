\subsection{Questionamentos}
\begin{onehalfspace}
        Considerando que:
    \begin{itemize}
        \item Existe uma parcela de energia elétrica proveniente de fontes fósseis no 
        Estado e que este quadro pode se agravar ao longo do tempo, visto a falta de 
        representatividade das fontes alternativas de geração de energia na matriz 
        energética do Espírito Santo;
        \item A demanda energética das edificações comerciais poderia  ser reduzida,  se 
        desde a fase projetual fosse considerada as potencialidades e restrições 
        ambientais do entorno;
        \item A micro e mini geração de energia elétrica é uma possibilidade que deve ser 
        incrementada no Brasil, principalmente considerando o potencial de queda de custos 
        na implementação de fontes de geração de energia elétrica descentralizada;
        \item Os   componentes   da   edificação,   como   envoltória   e   os   sistemas   
        de   conforto termoenergético, são subutilizados ou mal dimensionados no âmbito do 
        recorte territorial considerado, acarretando a baixa eficiência energética do 
        edifício.
    \end{itemize}
    A pergunta foi estabelecida a partir do seguinte questionamento: 
    considerando as características do  ambiente  construído  no  âmbito  da  Região  
    Metropolitana  da  Grande  Vitória,  é  possível desenvolver edificações cujos valores 
    de demanda e produção de energia elétrica resultem em nulo ou quase nulo?
\end{onehalfspace}