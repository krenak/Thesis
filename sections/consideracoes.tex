\section{Considerações Finais}
\noindent O conceito \textit{Zero Energy} vem ganhando repercussão no cenário brasileiro nos últimos anos, visto que este conceito vai ao encontro à frequente discussão sobre a necessidade de diversificação da matriz elétrica e do significativo avanço na geração de energia solar fotovoltaica no Brasil. Estas são realidades cada vez mais próximas do consumidor comum. Simbolizando a importância do tema acerca da economia de energia elétrica nacional, o Ministério de Minas e Energia, por meio de portaria publicada em dezembro de 2018, lança o Plano Anual de Aplicação de Recursos (PAR-2018) do PROCEL. Em dezembro de 2019 foi publicada a chamada publica, fruto do PAR-2018. Esta chamada pública descreve o projeto “Concurso NZEB – Edificações \textit{Near Zero Energy Building}”, onde é proposta a construção de quatro edificações \textit{Near Zero Energy} em território nacional, a fim de monitorar e avaliar a aplicabilidade do conceito.\vspace*{0.3cm} \newline
\noindent O barateamento progressivo do custo de instalação e manutenção dos sistemas de produção de energia solar e de outras fontes menos populares, porém direcionadas a escala da edificação, são pontos positivos e um atrativo para o engajamento a respeito do balanço energético nulo. Entretanto, contrariamente a esta evolução de cenário acerca da energia solar, políticas públicas buscam penalizar a progressão deste setor com tributações sobre a energia gerada e sobre a comercialização de painéis em território nacional \cite{Warth2019a,Warth2019}. A mini e micro geração também demonstram potencial de desenvolvimento no Brasil, e em especial, no Espirito Santo, contando com usinas de geração solar de pequeno porte no interior do estado, além de regulamentações e normas que favorecem à implantação e expansão deste segmento no país.\vspace*{0.3cm} \newline
\noindent O caminho para a popularização dos métodos de redução de consumo de energia e da preocupação com a sustentabilidade no ambiente construído passa pela disseminação de conhecimento e desenvolvimento de conceitos levados a público.\vspace*{0.3cm} \newline
\noindent Este trabalho aponta para uma alternativa ao modo construtivo convencional, indicando formas de tornar a edificação eficiente e produtiva, do ponto de vista energético, utilizando estratégias que resultem em um ambiente construído mais eficaz. Para tal, buscou-se construir um cenário que viabilizasse a observação de evidências do balanço energético nulo de um edifício.\vspace*{0.3cm} \newline
\noindent Assim, foram verificadas dificuldades em levantar as características das edificações para a construção da tipologia genérica, uma vez que as fontes de informação sobre consumo de energia ou dados técnicos sobre o processo construtivo das edificações estudadas era de difícil acesso.\vspace*{0.3cm} \newline
\noindent A escolha da cidade de Vitoria para o desenvolvimento da pesquisa teve como objetivo facilitar a obtenção de dados, o que não se confirmou na maioria dos casos do levantamento em campo. As edificações de escritório por si só apresentam dificuldades em obtenção de informações, uma vez que possuem muitos proprietários e abriga uma complexidade maior em termos de número de usuários e horários de ocupação. Contudo, a quantidade de estudos em relação ao tema, tanto nacional quanto internacionalmente, facilitou o preenchimento de lacunas de informação para a modelagem da tipologia genérica proposta.\vspace*{0.3cm} \newline
\noindent Em seguida, a proposição de modelos computacionais que reunissem os atributos levantados em campo resultou em testes de ferramentas para a avaliação de confiabilidade e complexidade. O \textit{EnergyPlus}, juntamente ao \textit{plug-in OpenStudio}, se destacaram por serem ferramentas gratuitas, \textit{open-source}, e de ampla utilização no meio acadêmico. Entretanto, esta escolha se mostrou limitada em aspectos que exigiram análises mais específicas de variáveis, ou de simulações que requeriam maior poder de processamento por apresentar uma quantidade de dados acima da capacidade de processamento da ferramenta de simulação, limitando o alcance dos resultados pretendidos para a pesquisa.\vspace*{0.3cm} \newline
\noindent Inicialmente, a função destes modelos foi atestar a baixa eficiência energética das edificações observadas Vitória, servindo como objeto para melhorias dos aspectos construtivos e energéticos. No entanto, as restrições de equipamento e de tempo para simulação foram cruciais para a determinação de resultados mais abrangentes, exigindo adaptações e simplificações de processo de simulação e análise de dados para a validação do método proposto. A simulação paramétrica necessária para a análise aleatória dos cenários, de forma plena, foi impossibilitada, em parte, devido aos custos financeiros exigidos pela ferramenta. Os recursos computacionais disponíveis seriam incapazes de processar a grande quantidade de variáveis simultaneamente, o que resulta normalmente na utilização do processamento de dados via nuvem. Este recurso é oferecido pela ferramenta, porém, por meio de um servidor pago estrangeiro, o que tornou financeiramente inviável a adoção deste tipo de simulação para a metodologia.\vspace*{0.3cm} \newline
\noindent Contudo, os resultados obtidos se mostraram satisfatórios para constatar a potencialidade de adoção do conceito \textit{Zero Energy} para as edificações propostas, assim como para os futuros edifícios do recorte territorial utilizado. O estudo econômico sobre a implantação dos sistemas propostos de produção de energia solar fotovoltaica e o resultado positivo reforça o alcance e a viabilidade de implementação da tecnologia.\vspace*{0.3cm} \newline
\noindent Porém, foi observado que edificações de menor porte, como o modelo de 8 pavimentos utilizado nesta pesquisa, podem enfrentar dificuldades em atingir o estado \textit{Zero Energy}, principalmente se estas construções estiverem em meio a outras com altura maior ou mais próximas, dificultando a produção de energia solar. Observou-se que a área de fachada e de piso são características importantes na produção de energia das tipologias estudadas, contrariando o senso comum para uma edificação com 19 pavimentos e toda a demanda energética que esta apresenta, o que impediria seu balanço energético nulo.\vspace*{0.3cm} \newline
\noindent Os parâmetros de construção dos modelos computacionais foram definidos de forma a tornar suas atribuições genéricas, o que resultou na potencialidade de autonomia energética dos edifícios de escritório de Vitória. Além das atribuições genéricas, os resultados alcançados nesta pesquisa representam o objetivo final de proporcionar ferramentas e dados para incentivar o estudo do tema e dar suporte a futuros trabalhos.\vspace*{0.3cm} \newline
\noindent O processo de elaboração desta pesquisa demonstra que a forma como o consumo de energia é abordada hoje é passível de reavaliação. Esta reavaliação é importante principalmente em aspectos práticos, como na utilização de equipamentos energeticamente eficientes, de investimento em tecnologias de maior eficiência energética aplicadas ao ambiente construído e a reeducação dos usuários destes espaços. Estas medidas são parte integrante de um processo maior de configuração e de adaptação do ambiente edificado à conscientização do uso racional de energia.\vspace*{0.3cm} \newline
\noindent Por fim, a continuidade deste trabalho é fundamental para refinar os resultados alcançados, estendendo a metodologia criada a outras tipologias como a residencial, às edificações públicas, ou relacionar o balanço energético nulo a um conjunto de edificações, estudando diferentes escalas de consumo e produção de energia, por exemplo. Estudos de implementação do conceito \textit{Zero Energy} direcionados a comunidade em situação de vulnerabilidade socioeconômica, aplicado a habitações de interesse social, também representam tópicos relevantes a um estudo aprofundado. A integração de novas ferramentas ao ato de projetar, principalmente as que consideram o planejamento do uso e geração de energia do edifício, é significativo como complemento ao processo construtivo praticado no Brasil.\pagebreak