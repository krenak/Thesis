\section{Introdução}
     A energia elétrica é um recurso essencial para o desenvolvimento econômico de um país, 
     para a qualidade  de  vida  da  população  e para  a  manutenção  do meio  ambiente
     por  meio  de  seu  uso eficiente \cite{Fonseca2016}. A importância do uso racional
     e eficiente deste recurso torna imprescindível a conservação e redução do seu 
     desperdício para a sustentabilidade do ambiente em que se vive.\\ Desde  a  crise  do  
     petróleo,  ocorrida nos  anos  de 1970,  a eficiência  energética tem  a  função  de 
     proporcionar  condições  para  suprir  à  demanda  futura  de  energia.  Esta  gestão  
     eficiente  do consumo  de  energia  é  essencial  para  reduzir  o  impacto  energético  
     de  setores  como  o  de edificações, o qual consome de 36 a 40\% da energia total final
     global. A necessidade de expansão dos   setores   econômicos   provoca   demanda   por
     energia   elétrica.   Esta   busca   resulta   em desperdícios oriundos da falta de 
     políticas públicas efetivas, de investimento em tecnologia e de fiscalização   sobre 
     o   consumo   deste   insumo \cite{InternationalEnergyAgency-IEA2019,InternationalEnergyAgency-IEA2019a,
     UnitedNationsEnvironmentProgramme-UNEP2019,UnitedNations2017}.\\
     Em contraponto à demanda e ineficiência energética, as edificações comerciais, em particular 
     as de  escritório,  podem  desempenhar  funções  estratégicas  como  minimizar  o  uso  
     energético  e produzir eletricidade, aproximando ou equalizando a zero a razão entre a 
     produção e o consumo de energia. Estas edificações são denominadas edificações com balanço 
     energético nulo, ou Zero Energy Buildings  –  ZEB \cite{Crawley2009,Torcellini2006,Kurnitski2011,Kurnitski2015,Torcellini2015}.\\
     Calcula-se  que  a  tendência  de  adoção  desta  forma  de  projetar edificações crescerá 
     até 2050, haja vista que a publicação de normas e regulamentações acerca do tema vêm 
     crescendo ao redor do mundo \cite{UnitedNationsEnvironmentProgramme-UNEP2019}. Com  a  introdução  de  uma  ZEB,  a  
     exploração  de  recursos  renováveis  complementares  como  a energia solar, e a utilização 
     de tecnologia solar fotovoltaica, surgem como opção para minimizar as    consequências    
     negativas    causadas    por    condições    climáticas,    de    infraestrutura    
     e socioeconômicas adversas \cite{Pikas2014,Pikas2017}.\\
     A quantidade de radiação solar recebida no Brasil, por exemplo, alcança a ordem de 1.013 MWh, 
     nível acima de países com grande capacidade de geração de energia solar. Este fato 
     torna viável a  adoção  deste  recurso  como  forma  de  reduzir  o  uso  de  fontes  
     de  energia  fósseis  e  como economia  no  consumo  de  água.  A  disponibilidade  de  energia
     solar  no  Brasil  alcança  cerca  de 6,5 kWh/m²  ao  ano  e,  no  Espírito  Santo,  
     entre  4,8  a  5,2 kWh/m²  ao  ano \cite{AgenciadeRegulacaodeServicosPublicosdoEspiritoSanto-ARSP2019,
     Didone2014,InternationalEnergyAgency-IEA2018}.\\ 
     A relação entre fontes da matriz energética brasileira é composta por 45\% de fontes renováveis
     e 55\% de fontes não-renováveis de energia. Há, ainda, a previsão de que a parcela de geração
     de eletricidade  por  meio  de  fontes  renováveis,  que  em  2018  era  de  83,3\%,  atinja  
     87\%  até  2040 \cite{EmpresadePesquisaEnergetica-EPE2017}. No entanto, há controvérsias na 
     classificação da fonte hidrelétrica como renovável, considerando a dependência da água, 
     dos ciclos de chuva e dos impactos gerados na construção das usinas \cite{Leme2012}.\\
     Dentro  do  contexto  de  segurança  energética,  a  crise  brasileira,  ocorrida  em  2001,  
     provocou mudanças no planejamento do fornecimento de energia elétrica, com o posterior 
     surgimento de medidas  atenuantes  às dificuldades  de  cunho  ambiental  e  de  infraestrutura  
     da  época.  Em  seu ápice, no ano de 1999, o país passou pelo período popularmente denominado “apagão”, 
     o qual representou  a  falta  de  fornecimento  em  70\%  do  território  nacional.  
     O  consumo  de  energia elétrica, entre os anos de 1990 e 2000, sofreu aumento de 49\%, 
     enquanto a capacidade instalada foi expandida em 35\%, ocasionando o descompasso entre 
     consumo e fornecimento nesta época \cite{Conejero2016,Tolmasquim2000}.\\ 
     Verifica-se  também  que  a  centralização  de  geração  de  energia  representa  fragilidade  
     para  o modelo  de  comercialização  utilizado  no  Brasil \cite{Pinto2017}. 
     Logo,  as mudanças observadas sobre a incorporação e aumento da participação de fontes 
     renováveis de energia  no mix  energético  brasileiro,  além  da  inserção  de  edificações  
     com  alto  desempenho energético como as ZEB’s, pode servir como uma das respostas necessárias 
     visando a segurança energética  e  elevando  a  confiabilidade  do  SIN  -  Sistema  Interligado
     Nacional \cite{EmpresadePesquisaEnergetica-EPE2017a}.\\ No âmbito estadual, o Espírito Santo 
     vem apresentando redução na produção de energia limpa nos últimos 8 anos, quando comparado 
     proporcionalmente ao consumo de fontes tradicionais. Existe ainda a parcela de geração de 
     energia elétrica oriunda de fontes não-renováveis de energia, como usinas termelétricas, 
     correspondendo a 65\% de toda a capacidade instalada em operação do  Espírito  Santo,  
     restando  35\%  de  fontes  renováveis,  composta  por  usinas  hidrelétricas,  com 
     participação  de  34\%,  e  geradores  de  energia  solar  fotovoltaica,  com  1\% 
     \cite{AgenciadeRegulacaodeServicosPublicosdoEspiritoSanto-ARSP2019,EnergiasdePortugal-EDP2017}.\\
     Sabe-se  que  as  edificações  comerciais no  Brasil  utilizam  majoritariamente  a  eletricidade,  
     em especial  as  edificações  de escritório,  com  aproximadamente  92\%  do  consumo  total,  enquanto edificações  de  
     uso  não-comercial  utilizam  fontes  de  energia  diversificadas.  Assim,  a  redução 
     superficial de consumo de energia destas edificações nos últimos 5 anos, quando comparado 
     com os  outros  setores  econômicos,  é  da  ordem  de  2,74\%,  o  que  reforça  a  
     importância  em proporcionar  o  aumento  da  eficiência  energética  para  o  segmento  
     de  edificações  comerciais \cite{AgenciadeRegulacaodeServicosPublicosdoEspiritoSanto-ARSP2018,
     AgenciadeRegulacaodeServicosPublicosdoEspiritoSanto-ARSP2019,
     EmpresadePesquisaEnergetica-EPE2018}.\\
     À vista  destes dados,  a presente pesquisa objetivou avaliar  o  potencial de adoção  do  conceito Zero  
     Energy   enquanto   uma  das  possíveis   estratégias  visando   a   redução   dos  
     problemas energéticos e ambientais relacionados às edificações de escritório, como forma 
     de contribuir aos novos mecanismos para planejar e projetar o ambiente construído. Da mesma 
     forma, busca-se evidenciar  medidas que propiciem  a  redução  do impacto  do  consumo  
     energético  vinculado  ao uso destes edifícios. 