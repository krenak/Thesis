\section{Introdução}
     A energia elétrica é um recurso essencial para o desenvolvimento econômico de um país, 
     para a qualidade  de  vida  da  população  e para  a  manutenção  do meio  ambiente
     por  meio  de  seu  uso eficiente (FONSECA et al., 2016). A importância do uso racional
     e eficiente deste recurso torna imprescindível a conservação e redução do seu 
     desperdício para a sustentabilidade do ambiente em que se vive. Desde  a  crise  do  
     petróleo,  ocorrida nos  anos  de 1970,  a eficiência  energética tem  a  função  de 
     proporcionar  condições  para  suprir  à  demanda  futura  de  energia.  Esta  gestão  
     eficiente  do consumo  de  energia  é  essencial  para  reduzir  o  impacto  energético  
     de  setores  como  o  de edificações, o qual consome de 36 a 40\% da energia total final
     global. A necessidade de expansão dos   setores   econômicos   provoca   demanda   por
     energia   elétrica.   Esta   busca   resulta   em desperdícios oriundos da falta de 
     políticas públicas efetivas, de investimento em tecnologia e de fiscalização   sobre 
     o   consumo   deste   insumo   (INTERNATIONAL...,   2019;   INTERNATIONAL...; UNITED..., 2019; UNITED ..., 2017, 2019). 
     Em contraponto à demanda e ineficiência energética, as edificações comerciais, em particular 
     as de  escritório,  podem  desempenhar  funções  estratégicas  como  minimizar  o  uso  
     energético  e produzir eletricidade, aproximando ou equalizando a zero a razão entre a 
     produção e o consumo de energia. Estas edificações são denominadas edificações com balanço 
     energético nulo, ou Zero Energy Buildings  –  ZEB  (CRAWLEY;  PLESS;  TORCELLINI,  2009;  KURNITSKI et al.,  2011,  2015; TORCELLINI et  al.,  2006).  
     Calcula-se  que  a  tendência  de  adoção  desta  forma  de  projetar edificações crescerá 
     até 2050, haja vista que a publicação de normas e regulamentações acerca do tema vêm 
     crescendo ao redor do mundo (UNITED..., 2019). Com  a  introdução  de  uma  ZEB,  a  
     exploração  de  recursos  renováveis  complementares  como  a energia solar, e a utilização 
     de tecnologia solar fotovoltaica, surgem como opção para minimizar as    consequências    
     negativas    causadas    por    condições    climáticas,    de    infraestrutura    
     e socioeconômicas adversas (PIKAS; THALFELDT; KURNITSKI, 2014; PIKAS et al., 2017). 
     A quantidade de radiação solar recebida no Brasil, por exemplo, alcança a ordem de 1.013 MWh, 
     nível acima de países com grande capacidade de geração de energia solar. Este fato 
     torna viável a  adoção  deste  recurso  como  forma  de  reduzir  o  uso  de  fontes  
     de  energia  fósseis  e  como