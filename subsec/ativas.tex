\subsubsection{Estratégias ativas}
As estratégias ativas são medidas que irão complementar a otimização inicial proporcionada 
por medidas passivas e, no caso deste trabalho, os esforços foram concentrados em aplicar 
medidas de aumento de eficiência sobre os sistemas de condicionamento de ar, de iluminação 
artificial e de equipamentos. Segundo a \textit{\textcite{InternationalEnergyAgency-IEA2014}}, as 
estratégias ativas importantes, considerando o contexto brasileiro, abrangem:
    \begin{itemize}
        \item O controle otimizado dos sistemas de condicionamento de ar, integrado com os 
        sistemas passivos de refrigeração do ar;
        \item Sistemas integrados de geração de energia por fonte renovável, auxiliados pela 
        iluminação natural, por sistemas fotovoltaicos e de cogeração de energia; e
        \item Sistemas automatizados de controle térmico, elevando a segurança e eficiência 
        energética;
    \end{itemize}
À semelhança das medidas de integração de sistemas propostas pelo IEA em relação às 
recomendações propostas pelo INI-C \cite{InstitutoNacionaldeMetrologiaNormalizacaoeQualidadeIndustrial-INMETRO2018}, devem ser observadas a relação de adaptação ao 
lugar, uma vez que ambas referências adotam conceitos e normas internacionais para a 
elaboração de seus estudos específicos.\vspace*{0.3cm} \newline
As edificações de escritório brasileiras utilizam sistemas de condicionamento de ar 
variados, e a disponibilidade e aplicação desses sistemas tem impacto direto no 
desempenho energético final do edifício. Em muitos casos, a escolha adequada do tipo 
mais eficiente, segundo sua tecnologia e funcionamento, é fundamental para garantir 
melhor desempenho e menor consumo de energia \cite{Kamal2019,Shin2019}. 
Dentre os sistemas de condicionamento de ar disponíveis em território nacional, 
pode-se citar \cite{ConselhoBrasileirodeConstrucaoSustentavel-CBCS2015}:
\begin{itemize}
    \item Sistema Central de Água Gelada – CAG/\textit{Fancoil}, sistema este que contempla dois 
    tipos de funcionamento: o sistema de condensação a água e a ar. No primeiro, o 
    sistema a água gelada é distribuída entre os pavimentos até os fancoils, e nestes a 
    água resfria o ar que é insulado no ambiente. No segundo sistema, o ar é resfriado 
    por fluido refrigerante. Há ainda a opção pela automatização e não-automatização da 
    distribuição de ar, denominados, respectivamente, Volume de Ar Variável – VAV e 
    Volume de Ar Constante – VAC. Este sistema foi observado como o mais recorrente entre 
    as edificações levantadas;
    \item Volume Refrigerante Variável, ou \textit{Variable Refrigerant Flow} – VRF, é o sistema 
    que resfria o ar por meio de fluido refrigerante, distribuindo unidades condensadoras 
    modularmente para cada pavimento da edificação. Esta forma de distribuição do sistema 
    VRF difere do sistema CAG/\textit{Fancoil}, que concentra esta central no pavimento técnico, 
    próximo ao reservatório de água. As condensadoras do VRF, assim, distribuem o ar 
    refrigerado para cada evaporadora, e esta, equipada com sensores e set point de 
    temperatura previamente configurados, proporciona o controle de temperatura para cada 
    zona térmica implantada. Como o sistema CAG/\textit{Fancoil}, utiliza sistema de condensação 
    a ar e a água;
    \item \textit{Self-contained} é um sistema que concentra todo o ciclo de refrigeração do ar 
    em uma máquina, comportando evaporadora, condensadora e compressor. É normalmente 
    instalada na casa de máquina dos pavimentos; e
    \item \textit{Split} é a classificação atribuída ao sistema de arrefecimento feito por \textit{splits}, 
    \textit{multiplits} e ar condicionado de janela – ACJ. Como a automação neste tipo de sistema é 
    rara, limitando-se a programação de acionamento e de desligamento do sistema, foi pouco 
    observado nos edifícios de escritório da amostra coletada, provavelmente em função do 
    ano em que os edifícios foram projetados, quando a relação custo/benefício dessa 
    tecnologia não era compensatória.
\end{itemize}
O sistema VRF oferece alta eficiência energética, custo-benefício e alto coeficiente de 
performance disponível em relação aos sistemas tradicionais disponíveis no mercado. É 
também utilizado como objeto de análises mais complexas, como deep learning sobre 
desempenho e vida útil do VRF, ou análises estatísticas para detecção de falhas no 
arrefecimento, demonstrando sua aplicabilidade e longevidade em relação às demais 
soluções apresentadas \cite{Cao2016,Guo2018,Liu2010,Teke2014,Wang2014}. 
Este sistema é indicado para simulações com área total condicionada maior que 4000 m², 
por possuir maior controle de temperatura por meio do ajuste de set point para cada zona 
térmica \cite{InstitutoNacionaldeMetrologiaNormalizacaoeQualidadeIndustrial-INMETRO2016}.\vspace*{0.3cm} \newline

\noindent O Guia Avançado de Planejamento Energético para Pequenos e Médios Edifícios Comerciais, 
ou \textit{Advanced Energy Design Guide for Small to Medium Office Buildings} – AEDG, disponibilizado 
pela ASHRAE, propõe recomendações para a redução do consumo de energia em edificações 
comerciais em 30\%, 50\% e 100\%, ou \textit{Zero Energy}. Estas reduções ocorrem por meio da 
adoção de estratégias passivas e ativas.\vspace*{0.3cm} \newline
Ao analisar estas recomendações, verifica-se a relação entre o AEDG e o INI-C acerca da 
proposição de medidas de redução de consumo de energia. Esta relação de semelhança é 
evidenciada quando analisado aspectos como adoção de materiais energeticamente mais 
eficientes, de equipamentos e sistemas de condicionamento de ar e iluminação artificial 
com alto desempenho energético \cite{AmericanSocietyofHeatingRefrigeratingandAir-ConditioningEngineers-ASHRAE2011,AmericanSocietyofHeatingRefrigeratingandAir-ConditioningEngineers-ASHRAE2014,AmericanSocietyofHeatingRefrigeratingandAir-ConditioningEngineers-ASHRAE2019}.\vspace*{0.3cm} \newline
Entre estas recomendações, destacam-se as medidas para o aumento da eficiência do 
sistema de iluminação e de equipamentos em 30\%, no mínimo, por meio de implementação de 
controles de luminosidade das lâmpadas, sensores fotoelétricos controlando o acionamento 
do sistema de iluminação, além do correto dimensionamento dos aparelhos de iluminação 
para os ambientes \cite{AmericanSocietyofHeatingRefrigeratingandAir-ConditioningEngineers-ASHRAE2019}.\vspace*{0.3cm} \newline
O INI-C propõe melhorias para os equipamentos e para o sistema de iluminação com o 
intuito de alcançar o nível de eficiência energética, indicado pela etiqueta “A”. 
Para isso, é definido que ambos os requisitos apresentem redução mínima de consumo em 30\%, 
assim como é proposto pelo AEDG \cite{InstitutoNacionaldeMetrologiaNormalizacaoeQualidadeIndustrial-INMETRO2016}. 
As modificações sugeridas são:
    \begin{itemize}
        \item A redução da Densidade de Potência de Iluminação Limite – DPIL, de 9,10 W/m², 
        correspondendo a uma redução de 31,20\% em relação ao DPIL de etiqueta “D”, 14,10 W/m²; e
        \item A sugestão de redução de consumo para equipamentos, onde a Densidade de 
        Potência de Equipamentos – DPE, é de 9,7 W/m², acompanhando o aumento de eficiência 
        da edificação junto à Densidade de Carga Interna – DCI, e da DPIL.
    \end{itemize}
