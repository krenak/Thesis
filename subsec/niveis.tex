\subsection{Níveis de eficiência energética}
Os níveis de eficiência dos aparelhos utilizados na manutenção do conforto da edificação 
exercem importante papel quando há a necessidade de racionamento do consumo energético. 
Para tal, estes aparelhos são avaliados segundo critérios de desempenho energético e, ao 
final dessa avaliação, é atribuído um índice representativo da eficiência energética 
alcançada. Os índices de desempenho apresentados neste trabalho foram baseados na 
Instrução Normativa Inmetro para Classe de Eficiência Energética de Edificações Comerciais, 
de Serviços e Públicas – INI-C.
\subsubsection{Determinação da classe de eficiência energética dos modelos genéricos}
As primeiras simulações energéticas servem como meio de ajuste e verificação de erros 
entre os dados de saída de consumo energético e intensidade de uso de energia. Os 
resultados das simulações iniciais foram comparados às análises feitas de Determinação 
de Classe de Desempenho Energético, baseado na metodologia do INI-C, e nos resultados 
apresentados pelo Relatório Final de Desempenho Energético do CBCS. A partir da 
discrepância dos valores obtidos entre as simulações iniciais e as referências 
selecionadas, foram realizados ajustes a fim de adequar o modelo computacional à 
tolerância estabelecida.\vspace*{0.3cm} \newline
Os modelos genéricos foram avaliados segundo a definição da Instrução Normativa, onde 
é definido que o Modelo Real representa a edificação a ser classificada, enquanto 
o Modelo de Referencia representa a edificação com baixo desempenho energético. 
Desta forma, os edifícios propostos são comparados com um modelo de baixa performance, 
evidenciando, assim, seu desempenho energético. As variáveis analisadas para a 
comparação com o Modelo de Referência foram o consumo de energia térmica, elétrica 
e primaria dos modelos.\vspace*{0.3cm} \newline
A partir do diagnóstico de desempenho energética dos modelos genéricos, foram 
implementadas as otimizações e medidas de produção de energia em contraponto ao 
consumo energético constatado em simulação.
\subsubsection{Eficiência energética do sistema de condicionamento de ar}
Os sistemas de condicionamento de ar presentes em edifícios de escritório são 
compostos basicamente por dois tipos de equipamentos: ar-condicionado Split e o 
Sistema Central de Água Gelada – CAG \cite{ConselhoBrasileirodeConstrucaoSustentavel-CBCS2015}.\vspace*{0.3cm} \newline
Segundo o \textcite{InstitutoNacionaldeMetrologiaNormalizacaoeQualidadeIndustrial-INMETRO2018}, esses equipamentos são classificados segundo a 
capacidade de resfriamento e a potência absorvida pelos motores em pleno 
funcionamento. Esta relação é representada pelo Coeficiente de Performance – COP, 
expresso em W/W. O COP para equipamentos de ar condicionado é categorizado 
segundo uma Classe de Eficiência Energética – CEE, como apresentado na Tabela \ref{tab:tabela6}.

\begin{table}[ht]\centering
    \caption{\small  Relação entre o COP e as Classes de Eficiência Energética de Condicionadores de ar.}
    \vspace*{0.3cm}
    \label{tab:tabela6}
    \begin{tabular}{cccc}
        \hline
        \textbf{Classes}     & \multicolumn{3}{c}{\textbf{Densidade de Potência de Iluminação (DPI – W/m²)}}        \\ \hline
        A                    & \makecell[c]{3,23}       & \makecell[c]{\textless CEE}                  &            \\ \hline
        B                    & \makecell[c]{3,02}       & \makecell[c]{\textless CEE =\textless} & 3,23       \\ \hline
        C                    & \makecell[c]{2,81}       & \makecell[c]{\textless CEE =\textless} & 3,02       \\ \hline
        D                    & \makecell[c]{2,60}       & \makecell[c]{\textless CEE =\textless} & 2,81       \\ \hline
    \end{tabular}
    \begin{flushleft}
        \par \small Fonte: adaptado de INMETRO (2018).
    \end{flushleft}
\end{table}\vspace*{-0.5cm}

\noindent Esta classificação atribuída ao sistema de ar condicionado dos modelos genéricos, juntamente 
ao nível de eficiência energética do sistema de iluminação, é essencial para a etapa de 
simulação e identificação do consumo final de energia dos modelos genéricos, representantes 
do cenário observado dentro do recorte territorial.\vspace*{0.3cm} \newline
Como o sistema mais frequente observado nas edificações levantadas foi o Sistema Central 
de Água Gelada – CAG, com volume de ar constante, e este, tomando como base os modelos mais 
populares do mercado brasileiro, possui COP base próximo a classe “D” por demandar muita 
energia à sua operação, foi utilizado o valor de COP indicado à classe “D” pela tabela da 
INI-C. Posterior a etapa de determinação de classe, foi feita a substituição dos aparelhos 
pertencentes à classe “D” – 2,60, para a classe “A” – 3,23 como forma de otimização 
utilizando estratégia ativa de redução de consumo de energia.
\subsubsection{Eficiência energética do sistema iluminação artificial e equipamentos}
A eficiência do sistema de iluminação e equipamentos de um edifício de escritório, tal 
qual o sistema de condicionamento de ar, representa uma parcela importante no consumo 
final de energia elétrica da edificação \cite{AmericanSocietyofHeatingRefrigeratingandAir-ConditioningEngineers-ASHRAE2019,ConselhoBrasileirodeConstrucaoSustentavel-CBCS2015}.\vspace*{0.3cm} \newline
Dessa forma, para certificar que o sistema de iluminação artificial seja energeticamente 
eficiente e reduza o impacto desse sistema no consumo de energia elétrica, é avaliada a 
razão entre o somatório das potências das lâmpadas e reatores instalados e a área de um 
ambiente ou zona térmica, razão denominada como Densidade de Potência de Iluminação – DPI, 
expressa em W/m². O mesmo procedimento é aplicado aos equipamentos, definidos pela Densidade 
de Potência de Equipamentos – DPE, expressa em W/m². A união das duas densidades é definida 
pela Densidade de Carga Interna – DCI.\vspace*{0.3cm} \newline
Após a avaliação do DPI, os equipamentos de iluminação artificial são classificados 
segundo a classe de eficiência energética estabelecida pelo PBE/Inmetro, conforme a Tabela \ref{tab:tabela7}.

\begin{table}[ht]\centering
    \caption{\small Densidades de Potência de Iluminação definidas pelo INI-C e as Classes de Eficiência Energética.}
    \vspace*{0.3cm}
    \label{tab:tabela7}
    \begin{tabular}{cc}
        \hline
        \textbf{Classes}                & \textbf{Densidade de Potência de Iluminação (DPI – W/m²)}\\ \hline
        A                               & 8,50                                                     \\ \hline
        B                               & 10,40                                                    \\ \hline
        C                               & 12,20                                                    \\ \hline
        D                               & 14,10                                                    \\ \hline
    \end{tabular}
    \begin{flushleft}
        \par \small Fonte: adaptado de INMETRO (2018).
    \end{flushleft}
\end{table}\vspace*{-0.5cm}
\noindent Assim como o sistema de condicionamento de ar, a definição da DPI possibilita classificar o 
sistema de iluminação artificial do modelo genérico segundo sua eficiência energética. 
Para a determinação da classe de eficiência energética, o procedimento é o mesmo adotado 
para definição do COP para a etapa de simulação, e neste caso, como os equipamentos 
elétricos e de iluminação não foram levantados nominalmente, partiu-se da situação requerida 
e indicada pelo INI-C, com DPI classe “D”, de 14,10 W/m², e equipamentos com DPE de 9,7 W/m². 
Desta forma, espera-se que seja evidenciada a influência do sistema de iluminação no 
contexto geral de otimização da edificação.

