\subsubsection{Parâmetros de conforto térmico}
Segundo a norma americana ASHRAE Fundamentals (2017), a sensação de conforto térmico 
é a situação onde a temperatura do corpo é mantida dentro de um limite confortável, 
a umidade da pele é baixa e o esforço fisiológico para manter estas condições é minimizado. 
Adicionalmente, a faixa de temperatura de conforto em ambientes como um escritório 
transita entre temperaturas acima de 21°C e abaixo de 24°C. Com base nessas definições 
e para avaliar o estado de conforto térmico dos ocupantes de um edifício de escritórios, 
aplicam-se modelos preditivos para avaliação desta condição do ambiente.\vspace*{0.3cm} \newline
Em se tratando de edificações condicionadas artificialmente, um dos modelos utilizados 
para esta avaliação é o Voto Médio Predito e Porcentagem Predita de Insatisfeitos, 
ou \textit{Predicted Mean Vote} – PMV, juntamente ao \textit{Predicted Percentage of Dissatisfied} – PPD.\vspace*{0.3cm} \newline
Publicado em 1970 por Povl Ole Fanger, este modelo é utilizado por normas como 
a ISO 7730, para avaliar o conforto térmico de um ambiente por meio de um questionário 
aplicado aos ocupantes da edificação avaliada \cite{AmericanSocietyofHeatingRefrigeratingandAir-ConditioningEngineers-ASHRAE2017}. Entretanto, este 
modelo apresenta discrepâncias quando aplicado à realidade brasileira, além de ser um 
modelo inviável para cenários que utilizem ventilação natural \cite{Rupp2016}. 
Estes fatores tornaram sua utilização inapropriada para o presente estudo.
Outra forma de avaliação do conforto térmico no local de trabalho, como o ambiente 
condicionado de um escritório, é o modelo de conforto adaptativo. Este modelo avalia a 
adaptação do usuário em diferentes campos como o fisiológico, o comportamental e o 
psicológico \cite{AmericanSocietyofHeatingRefrigeratingandAir-ConditioningEngineers-ASHRAE2017a} e pode ser expresso pela Equação 1.
\begin{equation}\label{eq:eq1}
            t_{c}=24,2+0,42\times(t_{ext}-22)exp-\left(\frac{T_{ext}-22}{24\sqrt{2}}\right)^2
\end{equation}

Onde:\par
\setlength\parindent{1.5cm} t\textsubscript{c} é a temperatura de conforto; e\par
\setlength\parindent{1.5cm} t\textsubscript{ext} é a temperatura externa ou temperatura de bulbo seco.\par
\noindent A temperatura de conforto auxilia a definir o set point de acionamento do sistema de 
condicionamento de ar temperatura para o ambiente analisado. Dada as condições de 
adaptação proporcionadas pelo ambiente avaliado e os ajustes naturais feitos pelos 
ocupantes deste espaço, a temperatura do ar aceitável em ambientes de escritório, por 
exemplo, pode variar entre 17°C a 31°C \cite{AmericanSocietyofHeatingRefrigeratingandAir-ConditioningEngineers-ASHRAE2017a}.\vspace*{0.3cm} \newline
Associado ao modelo de conforto adaptativo, a ventilação híbrida é uma alternativa de 
controle de temperatura e redução de consumo de energia ao combinar a ventilação natural 
com o sistema de condicionamento de ar da edificação. Esta forma de controle térmico 
representa uma opção para avaliar o consumo de energia \cite{AmericanSocietyofHeatingRefrigeratingandAir-ConditioningEngineers-ASHRAE2017}.\vspace*{0.3cm} \newline
Contudo, a ventilação híbrida não foi observada como solução corrente entre as 
edificações de Vitória e, de acordo com \textcite{Shaviv2001,Zhang2014,Navarro2016,SCHULZE2018,Sudhakar2019}, 
dificuldades foram identificadas quanto a implementação deste sistema, dentre elas:
\begin{itemize}
    \item A necessidade de uma ferramenta computacional a parte que simule a influência 
    da ventilação natural sobre a edificação;
    \item A intermitência da ventilação, fundamental para o funcionamento do sistema; e
    \item A configuração, em caráter individual, de materiais e componentes que reduzam 
    a ação da umidade dentro do ambiente construído onde a ventilação natural atuará.
\end{itemize}
Assim, conclui-se que ambos os modelos de conforto ambiental PMV/PPD e conforto adaptativo 
apresentados são válidos para mensurar o conforto de um ambiente, diferindo entre ambientes 
controlado e real. Desta forma, foi utilizado o modelo de conforto adaptativo como forma 
de avaliar a efetividade do condicionamento de ar, por se adequar a proposta da pesquisa ao 
possibilitar ajustes de conforto térmico voltados ao comportamento do usuário.