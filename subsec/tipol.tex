\subsection{Caracterização da tipologia de referência}
\textcite{Didone2014} utiliza um banco de dados de 1103 edificações brasileiras para determinar as propriedades construtivas e uso de energia elétrica, e desta amostra, 35 edificações foram selecionadas para análise das características energéticas e construtivas delimitadas pela autora. As principais propriedades construtivas enunciadas pela autora foram paredes externas feitas por blocos cerâmicos com argamassa em ambas as faces, cobertura composta por laje de concreto e, em alguns casos, telhas cerâmicas, e o piso de concreto.\vspace*{0.3cm}\newline
\textcite{Costa2017} destacam as principais características dos edifícios de escritório presentes em Brasília, onde estes, em sua maioria, apresentam características como forma retangular, orientação solar da fachada principal para Leste, 1 a 28 pavimentos-tipo, PAF\textsubscript{T} entre 50\% a 75\% e mais de 70\% das edificações levantadas não apresentavam proteção solar. Os vidros empregados não possuíam performance adequada para redução da influência térmica e luminosa da radiação solar, sendo o vidro incolor e o vidro fumê os mais empregados. No entanto, é importante ressaltar a atipicidade de Brasília, visto ser uma cidade planejada cujo Plano Piloto condiciona a forma – retangular – e orientação das edificações.\vspace*{0.3cm}\newline
\textcite{Sorgato2018} definem em seu artigo uma edificação comercial de escritório genérica de pequeno porte como uma edificação com 600 m² de área de piso, 4 pavimentos-tipo, e dimensões de 30 metros de comprimento por 12 metros profundidade e 14 metros de altura. Esta edificação foi submetida às condições climáticas de 6 cidades brasileiras – Rio de Janeiro, São Paulo, Florianópolis, Curitiba, Brasília e Belém – a fim de avaliar a produção de energia solar fotovoltaica e validade do balanço energético nulo em diferentes regiões do país. O estudo apontou que todas as edificações atingiram o balanço energético nulo.\vspace*{0.3cm}\newline
O \textcite{ConselhoBrasileirodeConstrucaoSustentavel-CBCS2015}, por meio do Relatório Final de Desempenho Energético, descreve que edificações comerciais podem ser segmentadas em três tipos. O primeiro abrange todos os tipos de uso aplicados ao comércio sobre o espaço utilizado, tais como hotéis, shoppings, centros comerciais e edifícios públicos. A segunda tipologia é a edificação comercial corporativa, caracterizada por edifícios com mais de 10 pavimentos, com salas que normalmente ocupam completamente o pavimento, limitando-se a 1 ou 2 ambientes. Estas edificações são necessariamente ocupadas apenas por uma ou duas empresas ou entidades públicas. Por fim, a terceira tipologia, a edificação comercial de escritórios, é configurada com cerca de 700 m² de área de piso, com altura que varia entre 60 a 120 metros e com 10 a 20 pavimentos-tipo. Estas edificações de escritório contêm salas com usos variados e proprietários autônomos.\vspace*{0.3cm}\newline
A maior frequência de ocorrência de número de pavimento observada por \textcite{Bernabe2012} foi de 15 pavimentos. A conclusão foi baseada no Plano Diretor Urbano – PDU, em vigor à época da pesquisa, aplicado nas regiões onde os edifícios de escritório foram selecionados. Da mesma forma, foi constatado que nas zonas onde as edificações foram construídas o gabarito praticado é livre, mesma característica observada no levantamento. Mantendo a exigência de gabarito livre para as zonas urbanas utilizadas nesta pesquisa, o último PDU de Vitória, publicado em 22 de maio de 2018, após a revisão realizada entre os anos de 2015 e 2017, restringe o gabarito em áreas próximas ao cone de aterrisagem do aeroporto, zona onde se situa apenas uma edificação da amostra desta pesquisa \cite{PrefeituraMunicipaldeVitoria-PMV2018}.\vspace*{0.3cm}\newline
Alguns parâmetros arquitetônicos destacados por \textcite{Lamberts2006}, \textcite{Bernabe2012}, e \textcite{Fonseca2016} como importantes para a avaliação do consumo de energia das edificações de escritório foram organizados na Tabela \ref{tab:tabela3}.\newline
    \begin{table}[ht]
        \caption{\small Parâmetros arquitetônicos a serem analisados}
        \label{tab:tabela3}
        \begin{tabular}{lp{8cm}l}
        \hline
        \textbf{Parâmetro}                      & \textbf{Descrição}                                                                                                                                                                                                                              \\ \hline
        Fator Forma (FA)                        & Razão entre a área da envoltória e o volume total da edificação                                                                                                                                                                                 \\ \hline
        Forma do edifício                       & Retangular, quadrada, circular, irregular (BERNABÉ, 2012; FONSECA et al., 2016; LAMBERTS; GHISI; RAMOS, 2006).                                                                                                                                  \\ \hline
        Fator Altura (FA)                       & Razão entre a área da projeção da cobertura e a área construída (INMETRO, 2018).                                                                                                                                                                \\ \hline
        Número da área do pavimento             & (BERNABÉ, 2012; FONSECA et al., 2016; INMETRO, 2018; LAMBERTS; GHISI; RAMOS, 2006).                                                                                                                                                             \\ \hline
        Percentual de Abertura da Fachada (PAF) & Razão entre as áreas de abertura envidraçada, ou com fechamento transparente ou translúcido, de cada fachada e a área total da fachada da edificação (FONSECA et al., 2016; INMETRO, 2018; LAMBERTS; GHISI; RAMOS, 2006; WERNECK et al., 2017). \\ \hline
        Fator solar do vidro (FS)               & Razão entre o ganho de calor que entra em um ambiente através de uma abertura e a radiação solar incidente nesta mesma abertura (INMETRO, 2018).                                                                                                \\ \hline
        Proteção solar                          & Ângulos de sombreamento externo das aberturas envidraçadas / orientação solar.                                                                                                                                                                  \\ \hline
        \end{tabular}
        \begin{flushleft}
            \par \small Fonte: autor (2019).
        \end{flushleft}
    \end{table}\newline
\noindent As referências selecionadas contribuem para a compreensão de definições utilizadas 
sobre o tema do trabalho, além de dados sobre as condições energéticas e ambientais 
necessárias para a implementação do conceito \textit{Zero Energy} aplicado à edificações.
