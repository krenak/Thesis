\subsubsection{Estratégias passivas}
As estratégias passivas são definidas pela adoção de medidas que não utilizem energia 
elétrica para controlar as condições de conforto e redução do consumo de energia de um 
ambiente construído. Estas medidas, normalmente aplicadas à envoltória, tem como função 
tornar o edifício o mais confortável e eficiente possível, reduzindo a necessidade de 
arrefecimento e iluminação artificial suplementares \cite{AmericanSocietyofHeatingRefrigeratingandAir-ConditioningEngineers-ASHRAE2014,Athienitis2015,Hensen2012,U.S.DepartmentofEnergy-USDOE2011}.\vspace*{0.3cm} \newline
Nesse sentido, destacam-se os elementos externos de sombreamento, o isolamento térmico 
da cobertura, a adoção de vidros com baixo fator solar e o uso de materiais termicamente 
reflexivos para a envoltória. Além disso, considerando o contexto climático de locais 
com climas quentes e úmidos, é recomendado um Percentual Total de Abertura na Fachada – PAFt, 
entre 30\% e 50\% \cite{Didone2014a}.\vspace*{0.3cm} \newline
Segundo a \textcite{AssociacaoBrasileiradeNormasTecnicas-ABNT2003} por meio da Norma Brasileira nº 15220,
medidas que isolem a cobertura termicamente são necessárias para o aumento da eficiência 
energética e do desempenho térmico da edificação, principalmente para as concebidas 
em apenas um pavimento. Estas medidas são comumente adotadas como estratégias passivas 
para climas quentes, aliadas à ventilação natural e o resfriamento das massas térmicas 
expostas diretamente à radiação solar. Complementarmente, busca-se o controle da 
iluminação natural do ambiente utilizando elementos de proteção solar, com o intuito de 
explorar a iluminação natural e a proteção contra a radiação solar.\vspace*{0.3cm} \newline
\textcite{Didone2014} aborda o balanço energético nulo em edifícios de escritório do Brasil e 
Alemanha, parametrizando as variáveis de altura, largura e comprimento. A autora 
estipulou a produção de energia por meio de módulos fotovoltaicos semitransparentes 
localizados nas janelas, proteções solares e na cobertura, concluindo que as células 
fotovoltaicas utilizadas são válidas para obter a condição de \textit{Zero Energy} nas 
edificações avaliadas, apesar da pouca produção dos elementos inseridos na janela e 
nas proteções solares. Os resultados obtidos por \textcite{Didone2014a} mostraram que 
a otimização aplicada aos edifícios de escritórios das cidades de Florianópolis e 
Fortaleza foi eficaz, utilizando módulos fotovoltaicos na cobertura e na fachada, 
tornando as edificações energeticamente balanceadas.\vspace*{0.3cm} \newline
Segundo \textcite{Noguchi2016}, para obter sucesso na implementação de estratégias passivas, 
é essencial aliar o conforto do ambiente ao clima da região analisada. Usufruir 
de soluções para o microclima da edificação e das características climáticas locais 
é fundamental para elevar a qualidade de vida dos habitantes e do ambiente construído. 
Dentre as estratégias passivas disponíveis, vale destacar as mais importantes para 
climas tropicais, tais como:
\begin{itemize}
    \item Escolha de orientação solar apropriada;
    \item Estudo da volumetria da edificação;
    \item Adaptação do Percentual Total de Abertura da Fachada – PAFT;
    \item Escolha de materiais e componentes adequados, como vidros mais eficientes;
    \item Valer-se da iluminação natural como fonte luminosa complementar;
    \item Estudo e aplicação de proteção solar como meio de condicionamento térmico e lumínico do espaço; e
    \item Cobertura com isolamento térmico;    
\end{itemize}
Como a escala proposta para este trabalho está restrita à edificação de escritório, 
estratégias passivas que fogem desta escala não foram abordadas. Pode-se mencionar como 
estratégias que fogem desta escala as soluções de paisagismo, vegetação e corpos d’água, 
tratados como microclima, e a disposição da edificação em relação ao entorno. Da mesma 
forma, estratégias que são direcionadas a edificações não-comerciais não foram consideradas.