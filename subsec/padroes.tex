\subsection{Padrões de conforto}
O conforto térmico e de iluminação natural são parâmetros essenciais para a avaliação da qualidade do ambiente em que se vive. Não obstante, é necessário estabelecer padrões de conforto para a simulação de desempenho termoenergético do modelo genérico. Os próximos subitens tratam dos parâmetros adotados para a posterior inserção de dados e informações nos processos de simulação.
\subsubsection{Conforto térmico}
Utilizando os conceitos de conforto adaptativo, foi calculada a temperatura de conforto (t\textsubscript{c}) descrita pela \textcite{AmericanSocietyofHeatingRefrigeratingandAir-ConditioningEngineers-ASHRAE2017}, e assim estipulada a faixa de temperaturas de conforto e set point para controle de temperatura dos ambientes dos modelos genéricos. Este modelo avalia a adaptação do usuário em diferentes campos como o fisiológico, o comportamental e o psicológico \cite{AmericanSocietyofHeatingRefrigeratingandAir-ConditioningEngineers-ASHRAE2017a} e pode ser expresso pela Equação \ref{eq:eq1} tratada no Referencial Teórico.\vspace*{0.3cm} \newline
Foram utilizadas as temperaturas de bulbo seco máximas e mínimas como valores representantes da temperatura externa da edificação \cite{InstitutoNacionaldeMetereologia-INMET2018}. Estes dados de temperatura serão fundamentais para a caracterização do meio em que o modelo genérico será simulado, influenciando no desempenho simulado da envoltória e dos sistemas de condicionamento de ar e iluminação.
\subsubsection{Conforto lumínico}
A norma NBR/ISO CIE 8995-1 \cite{AssociacaoBrasileiradeNormasTecnicas-ABNT2013} determina que a condição de conforto visual em ambientes de escritório requer valor igual ou maior que 500 lux de iluminância de entorno imediato da tarefa a ser desempenhada \cite{AssociacaoBrasileiradeNormasTecnicas-ABNT2013,Ramos2013}. Assegurar a iluminância é uma condição importante para garantir o conforto, desempenho e segurança visual aos usuários, seja por fonte luminosa artificial ou natural.\vspace*{0.3cm} \newline
Para que a iluminação interna se mantenha dentro do padrão estabelecido por norma, foi necessário calcular o Fator de Luz Diurna – FDL, como forma de verificar se a adoção iluminância proposta pela norma seria adequada as condições empregadas aos modelos de referência. Este fator, uma vez incorporado nas rotinas de simulação computacional, influencia diretamente nas condições de iluminação no ambiente, indicando se estão adequadas para a realização de tarefas que exigem um determinado volume de iluminação.\vspace*{0.3cm} \newline
Considerando a ocupação anual parcial dos ambientes, foi possível estabelecer a relação entre a iluminância interna desejada, 500 lux, e a mediana da iluminância difusa horizontal externa, dado retirado do arquivo climático de Vitória \cite{InstitutoNacionaldeMetrologiaNormalizacaoeQualidadeIndustrial-INMETRO2018}, de 50019,16 lux. Assim, o FLD necessário para manter a iluminância interna das salas de escritório durante o período de ocupação levantado pode ser expresso pela Equação \ref{eq:eq2}.
\begin{equation}\label{eq:eq2}
    FLD=\frac{E_{interna}}{E_{externa}}*100
\end{equation}

Onde:\par
\setlength\parindent{1.5cm} \textit{FLD} é o Fluxo de Luz Diurna, em porcentagem;\par
\setlength\parindent{1.5cm} E\textsubscript{interna} é a iluminância interior em um ponto de um plano, em lux; e\par
\setlength\parindent{1.5cm} E\textsubscript{externa} é a iluminância externa simultânea em um plano horizontal, em lux.\vspace*{0.3cm} \newline
\noindent Concluída a caracterização dos padrões de conforto a serem avaliados, são definidos os parâmetros para os níveis de eficiência energética esperados dos equipamentos de condicionamento de ar e iluminação artificial presentes no modelo genérico. A partir desta definição embasada em regulamentos como o INI-C, é possível estimar a otimização de consumo energético da edificação proposta.
