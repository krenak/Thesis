\subsection{Levantamento das características dos edifícios de escritório de Vitória}
As edificações de escritório de Vitória selecionadas após a definição do recorte 
territorial, apresentam características que foram complementadas aos atributos 
observados em edificações comerciais brasileiras, como suporte as informações não 
encontradas in loco. As características com maior frequência de ocorrência no 
levantamento realizado são apresentadas na Tabela 3. Todavia, a amostra coletada 
abrange edifícios iniciados em 2003 e concluídos até o fim do primeiro trimestre 
de 2018, data do início do levantamento. Este fato inviabiliza aplicar a última 
revisão do Plano à amostra.\vspace*{0.3cm} \newline
Foram considerados para o levantamento atributos como gabarito, número de 
pavimentos-tipo, número de salas por pavimento-tipo, dimensão e forma, altura 
dos pavimentos-tipo. Não foram consideradas as dimensões dos lotes onde as 
edificações da amostra estacam implantadas, já que este atributo não foi 
pertinente ao objetivo do trabalho. Estes parâmetros foram reunidos em consulta 
ao material técnico disponibilizado pelas construtoras, visitas a campo e 
complementação de dados utilizando a ferramenta computacional \textit{Google Street View}.\vspace*{0.3cm} \newline
Os trabalhos de \textcite{Lamberts2006,AmericanSocietyofHeatingRefrigeratingandAir-ConditioningEngineers-ASHRAE2010,Bernabe2012,Ramos2013,Didone2014,Didone2014a,ConselhoBrasileirodeConstrucaoSustentavel-CBCS2015,Fonseca2016,Werneck2017,InstitutoNacionaldeMetrologiaNormalizacaoeQualidadeIndustrial-INMETRO2018},
foram utilizados como principais fontes de informação para a análise de envoltória 
e sistema de iluminação e condicionamento de ar.\vspace*{-0.3cm}
\begin{table}[ht]\centering
    \caption{\small Características observadas em campo e em pesquisas anteriores.}
    \vspace*{0.2cm}
    \label{tab:tabela1}
    \begin{tabular*}{\columnwidth}{@{\extracolsep{\fill}}lll}
    \hline
    \textbf{Parâmetro}                                             & \textbf{Descrição}                                                                    & \textbf{Referências} \\ \hline
    Gabarito                                                       & 24 a 60 m (8 a 19 pav.)                                                               & \makecell[l]{Levantamento \textit{in loco} e referências\\ (BERNABÉ, 2012; CBCS, 2015; \\FONSECA et al., 2016; LAMBERTS; \\GHISI; RAMOS, 2006; RAMOS \\et al., 2013).} \\ \hline
    Altura do pavimento                                            & 3 m                                                                                   & Levantamento \textit{in loco}.                                                                                                                                         \\ \hline
    Planta-baixa (forma)                                           & Retangular                                                                            & \makecell[l]{Levantamento \textit{in loco} e referências\\ (FONSECA et al., 2016; INMETRO,\\ 2018).}                                                                   \\ \hline
    \makecell[l]{Dimensão das salas\\ por pav.-tipo}               & 40 m²                                                                                 & \makecell[l]{Foi fixado a área das salas \\(zonas térmicas) de acordo com a \\média de ofertas de salas observadas\\ em levantamento \textit{in loco}.}                  \\ \hline
    Proteção solar                                                 & Sem proteção                                                                          & \makecell[l]{Levantamento \textit{in loco} e referências\\ (FONSECA et al., 2016; WERNECK \\et al., 2017).}                                                            \\ \hline
    \makecell[l]{Componentes da\\ parede}                          & \makecell[l]{Bloco cerâmico,\\ 8 furos; \\14x19x29 cm; \\argamassa de\\ assentamento} & Levantamento \textit{in loco}.                                                                                                                                         \\ \hline
    Cobertura                                                      & \makecell[l]{Laje impermeabilizada\\ com 20 cm de\\ espessura}                        & \makecell[l]{Levantamento \textit{in loco} e referências \\(CB3E; ABIVIDRO, 2015).}                                                                                    \\ \hline
    \multicolumn{3}{c}{Continua}\\\hline
    \end{tabular*}
\end{table}\pagebreak
\begin{table}[ht]\centering
    \begin{tabular*}{\columnwidth}{@{\extracolsep{\fill}}lll}
    \hline
    \multicolumn{3}{c}{Conclusão}\\\hline
    Vidros                                                         & \makecell[l]{Laminado; Reflexivo;\\ 8 mm; Verde}                                      & \makecell[l]{(FONSECA et al., 2016; \\INMETRO, 2018).}                                                                                                                   \\ \hline
    PAF\textsubscript{T}                                           & 30\%; 50\%; 80\%                                                                      & \makecell[l]{Levantamento \textit{in loco}\\ e referências.}                                                                                                             \\ \hline
    \makecell[l]{Orientação solar da \\fachada principal}          & Sul                                                                                   & \makecell[l]{Levantamento \textit{in loco}\\ e referências.}                                                                                                             \\ \hline
    \makecell[l]{Densidade de Carga de\\ Iluminação Limite – DCIL} & 14,1 W/m²                                                                             & \makecell[l]{Consulta pública do RTQ-C \\(INMETRO, 2018).}                                                                                                               \\ \hline
    \makecell[l]{Densidade de Carga de\\ Equipamentos – DCE}       & 9,7 W/m²                                                                              & \makecell[l]{Consulta pública do RTQ-C \\(INMETRO, 2018).}                                                                                                               \\ \hline
    \makecell[l]{Absortância/transmitância \\das paredes}          & \makecell[l]{0,59 (cor \\camurça)/3,75}                                               & \makecell[l]{Valores consultados na \\NBR 15220-2 e referências \\(ABNT, 2003; FONSECA et \\al., 2016; INMETRO, 2018).}                                                                     \\ \hline
    \makecell[l]{Absortância/transmitância \\das coberturas}       & \makecell[l]{0,65 (concreto \\aparente)/2,06}                                         & \makecell[l]{Valores consultados na \\NBR 15220-2 e referências \\(ABNT, 2003; FONSECA et \\al., 2016; INMETRO, 2018).}                                                 \\ \hline
    \end{tabular*}
    \begin{flushleft}
        \par \small Fonte: autor (2019).
    \end{flushleft}
\end{table}\vspace*{-0.3cm}

\noindent As informações coletadas nos estudos e em levantamento formam a base conceitual para 
determinar os aspectos arquitetônicos relevantes para compor os modelos genéricos e, 
posteriormente, determinar os parâmetros de otimização e do consumo energético padrão 
aproximado para um edifício de escritório. Além disso, a quantidade de simulações necessárias 
para determinar o consumo de energia dos modelos genéricos é identificada por meio da 
organização dos dados coletados em campo e, assim, a resultante do número de variáveis.