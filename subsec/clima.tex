\subsection{Caracterização climática e socioeconômica da cidade de Vitória}
O Espírito Santo integra a região de alto desenvolvimento socioeconômico do Brasil, 
estando posicionado em 7º lugar, com índice de 0,740, e sua capital, Vitória, está 
entre os municípios com índice classificado como “muito alto” de desenvolvimento 
humano, com 0,845 \cite{UnitedNationsEnvironmentProgramme-UNEP2019,InstitutoBrasileirodeGeografiaeEstatistica-IBGE2019}. Apresenta, entretanto, 
indicadores discretos quando comparados aos outros estados pertencentes à Região Sudeste, 
por exemplo \cite{Cacador2013}.\vspace*{0.3cm} \newline
Apesar da condição discreta de desenvolvimento do estado, Vitória apresentou um 
perceptível crescimento em suas atividades econômicas e em seu porte empresarial 
entre os anos de 2002 a 2012 \cite{PrefeituraMunicipaldeVitoria-PMV2012}. Este crescimento pode ter sido 
resultado do crescimento da qualidade de vida e desenvolvimento verificado após o 2º 
ciclo de desenvolvimento econômico, ocorrido entre os anos de 1960 e 2000, alavancada 
pela produção de commodities e diversificação econômica no período \cite{Cacador2013}.\vspace*{0.3cm} \newline
Consequentemente, foi observado o aumento da produção de empreendimentos como 
edificações comerciais e residenciais \cite{SindicatodaIndustriadaConstrucaoCivildoEspiritoSanto-SINDUSCON2017}. Como resultado deste 
aumento de volume de produção e desenvolvimento econômico da cidade, ocorreu a 
necessidade de expansão da oferta de energia elétrica para suplantar estes eventos, 
assim como o progressivo aumento no consumo energético, que cresceu a uma taxa de 5,8\% 
ao ano \cite{Amarante2009}.\vspace*{0.3cm} \newline
Para a caracterização do contexto climático da capital do Espírito Santo, foram 
coletados dados climáticos por meio de arquivo climático contendo a série de medições 
meteorológicas de Vitória do ano de 2018 \cite{InstitutoNacionaldeMetereologia-INMET2018}. Estas informações são 
importantes dados de entrada para as simulações e análises termoenergéticas realizadas 
no âmbito desta pesquisa. Os dados coletados seguem apresentados na Tabela \ref{tab:tabela2}.\pagebreak

\begin{table}[ht]\centering
    \caption{\small Características climáticas de Vitória.}
    \vspace*{0.2cm}
    \label{tab:tabela2}
    \begin{tabular}{lc}
    \hline
    \textbf{Dados climático}                                  & \textbf{Média anual total}   \\ \hline
    Temperatura de bulbo seco (°C) - média mínima mensal      & 20,67                        \\ \hline
    Temperatura de bulbo seco (°C) - média máxima mensal      & 29,00                        \\ \hline
    Umidade relativa do ar (\%) - média diária mensal         & 76,92\%                      \\ \hline
    Direção dos ventos (graus) - média diária mensal          & 181,04                       \\ \hline
    Velocidade do vento (m/s) - média diária mensal           & 2,05                         \\ \hline
    Radiação horizontal global (Wh/m²) - média diária mensal  & 425,62                       \\ \hline
    Nebulosidade (\%) - média diária mensal                   & 69,08\%                      \\ \hline
    Iluminância horizontal global (lux) - média diária mensal & 50019,06                     \\ \hline
    \end{tabular}
    \begin{flushleft}
        \par \small Fonte: adaptado de INMET (2018).
    \end{flushleft}
\end{table}\vspace*{-0.3cm}
