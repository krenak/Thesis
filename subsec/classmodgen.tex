\subsection{Classificação de desempenho energético dos modelos genéricos}
\noindent Para a determinação da classe de eficiência energética dos modelos genéricos, segundo a metodologia descrita pelo INI-C, é exigida a comparação entre dois cenários definidos como modelo real e o modelo de referência. Desta forma, foram configurados os cenários para a identificação do consumo total de energia térmica, elétrica e de energia primária, como descrito no capítulo 3. Como resultado, foi constatada a baixa eficiência energética dos modelos, como exposto na Tabela 15.
