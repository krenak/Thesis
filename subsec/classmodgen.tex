\subsection{Classificação de desempenho energético dos modelos genéricos}
\noindent Para a determinação da classe de eficiência energética dos modelos genéricos, segundo a metodologia descrita pelo INI-C, é exigida a comparação entre dois cenários definidos como modelo real e o modelo de referência. Desta forma, foram configurados os cenários para a identificação do consumo total de energia térmica, elétrica e de energia primária, como descrito no capítulo 3. Como resultado, foi constatada a baixa eficiência energética dos modelos, como exposto na Tabela \ref{tab:tabela15}.
\begin{table}[H]
    \centering
    \small
    \caption{Resultados da classificação de desempenho energético dos modelos genéricos iniciais.}
\begin{tabular}{llll}
    \hline
    \textbf{Descrição}                                    & \textbf{Variável}      & \makecell[c]{\textbf{Modelo} \\\textbf{Real}} & \makecell[c]{\textbf{Modelo de} \\\textbf{Referência}}  \\ \hline
    \multicolumn{4}{c}{\textbf{Classe de desempenho energético - modelo de 8 pavimentos}}                                                  \\ \hline
    \makecell[l]{Consumo total de \\energia térmica}                      & CTEt (kWh/ano)         & 418197               & 443178                         \\
    \makecell[l]{Consumo total de \\energia elétrica - CTEe}              & CTEe (kWh/ano)         & 1086401              & 1142584                        \\
    \makecell[l]{Consumo de energia primária \\da edificação selecionada} & \makecell[l]{CEP real/ref \\(kWh/ano)} & 2198258,3            & 2315630,2                      \\ \hline
    \multicolumn{3}{l}{Etiqueta da edificação real}                                                       & \multicolumn{1}{c}{\textbf{D}} \\ \hline
    \multicolumn{4}{c}{\textbf{Classe de desempenho energético - modelo de 19 pavimentos}}                                                 \\ \hline
    \makecell[l]{Consumo total de \\energia térmica}                      & CTEt (kWh/ano)         & 1115708              & 1137261                        \\
    \makecell[l]{Consumo total de \\energia elétrica - CTEe}              & CTEe (kWh/ano)         & 2643097              & 2676556                        \\
    \makecell[l]{Consumo de energia primária \\da edificação selecionada} & \makecell[l]{CEP real/ref \\(kWh/ano)} & 5456234              & 5533476,7                      \\ \hline
    \multicolumn{3}{l}{Etiqueta da edificação real}                                                       & \multicolumn{1}{c}{\textbf{D}} \\ \hline
    \end{tabular}
    \begin{flushleft}
        \par \small \vspace{0.1cm} Fonte: autor (2019).
    \end{flushleft}
    \label{tab:tabela15}
\end{table}
\noindent O desempenho enérgico dos modelos ampara a necessidade de otimização, quanto as características construtivas e de sistemas das edificações, para tornar as condições de performance propícias ao balanço energético.